\section{10503-2 矩陣轉換}

\subsection{解題思惟}
\begin{enumerate}
	\item 這一題要處理二維矩陣轉換的問題,基本上要處理的有兩個轉換,一個是翻轉,另一個是旋轉。題目說A矩陣經由一些轉換得到B,那已知B和轉換的運算,要求出A。那我們可以從B矩陣經由相反的轉換過程把A找出來。也就是說,假如$T_i$代表第$i$個轉換,那我們知道$T_nT_{n-1}\cdots T_1(A)=B$,反過來說$A=T_1^{-1}T_2^{-1}\cdots T_n^{-1}(B)$。這邊$T_i$是給定的翻轉和旋轉兩種轉換之一,那翻轉的反轉換還是翻轉,旋轉的反轉換也是旋轉(不同方向),所以只要先把兩種轉換用函數實現出來,這一題就容易解答了。
	\item 要怎麼表達矩陣呢?基本上可以使用二維陣列,因為題目給定的R和C都不會超過10,所以宣告a[10][10]就足夠使用了。那我們宣告大一點,但實際上只使用R$\times$C或C$\times$R的大小就可以了。另外也可以使用一維陣列,自己處理二維元素和一維元素位置對應的問題,這樣也是可以的。如果讀者熟悉\cc{}裡面的向量(vector),也可以使用向量的向量來處理這個問題。
	\item 此處二維陣列連同其維度變數,都一併宣告為全域變數,然後撰寫翻轉和旋轉的函數來解題。翻轉的部份實作較容易,假設矩陣大小為$R\times C$,基本上是把矩陣的第$i$列和第$R-i$列對調。那旋轉的部份呢?稍微思考推導一下,可以知道逆時針轉90度,其實就等於轉置(transpose)加上翻轉。那轉置的話,是令$B_{ij}=A_{ji}$,這是很容易實作的。至於轉換的運算過程,可以用一個陣列存起來,最後倒過來讀取並施行於矩陣$B$,最後就可以得到矩陣$A$了。
	\item 原題目只有一筆測資,在高中生解題平台中的題目則改為多筆測資,讀到檔尾結束。如果只有一筆測資,處理上較為單純,先讀取$r, c, m$,接著讀取矩陣的$r\times c$個元素,以及$m$個運算元,再繼續進行解題。那如果是多筆測資的話,把讀取$r, c, m$的部份放到while的判斷式中,並把接下來的讀取和解題放在while迴圈中即可,如下所示。
	\begin{inside}
		while (cin >> r >> c >> m) {
			// 讀取矩陣及運算元並進行解題。
		}
	\end{inside}		
\end{enumerate}

\subsection{程式碼}
\begin{cppcode}
#include <iostream>

using namespace std;

int r, c, b[10][10], m, op[10000]; // 矩陣及運算元變數

void flip(); // 翻轉
void rotate(); // 旋轉
void transpose(); // 轉置

int main()
{
	while (cin >> r >> c >> m) { // 這部份假設多筆測資,
		for (int i=0; i<r; i++) { // 雙重迴圈讀取矩陣
			for (int j=0; j<c; j++) cin >> b[i][j];
		}
		for (int i=0; i<m; i++) cin >> op[i]; // 運算元
		for (int i=m-1; i>=0; i--) { // 運算次序倒過來
			if (op[i]==0) rotate(); // 逆時針旋轉90度
			if (op[i]==1) flip(); // 翻轉
		}
		cout << r << " " << c << endl; // 輸出維度
		for (int i=0; i<r; i++) { // 輸出矩陣
			for (int j=0; j<c; j++) {
				if (j) cout << " ";
				cout << b[i][j];
			}
			cout << endl;
		}
	}
	return 0;
}

void flip()
{
	for (int i=0; i<r/2; i++) { // 處理上半與下半的對調
		for (int j=0; j<c; j++) swap(b[i][j], b[r-1-i][j]);
	}
}

void transpose()
{
	int t = max(r, c); // 取r,c中較大者
	for (int i=0; i<t; i++) { // 取右上半部元素
		for (int j=i+1; j<t; j++) {
			swap(b[i][j], b[j][i]); // Bij 和 Bji 對調
		}
	}
	swap(r, c); // 維度對調
}

void rotate()
{
	transpose(); // 逆時旋轉90度等於轉置加上翻轉
	flip();
}	
\end{cppcode}	
