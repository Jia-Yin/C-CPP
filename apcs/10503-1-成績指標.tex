\section{10503-1 成績指標}

\subsection{解題思惟}
\begin{enumerate}
	\item 這一題首先要輸入n和n筆成績,這部份使用陣列來處理就可以了。
	\item 第二個部份是要印出排序好的成績,這部份可以自己寫排序的函數,例如氣泡排序法。如果要考慮到速度的話,那氣泡排序法的複雜度是$O(n^2)$,在n很大的時候,可能會有超時的問題,這時必須改用快速排序法等$O(n\log(n))$的演算法,不過這部份對初學的同學來說比較困難,但我們可以直接使用\cc{}裡面STL函數。基本上先引入<algorithm>檔頭,假設陣列名稱為a,且要排序的個數為n,直接呼叫std::sort(a, a+n)就可以排序了。
	\item 另外還要找出不及格的最高分,和及格的最低分的問題。這部份有兩種作法,一種是從排序好的數列中尋找,另一種是從原始數列中尋找。基本上後者就是從一堆數列中找最大值(或最小值)的方法,這部份對一般同學來說比較熟悉,基本概念就是逐一比過,有更大(或更小)的就替換掉就可以了,不過在比之前要先確定這個數目是不及格(或及格)才進行比較。
\end{enumerate}

\subsection{程式碼}
\begin{cppcode}
#include <iostream>
#include <algorithm>

using namespace std;

int main()
{
	int n, score[200], t, tmax, tmin;
	
	while (cin >> n) {
		tmax = -999;
		tmin = 999;
		for (int i=0; i<n; i++) {
			cin >> score[i];
			if (score[i]<60 && score[i]>tmax) tmax = score[i];
			if (score[i]>59 && score[i]<tmin) tmin = score[i];
		}
		sort(score, score+n);
		for (int i=0; i<n; i++) {
			if (i) cout << " ";
			cout << score[i];
		}
		cout << endl;
		if (tmax==-999) cout << "best case\n";
		else cout << tmax << endl;
		if (tmin==999) cout << "worst case\n";
		else cout << tmin << endl;
	}
	return 0;
}
\end{cppcode}