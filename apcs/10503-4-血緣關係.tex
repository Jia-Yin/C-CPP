\section{10503-4 血緣關係}

\subsection{解題思惟}
\begin{enumerate}
	\item 這一題比較複雜一點,提供兩種解答思路供作參考。
	\item 第一種思路,使用陣列處理。我們使用陣列parent,其第i個元素值表示第i個節點的父節點位置。另外我們還使用兩個陣列,一個陣列為leaf,用來儲存所有葉節點編號,一個陣列為isparent,記錄節點i是否為父節點。每次讀取一對父子節點時,就更新陣列parent及isparent的值。所有輸入讀取之後,利用isparent來找出所有葉節點並存入leaf陣列中。最後我們針對所有的葉節點進行巡訪,計算每個葉節點至最上層祖先的距離,以及任兩個葉節點之間的距離,找出其中的最大值即為答案。在計算兩葉節點之距離時,要先找出其共同祖先。所以我們撰寫了三個輔助函數,第一個函數計算從a節點上溯至b節點要幾步,如果到不了就回傳-1,這主要是用來輔助尋找共同祖先及計算距離使用的。第二個函數計算節點a至節點b的距離,其中會使用到第一個函數。最後一個函數用來計算某節點至最上面祖先節點的距離。詳細的程式碼及說明可參考程式碼一。
	\item 第二種思路是同時使用向量(vector)來儲存某節點的所有子節點,因為各節點的子節點數是不定的,使用向量會比較便利。接著找出所有節點的高度,其方法如下:針對每個葉節點,設其高度為0,並開始不斷上溯,每次上溯一步高度便加1,如果這時上溯的父節點已設有高度,便進行比較,看是否需更新,如不需更新,則此葉節點的更新任務完成,可繼續巡訪下一葉節點,如需更新,則更新之後,再繼續找更上面的父親節,並重覆同樣步驟直到更新任務完成,再繼續巡訪下一葉節點。當所有葉節點巡訪完成之後,即得到所有節點的高度。接下來要計算最遠兩節點可分兩種情況,第一種情況,某節點只有一子節點,那就計算它自己的高度(等於和最遠葉節點的距離),第二種情況,某節點有兩個以上子節點,那麼就找出所有子節點中高度最大的兩個,把兩個值加起來再加上2就等於兩個最深葉節點的距離了。所以兩種情況的最大值就是答案了。詳細的程式碼及說明可參考程式碼二。
	\item 以上兩個演算法中,第一個演算法較慢,當節點數較多的時候會超時,無法通過所有測資。第二個演算法較快,可以通過所有測資。
\end{enumerate}

\subsection{程式碼}
程式碼一:這個方法較慢,當節點數較多的時候會超時,無法通過所有測資。
\begin{cppcode}
#include <iostream>

using namespace std;

int n, parent[100005];

int fromAtoB(int a, int b); // 計算從a上溯到b要幾步,到不了就回傳-1
int disAtoB(int a, int b); // 計算a與b的距離
int disAtoRoot(int a); // 計算a到最上面的祖先之距離

int main()
{
	int n, a, b, leaf[100000], isparent[100000], lfcnt;
	while (cin >> n) {
		for (int i=0; i<n; i++) { // 啟始值設定
			parent[i] = -1;  // 預設其父節點為空
			isparent[i] = 0; // 預設都不是父節點
		}
		for (int i=0; i<n-1; i++) {
			cin >> a >> b;
			parent[b] = a; // 標記:a為b之父節點
			isparent[a] = 1; // a是父節點
		}
		// 找出所有葉節點
		lfcnt=0;
		for (int i=0; i<n; i++) {
			if (isparent[i]) continue;
			leaf[lfcnt++] = i;
		}
		int maxdis = 0, dis;
		for (int i=0; i<lfcnt; i++) { // 從每個葉節點尋找
			if (disAtoRoot(leaf[i]) > maxdis) { // 至祖先距離為最大值?
				maxdis = disAtoRoot(leaf[i]); // 更新目前最大距離
			}
			for (int j=i+1; j<lfcnt; j++) { // 計算任兩個葉節點之距離
				dis = disAtoB(leaf[i], leaf[j]);
				if (dis > maxdis) maxdis = dis; // 超過最大值的話要更新
			}
		}
		cout << maxdis << endl;
	}
	return 0;
}

int disAtoRoot(int a)
{
	int step = 0;
	while (parent[a]>=0) { step++; a=parent[a]; } // 上溯並更新步數
	return step;
}

int disAtoB(int a, int b)
{
	int common=parent[a], step=1; // common為共同祖先,從a父節點找起
	
	while (fromAtoB(b, common)<0) { // 如果b到不了common
		step++; // 步數更新
		common = parent[common]; // 繼續往上找共同祖先
	}
	return step + fromAtoB(b, common); // 祖先至兩邊的距離和
}

int fromAtoB(int a, int b) // a上溯到b的步數,到不了的話回傳-1
{
	int step, found=0; // 步數,是否到得
	for (step=1; parent[a]>=0; step++) { // a父節點存在?
		if (parent[a]==b) { found=1; break; } // 是否為b?
		a = parent[a]; // 不是的話,繼續往上
	}
	if (found) return step; // 找到回傳步數
	else return -1; // 找不到回傳-1
}
\end{cppcode}
程式碼二:同時使用向量來儲存某節點的所有子節點,以及用陣列儲存某節點的高度。此方法較快,可通過所有測資。
\begin{cppcode}
#include <iostream>
#include <vector>

using namespace std;

int main()
{
	int n, a, b;
	while (cin >> n) {
		int high[100005]={0}, parent[100005];
		vector<int> node[100005], leaves;
		for (int i=0; i<n; i++) parent[i] = -1; // 所有節點之父為空
		for (int i=0; i<n-1; i++) {
			cin >> a >> b;
			node[a].push_back(b); // a節點之子節點加入b
			parent[b] = a; // b之父為a
		}
		// Find leaves
		for (int i=0; i<n; i++) { // 子節點數為0者即為葉節點
			if (node[i].size()==0) leaves.push_back(i);
		}
		// Determine height for each node
		for (int i=0; i<leaves.size(); i++) { // 使用每一個葉節點
			int num = leaves[i]; // num為葉節點編號
			int pnum = parent[num]; // pnum為父節點編號
			while (pnum>=0) { // 父節點非空,高=max(子高度+1,已設高度)
				if (high[num]+1 <= high[pnum]) break; // 不需更新
				high[pnum] = high[num] + 1; // 更新父節點高度
				num = pnum; // 這兩列更新目前節點及父節點編號值
				pnum = parent[num];
			}
		}
		// 最大距離情況有二:1)某點至最深葉節點,2)某點的兩個最深葉節點
		int maxdis = 0;
		for (int i=0; i<n; i++) { // 巡訪每個節點
			if (node[i].size()==0) continue; // 無子節點,跳過
			if (node[i].size()==1) { // 只有一葉節點,檢查情況1
				if (high[i] > maxdis) maxdis = high[i];
				continue;
			}
			// 以下為情況2,需找出兩最深葉節點
			int maxd1=0, maxd2=0, d; 
			for (int j=0; j<node[i].size(); j++) { // 巡訪所有子節點
				d = high[node[i][j]]; // 計算其高度
				if (d>maxd1) { maxd2=maxd1; maxd1=d; } // 更新二最大值
				else if (d>maxd2) maxd2=d; // 更新次大值
			}
			// 以下為情況2之距離,比較看是否為最大值
			if (maxd1+maxd2+2>maxdis) maxdis = maxd1+maxd2+2;
		}
		cout << maxdis << endl;
	}
	return 0;
}


// 以下程式只得 40% ,在 N 可以到2000的時候,測資點 < 10M,超時

\end{cppcode}	
