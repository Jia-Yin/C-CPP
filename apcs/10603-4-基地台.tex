\section{10603-4 基地台}

\subsection{解題思惟}
\begin{enumerate}
	\item 這題要求最小直徑d,因為是整數,那我們試著將d從1開始往上測試。
	\item 先把服務點排序,假設最左邊的點座標為x,要服務到這個點且基地台直徑為d的話,最好的情況是增設一個基地台在x+d/2,這樣的話,最右邊可服務到x+d。我們可以跳過所有可被服務到的點。
	\item 接下來如果有尚未被服務到的點,就找出下一個,假設位置為y,那我們可以像上一步一樣增設一個基地台在y+d/2,然後跳過所有可被服務的點(位置小於等於y+d)。
	\item 依此類推,一直到最後所有的點都可以被服務到,或基地台數量大於k。
	\item 檢查所有增設基地台的數量,如果小於等於k的話,那這個直徑就是我們的答案,所以用break跳出迴圈並印出結果。如果不是,把直徑增加一個最小的時間單位,然後重覆第2個步驟到上個步驟。
\end{enumerate}

\subsection{程式碼}
\begin{cppcode}
#include <iostream>
#include <algorithm>

using namespace std;

int main()
{
	int n, d, k, p[1000];
	cin >> n >> k;
	for (int i=0; i<n; i++) cin >> p[i]; // 輸入服務點
	sort(p, p+n); // 排序
	for (d=1; ; d++) { // 從1開始測試直徑
		int r=0, cnt=0, idx=0; // r=目前可服務到的範圍
		while (r<p[n-1]) { // 尚有未服務的點
			cnt++; // 新增一個基地台
			if (cnt>k) break;
			r=p[idx]+d; // 基地台放p[idx]+d/2,可服務到p[idx]+d
			while (idx<n-1 && p[idx+1]<=r) idx++; // 反覆移到下一個可服務的點
			idx++; // 再移到下一個(不能被服務到的)點
		}
		if (cnt<=k) break; // 如果總部不超過k即找到答案
	}
	cout << d << endl;
	return 0;
}
\end{cppcode}
