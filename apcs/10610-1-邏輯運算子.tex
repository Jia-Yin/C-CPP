\section{10610-1 邏輯運算子}

\subsection{解題思惟}
\begin{enumerate}
	\item 在C/\cc{}中,做邏輯判斷時,除了給判斷式,也可以給數值,給數值的時候,0會看成是false,而非0的數則看成true。
	\item 題目所給的AND運算,如果把得到的值c=1當做true,c=0當做false,那麼就可以把它看成是邏輯運算的AND(\&\&)。怎麼把c=1變成true,c=0變成false呢?只要使用c==1的判斷式就可以了。所以給定a, b, c之後,要看AND成立與否,可以用下列的式子來判斷
	\begin{inside}
		if ((a && b) == (c==1)) cout << "AND" << endl;
	\end{inside}
	\item 同樣的道理,題目所給的OR運算,把得到的值1當做true,0當做false,那麼就可以把它看成是邏輯運算的OR(||)。所以給定a, b, c之後,要看OR成立與否,可以用下列的式子來判斷
	\begin{inside}
	if ((a || b) == (c==1)) cout << "OR" << endl;
	\end{inside}
	\item 題目所給的XOR運算,把得到的值1當做true,0當做false,那麼可以把它看成是邏輯運算的(a\&\&!b) || (!a\&\&b)。所以給定a, b, c之後,要看OR成立與否,可以用下列的式子來判斷
	\begin{inside}
	if (((a && !b) || (!a && b)) == (c==1)) cout << "XOR" << endl;
	\end{inside}
	\item 以上三個式子可以針對給定的a, b, c三數,分別判斷AND, OR, 及XOR是否成立。但題目說,如果三個都不成立的話,要輸出IMPOSSIBLE,那這個部份,在程式中可以設定一個旗標來處理,如果上面三個式子,任一個被執行了,就設定旗標,如果最後發現旗標還沒有被設定,就輸出IMPOSSIBLE。
\end{enumerate}

\subsection{程式碼}
\begin{cppcode}
#include <iostream>

using namespace std;

int main()
{
	int a, b, c, found=0;
	cin >> a >> b >> c;
	if ((a && b) == (c==1)) { cout << "AND" << endl; found = 1; }
	if ((a || b) == (c==1)) { cout << "OR" << endl; found = 1; }
	if (((a && !b) || (!a && b)) == (c==1)) { cout << "XOR" << endl; found = 1; }
	if (!found) cout << "IMPOSSIBLE" << endl;
	return 0;
}

\end{cppcode}