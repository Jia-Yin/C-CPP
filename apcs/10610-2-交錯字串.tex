\section{10610-2 交錯字串}

\subsection{解題思惟}
\begin{enumerate}
	\item 這個題目針對輸入的字串,只要判斷大小寫就好了,至於字母是什麼其實並不重要。另外要處理的是大小寫交錯長度的問題,那如果我們把輸入轉換成交錯的連續大小寫長度陣列,應該會有幫助,例如字串``aafAXbbCDCCC",有3個連續小寫,2個連續大寫,2個連續小寫,5個連續大寫,也就是把它轉換成\{3,2,2,5\}四個數的陣列。這樣的話後對於判斷最長連續交錯長度會有很大幫助。
	\item 怎麼判斷大小寫呢?這個簡單,寫個函數來處理就好了。
	\begin{inside}
	int capital(char c)
	{
		if (c>='A' && c<='Z') return 1;
		return 0;
	}    
	\end{inside}
	\item 怎麼把給定的字串轉換成交錯的連續大小寫長度陣列呢?基本上就是看目前的字元是否和前一個字元的大小寫相同,如果相同,要持續累加個數,如果不同,就儲存目前的長度,然後記錄新的大小寫狀態和新的長度1。可以用以下的程式碼來實現:
	\begin{inside}
	int curcase, precase = capital(str[0]); // 記錄起始的大小寫狀態
	int cnt = 1; // 起始的連續長度為1
	for (int i=1; str[i]; i++) { // str[i]非0表示還有字元
		curcase = capital(str[i]); // 取得目前字元大小寫狀態
		if (curcase == precase) cnt++; // 如果和之前的大小寫相同則將長度加1
		else { // 不同的話將目前長度存起來,並重設連續長度
			h[idx++] = cnt;
			cnt = 1;
		}
		precase = curcase; // 儲存目前大小寫狀態
	}
	h[idx++] = cnt; // 儲存最後一組的長度
	\end{inside}
	\item 有了交錯的連續大小寫長度陣列h[]之後,怎麼計算最長的連續交錯長度呢?我們可以先寫一個函數,用來計算從h的某一個位置開始的最長交錯長度。程式碼如下:
	\begin{inside}
	int alen(int pos, int len) // pos是目前位置,len是陣列長度
	{
		if (h[pos] < k) return 0; // 找不到長度為k的連續大小寫
		int sum = k; // 找到了,長度至少為k
		pos++; // 看下一個數。下面的while:如果是k,就遞增位置,並持續加k
		while (pos<len && h[pos]==k) { sum += k; pos++; }
		if (pos<len && h[pos]>k) sum += k; // 最後如果大於k,還要加k
		return sum;
	}
	\end{inside}
	\item 有了alen函數之後,接下來就簡單了,使用alen函數,把位置從頭到尾算一次,看哪個地方開始的值最大就好了。
\end{enumerate}

\subsection{程式碼}
\begin{cppcode}
#include <iostream>

using namespace std;


int k, h[100005]={0}, idx=0;

int capital(char c);
int alen(int pos, int len);

int main()
{
	char str[100005];
	cin >> k >> str;
	int curcase, precase = capital(str[0]);
	int cnt = 1;
	for (int i=1; str[i]; i++) {
		curcase = capital(str[i]);
		if (curcase == precase) cnt++;
		else {
			h[idx++] = cnt;
			cnt = 1;
		}
		precase = curcase;
	}
	h[idx++] = cnt;
	
	int maxlen = 0;
	for (int i=0; i<idx; i++) {
		int len = alen(i, idx);
		if (len>maxlen) maxlen = len;
	}
	cout << maxlen << endl;
	return 0;
}

int capital(char c)
{
	if (c>='A' && c<='Z') return 1;
	return 0;
}    

int alen(int pos, int len)
{
	if (h[pos] < k) return 0;
	int sum = k;
	pos++;
	while (pos<len && h[pos]==k) { sum += k; pos++; }
	if (pos<len && h[pos]>k) sum += k;
	return sum;
}
\end{cppcode}

\subsection{另解}
這一題也可以把大寫字元改成1,小寫字元改成0,然後將輸入的01陣列的「部份和陣列」求出來($S_n=\sum_{i=1}^n$)。有了$S_n$之後,要計算位置$p$到位置$q$有幾個大寫字元,只要使用$S_q-S_{p-1}$就可以求得。那麼要檢查位置$p$開始是否有連續$k$個大寫,可以使用$S_{p+k-1}-S_{p-1}==k$來求得,要計算是否有連續$k$個小寫,則可以使用$S_{p+k-1}-S_{p-1}==0$來求得。使用這樣的方法,也可以很快速地求出解答。稍微複雜一點,有興趣的同學可以思考看看,程式碼如下:
\begin{cppcode}
#include <iostream>
#include <cstring>

using namespace std;

int slen, h[100005]={0};

int alen(int start, int k);

int main()
{
	char str[100005];
	int k;
	cin >> k >> str;
	slen = strlen(str);
	for (int i=0; i<slen; i++) {
		if (str[i]>='a' && str[i]<='z') h[i+1] = h[i];
		if (str[i]>='A' && str[i]<='Z') h[i+1] = h[i] + 1;
	}
	
	int maxlen = 0;
	for (int i=1; i<=slen; i++) {
		int len = alen(i, k);
		if (len > maxlen) maxlen = len;
	}
	cout << maxlen << endl;
	return 0;
}

int alen(int start, int k)
{
	int last = start+k-1;
	if (last>slen) return 0;
	int sum = h[last] - h[start-1];
	if ((sum != k) && (sum != 0)) return 0;
	sum = k - sum;
	while (last+k<=slen && h[last+k]-h[last]==sum) {
		last += k;
		sum = k - sum;
	}
	return last-start+1;
}
\end{cppcode}	