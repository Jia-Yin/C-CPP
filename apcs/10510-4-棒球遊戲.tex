\section{10510-4 棒球遊戲}

\subsection{解題思惟}
\begin{enumerate}
	\item 這題看起來覺得很難,但實際上仔細分析,會發現其實並不難,只是比較繁雜一點。
	\item 輸入的訊息用二維陣列儲存,因為每個事件的第一個字母都不相同,讀取之後存第一個字母就可以了,判斷上也會比較方便。
	\item 儲存輸入事件之後,依次取出判斷並更新狀態。基本上要更新的狀態包括各壘的情況、本局出局人數、總共出局人數、以及得分等資訊。
	\item 各壘的情況可用一個變數,以各位元代表各壘狀況,位元0表示一壘,位元1表示二壘,位元2表示三壘,那麼如果一三壘有人,這個數就是0b101=5。這樣表示的好處在哪裡呢?如果接下來擊出二壘打,就把所有位元往左移二個位元(或乘以4),那移完之後變成0b10100=20,第三個位元以後都表示得分,所以得分為0b10=2,或者20/8=2。此外第2個位元為1,表示三壘有人。所以出現安打的時候,只要處理位移,之後要計分或更新壘上的狀況都會很容易。
\end{enumerate}

\subsection{程式碼}
\begin{cppcode}
#include <iostream>

using namespace std;

int main()
{
	char event[9][5], ss[3];
	int k;
	for (int r=0; r<9; r++) {
		cin >> k; // 每位打手事件次數
		for (int i=0; i<k; i++) {
			cin >> ss; // 讀取事件
			info[r][i] = ss[0]; // 存第一個字元
		}
	}
	cin >> k; // 總出局人數
	
	int status = 0; // 各壘狀態
	int outplayer = 0; // 各局出局人數 
	int totalout = 0; // 總出局人數
	int score = 0; // 得分
	
	for (int i=0; i<5; i++) { // 事件最多五回合
		for (int r=0; r<9; r++) { // 打者順序
			switch (info[r][i]) { // 取出事件
				case '1': status = (status<<1) + 1; break; // 一壘打
				case '2': status = (status<<2) + 2; break; // 二壘打
				case '3': status = (status<<3) + 4; break; // 三壘打
				case 'H': status = (status<<4) + 8; break; // 全壘打
				case 'S': // 出局,繼續往下
				case 'F': // 出局,繼續往下
				case 'G': totalout++; outplayer++; break; // 出局
			}
			int sf = status >> 3; // 得分狀態
			status &= 7; // 更新壘上狀態
			while (sf) { // 計算得分狀態有幾個1,表示得幾分
				score += sf & 1;
				sf >>= 1;
			}
			if (totalout==k) { // 檢查總出局人數是否到達
				cout << score << endl; // 輸出積分
				return 0;
			}
			if (outplayer==3) { // 檢查該局出局人數是否達到3
				status = 0; // 重設狀態
				outplayer = 0; // 重設出局人數
			}
		}
	}
	
	return 0;
}
\end{cppcode}
