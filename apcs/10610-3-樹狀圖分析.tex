\section{10610-3 樹狀圖分析}

\subsection{解題思惟}
\begin{enumerate}
	\item 一般儲存樹狀結構多使用指標,但我們也可以設法使用陣列來處理。基本上每個節點最多只有一個父節點,那這個題目共有n個節點,我們可以用一個陣列來表示,其中第i個元素($i\le 1\le n$)的值表示其父節點的位置;如果沒有父節點的話,就把它的值設為0。我們可以用下列程式來建構樹狀圖:
	\begin{inside}
		int n, k, node[100005]={0};
		cin >> n;
		for (int nd=1; nd<=n; nd++) {
			cin >> k;
			for (int i=0; i<k; i++) { // read child
				cin >> child;
				node[child] = nd; // connection
			}
		}
	\end{inside}
	\item 這個題目要計算每個節點的高度,這個部份怎麼處理呢?如果我們把葉節點存起來,它們的高度都是0,那麼從葉節點依次不斷尋找父節點,每次高度都增加1。應該注意的是,有些節點可能有超過一個以上的子節點,如果在往上尋找父節點的過程中,遇到一個父節點,其高度值已經存在,並且大於目前遞增的高度值,那這個遞增的高度值就沒有什麼用處了(因為高度會取最大的可能值)。假設要處理的葉節點為curnode,節點i的高度為h[i],以下程式碼可以處理往上找父節點遞增高度的問題:
	\begin{inside}
	int pnode = node[curnode];
	for (int lvl=1; pnode; lvl++) {
		if (h[pnode] >= lvl) break;
		h[pnode] = lvl;
		pnode = node[pnode];
	}
	\end{inside}
	\item 那要怎麼找到葉節點呢?基本上在讀入樹狀結構時,會讀取每個節點的子節點,如果其數目為0,就表示這個節點是葉節點,可以用以下程式碼處理:
	\begin{inside}
	int zeros[100000], zidx=0;
	for (int nd=1; nd<=n; nd++) {
		cin >> k;
		if (!k) zeros[zidx++] = nd; // k==0 ==> Add to zeros array
	}
	\end{inside}
	\item 這個題目還要找根節點,要怎麼找根節點呢?基本上當樹狀結構用陣列建構好之後,每個位置都會儲存它的父節點的位置,如果找到一個位置i,其儲存的值為0,就表示它沒有父節點,換句話說,i就是根節點,程式碼如下:
	\begin{inside}
	int root;
	for (int i=1; i<=n; i++) {
		if (!node[i]) root = i;
	}
	\end{inside}
	\item 有了以上各個部份的處理,最後只要把所有的節點的高度加起來就可以了。但要注意的是,這個題目給的節點數可能高達100000,如果整個串成一直線的話,它們的高度和會超過整數的最大值。所以求和的時候,要宣告成long long才不會發生溢位的問題。
\end{enumerate}

\subsection{程式碼}
\begin{cppcode}
#include <iostream>

using namespace std;

int n, node[100005]={0}, h[100005]={0}, zeros[100000], zidx=0;

int main()
{
	cin >> n;
	int k, child, level, pnode, curnode;
	for (int nd=1; nd<=n; nd++) {
		cin >> k;
		if (!k) zeros[zidx++] = nd;
		else for (int i=0; i<k; i++) { // read child
			cin >> child;
			node[child] = nd; // connection
		}
	}
	
	for (int i=0; i<zidx; i++) {
		curnode = zeros[i];
		pnode = node[curnode];
		for (int lvl=1; pnode; lvl++) {
			if (h[pnode] >= lvl) break;
			h[pnode] = lvl;
			pnode = node[pnode];
		}
	}
	
	int root;
	long long sum=0;
	for (int i=1; i<=n; i++) {
		if (!node[i]) root = i;
		sum += h[i];
	}
	cout << root << endl << sum << endl;
	return 0;
}
\end{cppcode}