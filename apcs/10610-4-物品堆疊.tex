\section{10610-4 物品堆疊}

\subsection{解題思惟}
\begin{enumerate}
	\item 這個題目看起來似乎有點複雜,但仔細想一想,會發現一個很簡單的思路:我們先考慮一個特定的排列順序$T_1T_2\cdots T_N$ (從上到下),其中$T_1, T_2\cdots,T_N$為$1,2,\cdots N$的重排。現在只考慮$T_iT_{i+1}$的順序,如果我們把它們對調過來,會不會變得更好呢?
	\item 對調過來的話,$T_i$變到$T_{i+1}$的下面,這對於$T_{i+1}T_i$兩者上面的物品,取用所需的能量是不受影響的,對於$T_{i+1}T_i$兩者以下的物品,也不會有影響。但是對$T_i$來說,現在多了一個$T_{i+1}$在上面,所以要搬動的時候,要多搬動$W(T_{i+1})$的重量,另一方面,$T_{i+1}$現在跑到$T_i$上面了,所以搬動的時候,可以省下$W(T_i)$的重量,把兩者的搬動次數也考慮進去,對調後增加的能量為$W(T_{i+1})F(T_i)-W(T_i)F(T_{i+1})$。
	\item 如果上面增加的能量為負的,那表示我們把$T_iT_{i+1}$對調,會得到更省能量的方式。也就是說,對於一個最好的排列順序$T_1T_2\cdots T_N$來說,必須有$W(T_{i+1})F(T_i)-W(T_i)F(T_{i+1})\ge 0$,或者說,
	$$W(T_{i+1})/F(T_{i+1}) \ge W(T_i)/F(T_i)$$
	把$T_1\cdots T_N$全部考慮進去,會得到以下的不等式:
	$$W(T_1)/F(T_1)\le W(T_2)/F(T_2)\le\cdots\le W(T_N)/F(T_N)$$
	\item 所以這一題等於透過排序就可以把最好的順序找出來,要排的數目字是每個物品的重量除以其搬動次數,從小到大。
	\item 這一題要拿到全部的分數,有兩個地方必須注意:1)物品的個數可能高達100000,如果我們使用氣泡排序法等$O(n^2)$複雜度的排序法,會超時。這個部份可以直接呼叫std::sort函數,其複雜度為$O(n\log(n))$。2)計算總共需花費的能量時,因為每個物品重量可能高達1000,如果使用整數的話,也可能產生溢位的問題,所以用來計算總能量的變數必須宣告為long long才不會發生溢位的問題。
	\item 這邊在使用std::sort排序函數時,要排的數目是$1,2,\cdots N$,但是排序的標準是使用要排序的數字的函數($W_i/F_i$),所以使用了三個參數的版本,其中第三個參數是用來決定兩數大小的函數,基本上是回傳一個布林值,當順序正確時要回傳true。
\end{enumerate}

\subsection{程式碼}
\begin{cppcode}
#include <iostream>
#include <algorithm>

using namespace std;

int weight[100005], freq[100005], order[100005];

bool myfunction (int i, int j) { return weight[i]*freq[j] < weight[j]*freq[i]; }

int main()
{
	int n;
	cin >> n;
	for (int i=0; i<n; i++) cin >> weight[i];
	for (int i=0; i<n; i++) cin >> freq[i];
	for (int i=0; i<n; i++) order[i] = i;
	
	sort(order, order+n, myfunction);
	
	long long upweight=0, sum=0;
	for (int i=1; i<n; i++) {
		upweight += weight[order[i-1]];
		sum += upweight * freq[order[i]];
	}
	cout << sum << endl;
	return 0;
}
\end{cppcode}

如果學過結構的話,也可以把上面的程式碼修改如下:

\begin{cppcode}
#include <iostream>
#include <algorithm>

using namespace std;

struct item {
	long long weight;
	long long freq;
};

item items[100005];

bool myfunction (item i, item j) { return i.weight*j.freq < j.weight*i.freq; }

int main()
{
	int n;
	cin >> n;
	for (int i=0; i<n; i++) cin >> items[i].weight;
	for (int i=0; i<n; i++) cin >> items[i].freq;
	
	sort(items, items+n, myfunction);
	
	long long upweight=0, sum=0;
	for (int i=1; i<n; i++) {
		upweight += items[i-1].weight;
		sum += upweight * items[i].freq;
	}
	cout << sum << endl;
	return 0;
}
\end{cppcode}
	