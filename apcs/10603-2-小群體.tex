\section{10603-2 小群體}

\subsection{解題思惟}
\begin{enumerate}
	\item 先用陣列把N個人的好友編號存起來。
	\item 接下來從陣列最前面開始,依序找每個人的好友,找到好友編號之後,接下來再繼續找他的好友,這樣反覆讀取,都是同一個群組的,那找到什麼時候停止呢?
	\item 我們可以另外設一個陣列,來表示某個人是否曾經被找過。一開始把這個陣列全設為0,那如果某個人被找過了,就把他的位置改成1。這樣一來,上面的尋找過程,只要碰到被找過的人,就可以停止尋找了。另外在上面依序處理的過程中,被找過的人也不用再處理,可以直接跳過。
	\item 用另外一個變數計算總共有幾個群組就好了。
\end{enumerate}

\subsection{程式碼}
\begin{cppcode}
#include <iostream>

using namespace std;

int main()
{
	int gf[50005], n, mark[50005]={0}, group=0;
	cin >> n;
	for (int i=0; i<n; i++) cin >> gf[i];
	for (int i=0; i<n; i++) {
		if (mark[i]) continue; // 這個人被找過了,跳過
		group++; // 新增一個群組
		do {
			mark[i]=1; i=gf[i]; // 標記該人,並繼續找他的好友
		} while (mark[i]==0); // 如果好友還沒被找過,繼續處理
	}
	cout << group << endl;
	return 0;
}
\end{cppcode}