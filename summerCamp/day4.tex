
\section{8/17(四)上午:遞迴}

\subsection{遞迴}
\begin{enumerate}
	\item 講解:JA-008:遞迴解1+2+...+n
		\begin{enumerate}
			\item 題目說明:
			\subitem 使用遞迴方式算出 1+2+...+n
			
			\item 解題思維:
			\subitem 假設$f(n)=1+2+...+n$,則遞迴的計算方法為$f(n)=n+f(n-1)$。
			
			\item 程式碼:
			\begin{cppcode}
				#include <cstdio>
				
				int f(int n);
				
				int main()
				{
					int n;
					scanf("%d", &n);
					printf("%d", f(n));
					return 0;
				}
				
				int f(int n)
				{
					if (n==1) return 1;
					return n + f(n-1);
				}
								
			\end{cppcode}
		\end{enumerate}
	
	\item 練習:A059:遞迴計算n階乘
		\begin{enumerate}
			\item 題目說明:
			\subitem 輸入一正整數N,輸出N!。其中$N! = 1\times2\times3\times...\times N$
			
			\item 解題思維:
			\subitem 假設函數$fact(n)=n! $,其遞迴的計算方式為$fact(n)=n\times fact(n-1)$。
			
			\item 程式碼:
			\begin{cppcode}
				#include<iostream>
				using namespace std;
				
				int fact(int n);

				int main()
				{
					int n;
					cin >> n;
					cout << fact(n);
					return 0;
				}

				int fact(int n)
				{
					if (n) return n * fact(n-1);
					else return 1;
				}

			\end{cppcode}
		\end{enumerate}
	
	\item 講解:A029︰費式數列
		\begin{enumerate}
			\item 題目說明:
			\subitem 費氏數列定義如下 $f(0)=0, f(1)=1, f(n)=f(n-1)+f(n-2)$。
			題目是從螢幕輸入一個正整數 n,輸出 $f(n)$。
			
			\item 解題思維:
			\begin{enumerate}
				\item 本題可用遞迴或非遞迴方式計算。
				\item 因為程式簡明易了,可直接觀看程式碼尋求理解。
			\end{enumerate}
			
			\item 程式碼:
			\begin{cppcode}
				#include <stdio.h>

				int f(int n);

				int main()
				{
					int n;
					scanf("%d", &n);
					printf("%d", f(n));
					return 0;
				}

				int f(int n)
				{
					if (n<2) return n;
					return f(n-1)+f(n-2);
				}
			\end{cppcode}
		\end{enumerate}
	
	\item 挑戰:JA-009:爬樓梯有幾種爬法
		\begin{enumerate}
			\item 題目說明:
			\subitem 小明爬樓梯,已知要爬的梯數有N階,但小明一次可以爬1~3階,請問總共有幾種爬法?
			
			\item 解題思維:
			\subitem
			當n=1, 2, 3時,分別有1, 2, 4種爬法,當n>3時,爬樓梯的方法為$f(n)=f(n-1)+f(n-2)+f(n-3)$。
			
			\item 程式碼:
			\begin{cppcode}
				#include <cstdio>
				
				int f(int n);
				
				int main()
				{
					int n;
					scanf("%d", &n);
					printf("%d", f(n));
					return 0;
				}
				
				int f(int n)
				{
					if (n==1) return 1;
					if (n==2) return 2;
					if (n==3) return 4;
					return f(n-1)+f(n-2)+f(n-3);
				}
			\end{cppcode}
		\end{enumerate}
	
	\item 講解:JB-04︰河內塔
	\begin{enumerate}
		\item 題目說明:
		\subitem 依課堂上講解之河內塔規則,從柱1移到柱3,柱2為輔助。輸入環的個數n,輸出所有移動過程。
		
		\item 解題思維:
		\subitem 河內塔的解法如下:
		\begin{enumerate}
			\item 當只有1個環的時候,直接把環搬到目標柱子上。
			\item 當有n個環的時候
			\begin{enumerate}
				\item 先將(n-1)層的河內塔搬到輔助的柱子上。
				\item 接著將第n個環搬到目標柱子上。
				\item 最後,再將(n-1)層的河內塔從輔助的柱子搬到目標柱子上。 
			\end{enumerate}
			
		\end{enumerate}
		
		\item 程式碼:
		\begin{cppcode}
			#include <iostream>
			
			using namespace std;
			
			void hanoi(int n, int from, int to, int buf);
			
			int main()
			{
				int n;
				cin >> n;
				hanoi(n, 1, 3, 2);
				return 0;
			}
			
			void hanoi(int n, int from, int to, int buf)
			{
				if (n==1) {
					cout << from << " => " << to << endl;
				} else {
				hanoi(n-1, from, buf, to);
				cout << from << " => " << to << endl;
				hanoi(n-1, buf, to, from);
			}
		}
	\end{cppcode}
\end{enumerate}

	
	
	
	
\end{enumerate}

\section{8/17(四)下午:陣列}

\subsection{陣列}
\begin{enumerate}
	\item 講解:A030︰百數反印
		\begin{enumerate}
			\item 題目說明:
			\subitem 輸入 100 個正整數,反向印出此 100 個數。
			
			\item 解題思維:
			\subitem 本題使用陣列儲存 100 個數,再反向印出即可,是很基本的題目。
			
			\item 程式碼:
			\begin{cppcode}
				#include<iostream>
				using namespace std;
				int main()
				{
					int data[100];
					for (int i=0; i<100; i++) cin >> data[i];
					for (int i=99; i>=0; i--) cout << " " << data[i];
					return 0;
				}
			\end{cppcode}
		\end{enumerate}
	
	\item 講解︰JT-30︰排序
		\begin{enumerate}
			\item 題目說明:
			\subitem 輸入N及N個數 (N<100),將N個數從小到大印出來。
			
			\item 解題思維:
			\subitem 本題是練習排序的演算法。基本上排序的演算法很多,以下程式使用\href{https://zh.wikipedia.org/wiki/%E5%86%92%E6%B3%A1%E6%8E%92%E5%BA%8F}{氣泡排序法},
			這也是最基本的排序演算法之一。在程式中,i 的範圍可以從 0
			到 n-2,或則倒過來從 n-1 到 1 也可以,基本上就是要執行 n-1 輪的意思,但是 i 的
			範圍寫法不同,j 的上限寫法也跟著(有)一些變化,這是在閱讀參考連結時,應注意的地
			方。
			
			\item 程式碼:
			\begin{cppcode}
				#include <stdio.h>
			
				int main()
				{
					int i, j, t, n, a[100];
					scanf("%d", &n);
					for (i=0; i<n; i++) scanf("%d", a+i);
					for (i=n-1; i>0; i--) {
						for (j=0; j<i; j++) {
							if (a[j]>a[j+1]) {
								t=a[j];
								a[j]=a[j+1];
								a[j+1]=t;
							}
						}
					}
					printf("%d", a[0]);
					for (i=1; i<n; i++) printf(" %d", a[i]);
					return 0;
				}
			\end{cppcode}
		\end{enumerate}
	
\end{enumerate}

