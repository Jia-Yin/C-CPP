\section{秘密差}

將一個十進位正整數的奇數位數的和稱為$A$,偶數位數的和稱為$B$,則A與B的絕對差值$|A-B|$稱為這個正整數的秘密差。

例如:263541的奇數位數的和$A=6+5+1=12$,偶數位數的和$B=2+3+4=9$,所以263541的秘密差是$|12-9|=3$。

給定一個十進位正整數$X$,請找出$X$的秘密差。

\subsection{解題思惟}
\begin{enumerate}
	\item 如何取出一個整數的個位數呢?答:把它除以10求餘數就可了。
	\item 如何將一個整數的個位數去掉呢?答:把它除以10所得的商就是了。
	\item 如何取出一個整數的每個位數呢?答:
	\begin{enumerate}
		\item 取出它的個位。
		\item 去掉它的個位。
		\item 如果不是0的話,回到(a)。
	\end{enumerate}
	\item 如何區別這個位數是奇數位還是偶數位呢?答:設一個整數旗標flag=1,每次取一個位數的時候,讓它在1和0之間切換(設flag=1-flag)。
\end{enumerate}

\subsection{程式碼}
\begin{cppcode}
#include <iostream>

using namespace std;

int main()
{
    int n, oddsum=0, evensum=0, flag=1;
    cin >> n;
    while (n) {
        if (flag) oddsum += n%10;
        else evensum += n%10;
        n /= 10;
        flag = 1-flag;
    }
    n = oddsum-evensum;
    if (n<0) n = -n;
    cout << n << endl;
	return 0;
}
\end{cppcode}