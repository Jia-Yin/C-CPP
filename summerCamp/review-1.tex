\documentclass[a4paper,12pt]{article}
%\usepackage[utf8x]{inputenc}
\usepackage{fontspec}
\usepackage{xeCJK}
\usepackage{xcolor,colortbl}
\setCJKmainfont{DFMing-Md-HK-BF.ttf}
%% Language and font encodings
%\usepackage[english]{babel}
%\usepackage[T1]{fontenc}
\usepackage{minted}
\usepackage{indentfirst}
%% Sets page size and margins
\usepackage[a4paper,top=2.5cm,bottom=2.5cm,left=2.5cm,right=2.5cm,marginparwidth=0.5cm]{geometry}
\usepackage{hyperref}
\usepackage{comment}
\renewcommand{\figurename}{圖}
\renewcommand{\figureautorefname}{圖}
\renewcommand{\tablename}{表}
\renewcommand{\tableautorefname}{表}
\setlength{\parindent}{2em}
\setlength{\parskip}{0.5em}
\renewcommand{\baselinestretch}{1.4}
%\usepackage[explicit]{titlesec}
%\usepackage{tikz}
%\usetikzlibrary{shapes,shadows,calc}
%\usepackage{lipsum}

%\definecolor{visgreen}{rgb}{0.733, 0.776, 0}

% the tikz picture that will be used for the title formatting
% \SecTitle{<signal direction>}{<node anchor>}{<node horiz, shift>}{<node x position>}{#5}
% the fifth argument will be used by \titleformat to write the section title using #1
%\newcommand\SecTitle[5]{%
%\begin{tikzpicture}[overlay,every node/.style={signal, draw, text=white, signal to=nowhere}]
%  \node[visgreen,fill, signal to=#1, inner sep=0.5em,
%    text=white,font=\Large\sffamily,anchor=#2,
%    xshift=\the\dimexpr-\marginparwidth-\marginparsep-#3\relax] 
%    at (#4,0) {#5};
%\end{tikzpicture}%
%}
%\titleformat{name=\section,page=odd}
%{\normalfont}{}{0em}
%{\SecTitle{east}{west}{16pt}{0}{#1}}[\addvspace{4ex}]

%\titleformat{name=\section,page=even}
%{\normalfont\sffamily}{}{0em}
%{\SecTitle{west}{east}{14pt}{\paperwidth}{#1}}[\addvspace{4ex}]



%% Useful packages
%\usepackage{amsmath}
%\usepackage{graphicx}
%\usepackage[colorinlistoftodos]{todonotes}
\definecolor{bgc}{rgb}{0.95,0.95,0.95}
\definecolor{LightCyan}{rgb}{0.88,1,1}
%\usepackage[colorlinks=true, allcolors=blue]{hyperref}
%\usepackage{abstract}
\renewcommand{\abstractname}{}%}\large 摘要}
\newcommand{\cc}{C\texttt{++}}

\newminted{cpp}{xleftmargin=.8cm,linenos,baselinestretch=1.2,autogobble=true,tabsize=4,showtabs=false,frame=single,framesep=10pt}
\newminted[inside]{cpp}{tabsize=4,autogobble=true,baselinestretch=1.2}

\title{2017年程式設計先修暑期夏令營\\學生上課講義}
%\author{王佳盈 梁家萁 侯怡安 朱睿謙 巫鈺瑩 周身鴻}
\begin{document}
\maketitle
%\begin{abstract}
%\chapter{前言}
我喜歡寫程式,寫程式可以說是學習和電腦打交道的過程。在寫程式的過程中,會發現電腦可以說得上是一位忠實、公正,又極有耐心的朋友,這些特質電腦發揮得極其盡緻,世間的朋友很難比得上。首先它非常客觀公正,無論情況多麼撲朔迷離,它仍然會在其中依循既定的原則而行,絕沒有一絲一豪偏離軌道;也不管你的程式功力多麼高深,程式寫得多麼巨大華麗,對於那豪不起眼的一點小小錯誤,它也絕不會輕易地把你放過,這是它忠實、公正的特質。另外,不管你的程式能力多麼拙劣,永無止盡的在相同的小錯誤中周旋,它也絕不失去耐心,乃至發出一點點怒火,仍然保持著冷靜沈著,一再地把最忠實而相同的錯誤訊息反應給你,這樣的耐心幾乎無人能及。如果要說它的一些缺點,大概就是有點不盡人情,有時候迂腐得可以。

寫程式跟學習世間的技能,過程非常一致。一個人學習游泳,學習騎腳踏車,如果沒有真正下水或上路練習,而只是詳細閱讀教學手冊,要真正在水中得到游泳的樂趣,或者在路上騎乘的快感,那幾乎是不太可能的事。同樣地,一個學習寫程式的人,如果沒有真正上機練習,而光是閱讀程式教學的書籍,要真正具備寫作程式的能力,那也是非常困難的事情。初學游泳的人,總是要在水中嗆幾口水;初學騎腳踏車的人,也不免有不穩或摔倒的時候,這都是學習過程普遍發生的事情。相同地,初學程式的人,總會有弄不清頭緒,在不解的錯誤中周旋的時候,這是一種除錯和學習的過程,不用因此而灰心喪志,只要持續努力,慢慢會掌握寫程式的一些絕竅。

我因為喜歡寫程式,後來也有機會開始擔任C/\cc{}程式的教學。因為了解學習程式必須實際練習,所以也一直在尋找合適的學習資源。現在網路非常發達,資訊詳盡而多元,要找到一些學習資源並不太難,但是在尋找C/\cc{}的學習資源的過程中,發現大部份的學習工具和平台多是英語介面,對於國內很多學生來說,還是有一些不容易跨越的門檻。後來在尋覓的過程中,發現了銘傳大學謝育平老師所寫作的「瘋狂程設」,這個平台不僅完全使用中文,而且還可以把許多教學、作業和解題的過程融合在一個系統中,不僅對初學程式的人極有幫助,對於程式教學的老師來說,也提供非常有用的資源,可以達到事半功倍的效果。後來也承蒙謝老師的幫助,連續幾年開課,我都一直使用這個學習平台,作為教學的輔助資源。

「瘋狂程設」的平台中,提供了很多練習題目,可以讓學習程式的人練習,而老師也可以使用這個系統,不斷開發新的題目。對於剛入門的同學來說,「瘋狂程設」有一個程式練習廣場,裡面很有系統地提供一些基礎的題目,讓初學的人練習解題,漸次深入,這就好像玩遊戲過關斬將一樣,讓學習程式也可以變得很有樂趣。對於老手來說,這些題目可能不堪一擊,但對初學者來說,很多人經常卡關,面對題目不知所措,甚至充滿了挫折。這本書的內容,主要就是針對程式練習廣場的題目,提供詳盡的思惟過程和解題參考,可以作為初學者的破關攻略,也可以從中看到各種不同的解題技巧。

這本書能夠完成,還要特別感謝實驗室幾位研究生,包括梁家萁、侯怡安、朱睿謙、巫鈺瑩、以及周身鴻等人。我讓他們每個人先練習撰寫書中的一小部份章節,一方面他們可以練習,一方面我也可以減輕一些負擔,等他們寫完各章節之後,我再整體看過並做修改。所以這本書能夠完成,也要非常感謝這些同學的幫忙。

我希望這本書可以幫助到初學C/\cc{}程式的人,也希望提供給程式教學的老師作為教學的參考,減輕彼此教和學的負擔和困難,乃至達到事半功倍的效果。書中的內容,主要針對問題如何思考和解答來做說明,所以不是完整的教科書,很多C/\cc{}語言基本的概念,還是應該要參閱其他的書籍和資源。

另外,寫程式要有相對應的開發工具,在教學上,我習慣使用的整合開發環境是Code::Blocks,這是一套跨平台的自由軟體,功能實用而且豐富;而編譯程式用的編譯器則是使用mingw,這是gcc移植到Windows的版本,也可以說是非常普及而強大的編譯工具。讀者只要連到Code::Blocks的官網,可以直接下載內含mingw的Code::Blocks版本,非常便利。本書第一部份也會針對Code::Blocks的安裝以及瘋狂程設的使用做一些說明。

我必須再次強調,學習程式一定要自己思考和練習,如果只是光看而不練的話,實際遇到問題,還是做不出來的。因此同學在使用本書的時候,除了閱讀之外,也要花一些時間自己思考,並且實際上機練習解題,務求每一個步驟都充份了解和熟悉,這樣才能達到理想的功效。

對於本書的內容,如果有什麼修正建議或錯誤指正,還請不吝提供給我們作為修正的參考,連絡方式,可以寄信到\ \href{mailto:jywglady@gmail.com}{jywglady@gmail.com}\ 電子信箱,我們先在此感謝您的幫忙。最後敬祝大家學習愉快,一切平安吉祥。

\begin{flushright}
	王佳盈 寫於 2018/06
\end{flushright}

%\end{abstract}
\ \\ \\ \\ \\ \\
\begin{center}
	本課程獲教育部扎根高中職資訊科學教育計畫補助
\end{center} 
\newpage
這份講義,僅提供給同學作為學習參考之用,希望說可以幫助大家學習C/\cc{}程式語言。
講義的內容,主要針對問題如何思考和解答來做說明,所以不是完整的教科書,很多C/\cc{}語言基本的概念,還是應該要參閱其他的書籍和資源。

另外,寫程式要有相對應的開發工具,我們使用的整合開發環境是Code::Blocks,這是一套跨平台的自由軟體,而編譯程式用的編譯器是使用mingw,這是gcc移植到Windows的版本。另外為了方便學習,我們也使用「瘋狂程設」線上學習系統做為輔助的學習資源,瘋狂程設提供了一個很好的解題學習環境,對於學習程式語言可以提供一些幫助,所以講義也會針對Code::Blocks的安裝以及瘋狂程設的使用做一些說明。

基本上,學習程式一定要自己思考和練習,如果只是光看而不練的話,實際遇到問題,還是做不出來的。因此大家在使用這份講義的時候,除了閱讀之外,也要花一些時間自己思考,然後實際上機練習解題,務求每一個步驟都充份了解和熟悉,這樣才能達到理想的功效。

另外,這份講義還在修改階段,請自行參考使用,勿隨意流傳。如果有什麼修正的建議,可以提供給我們,感謝大家。連絡方式,可以當面說明,或者寄信到 \href{mailto:jywglady@gmail.com}{jywglady@gmail.com} 或 \href{mailto:dachurita@gmail.com}{dachurita@gmail.com}。

\newpage
\setcounter{tocdepth}{2}
\tableofcontents

\section{秘密差}

將一個十進位正整數的奇數位數的和稱為$A$,偶數位數的和稱為$B$,則A與B的絕對差值$|A-B|$稱為這個正整數的秘密差。

例如:263541的奇數位數的和$A=6+5+1=12$,偶數位數的和$B=2+3+4=9$,所以263541的秘密差是$|12-9|=3$。

給定一個十進位正整數$X$,請找出$X$的秘密差。

\subsection{解題思惟}
\begin{enumerate}
	\item 如何取出一個整數的個位數呢?答:把它除以10求餘數就可了。
	\item 如何將一個整數的個位數去掉呢?答:把它除以10所得的商就是了。
	\item 如何取出一個整數的每個位數呢?答:
	\begin{enumerate}
		\item 取出它的個位。
		\item 去掉它的個位。
		\item 如果不是0的話,回到(a)。
	\end{enumerate}
	\item 如何區別這個位數是奇數位還是偶數位呢?答:設一個整數旗標flag=1,每次取一個位數的時候,讓它在1和0之間切換(設flag=1-flag)。
\end{enumerate}

\subsection{程式碼}
\begin{cppcode}
#include <iostream>

using namespace std;

int main()
{
    int n, oddsum=0, evensum=0, flag=1;
    cin >> n;
    while (n) {
        if (flag) oddsum += n%10;
        else evensum += n%10;
        n /= 10;
        flag = 1-flag;
    }
    n = oddsum-evensum;
    if (n<0) n = -n;
    cout << n << endl;
	return 0;
}
\end{cppcode}
\section{計算BMI}

輸入身高(公尺)及體重(公斤),計算BMI=體重/身高平方。

若BMI<18.5,則輸出``too thin"

若18.5<=BMI<24,則輸出``standard"

若BMI>=24,則輸出``too fat"

\subsection{解題思惟}
本題為if-else的基本題型。注意觀察程式碼中if-else及條件的使用方式。

\subsection{程式碼}
\begin{cppcode}
#include <iostream>
using namespace std;
int main()
{
	double height; cin>>height;
	double weight; cin>>weight;
	double bmi = weight/height/height;
	cout<<bmi;
	if (bmi<18.5) cout<<endl<<"too thin";
	else if (bmi<24) cout<<endl<<"standard";
	else cout<<endl<<"too fat";
	return 0;
}
\end{cppcode}

\section{猜數字}

電腦先隨機產生一個1--100的整數,使用者試著猜出這個數。每次猜的時候,如果猜得比較小,電腦會回答more,如果猜得比較大,電腦會回覆less,一直到猜中為止。猜中的時候,電腦會回應Bingo! You did it!,然後問是否重新再玩一次。

\subsection{解題思惟}
\begin{enumerate}
	\item 如何產生隨機亂數呢?答:使用 rand() 函數,注意檔頭必須引入<stdlib.h>。rand() 函數會產生一個隨機的整數,其值從 0--RAND\_MAX。
	\item 如何產生1--100的隨機亂數?答:1+rand()\%100。
	\item 其餘的部份善用分支及迴圈指令即可完成。
	\item 問:為何每次玩的時候,都是一樣的數字?答:rand()取亂數的時候,會使用一個種子值來產生亂數,而這個種子值預設是固定的,所以每次產生的亂數都一樣。那怎麼改種子值呢?可以使用srand(種子值)來完成。
	\item 問:給定種子值之後,只要種子值不變,每次出來的亂數不也一樣嗎?答:可以使用時間函數來設定種子值,一般常用的方式為 srand(time(NULL)),其中time(NULL)函數會傳回一個從以前固定某時間到現在經歷的秒數,因為時間會一直改變,這樣就可以讓每次的種子值都不一樣了。使用time(NULL)這個函數,要另外引入time.h。
\end{enumerate}

\subsection{程式碼}
\begin{cppcode}
#include <iostream>
#include <cstdlib>
#include <ctime>

using namespace std;

int main()
{
    int number, guess;
    char yn;

    srand(time(NULL));
    do {
        number = 1 + rand()%100;
        cout << "I have a number from 1-100. Please guess!\n\n";

        do {
            cout << "Your guess: ";
            cin >> guess;
            if (guess < number) cout << "  more" << endl;
            if (guess > number) cout << "  less" << endl;
        } while (guess != number);

        cout << "\nBingo! You did it!\n\n";
        cout << "Do you want to play again? (y/n) : ";
        cin >> yn;
        cout << endl << endl;

    } while (yn=='y' || yn=='Y');

    return 0;
}
\end{cppcode}

\end{document}
