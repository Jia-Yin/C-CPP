
\section{流程控制-分支}

	

\subsection{分支}

\subsubsection{if}
\begin{enumerate}
	\item 講解︰JA-001:兩數排序
		\begin{enumerate}
			\item 題目說明:
			\subitem 輸入a和b兩個數,將其依小到大的順序印出來。
			
			\item 解題思維:
			\begin{enumerate}
				\item 用if進行判斷,如果a大於b,則兩數交換。
				
				\item 交換兩數a和b,在\cc{}中可以直接使用swap函數,如果是在C裡面,則常用的方法是宣告另一個暫存變數t,然後使用以下敘述:
				\begin{inside}
				t=a; a=b; b=t;
				\end{inside}
			\end{enumerate}
			\item 程式碼:
			\begin{cppcode}
				#include <iostream>
				
				using namespace std;
				
				int main()
				{
					int a, b;
					cin >> a >> b;
					if (a>b) swap(a, b);
					cout << a << " " << b;
					return 0;
				}
					
			\end{cppcode}
		\end{enumerate}
	\item 講解︰A016:三數排序
		\begin{enumerate}
			\item 題目說明:
			\subitem 輸入三個正整數 a、b、c,將 a、b、c 從小排到大並輸出。
			
			\item 解題思維:
			\begin{enumerate}
			\item 先宣告三整數 a, b, c並輸入其值。
			\begin{inside}
			int a, b, c;
			cin >> a >> b >> c;
			\end{inside}
			\item 三數排序時,先比a和b,如果a>b則交換兩個數,使a<b,之後再比b和c,使b<c,此時c為最大值。最後再比較和調整一次a和b即可。
			\begin{inside}
			if (a>b) swap(a, b);
			if (b>c) swap(b, c);
			if (a>b) swap(a, b);
			\end{inside}
			\item 交換兩數x和y,在\cc{}中可以直接使用swap函數,如果是在C裡面,則常用的方法是宣告另一個暫存變數t,然後使用以下敘述:
			\begin{inside}
			t=a; a=b; b=t;
			\end{inside}
			\end{enumerate} 
			
			\item 程式碼:
			\begin{cppcode}
				#include <iostream>
				using namespace std;
				
				int main()
				{
					int a, b, c;
					cin >> a >> b >> c;
					if (a>b) swap(a, b);
					if (b>c) swap(b, c);
					if (a>b) swap(a, b);
					cout << a << " " << b << " " << c;
					return 0;
				}
			\end{cppcode}
		\end{enumerate}
	\item 練習︰JA-002:四數排序
		\begin{enumerate}
			\item 題目說明:
			\subitem 輸入a,b,c,d四個數,將其依小到大的順序印出來。
			
			\item 解題思維:
			\begin{enumerate}
				\item 將最大的整數置換到變數d。
				\item 對a,b,c由小到大進行三數排序。
			\end{enumerate}
\begin{comment}			
			
			\item 程式碼:
			\begin{cppcode}
				#include <iostream>
				
				using namespace std;
				
				int main()
				{
					int a, b, c, d;
					cin >> a >> b >> c >> d;
					if (a>b) swap(a, b);
					if (b>c) swap(b, c);
					if (c>d) swap(c, d);
					if (a>b) swap(a, b);
					if (b>c) swap(b, c);
					if (a>b) swap(a, b);
					cout << a << " " << b << " " << c << " " << d;
					return 0;
				}
				
			\end{cppcode}
\end{comment}			
		\end{enumerate}
	
\end{enumerate}
		
\subsubsection{if else}
\begin{enumerate}
	\item 講解︰A025:判斷閏年%第04關:分支)
		\begin{enumerate}
			\item 題目說明:
			\subitem 輸入西元年,如果該年是閏年,則輸出Yes,若該年不是閏年,則輸出No。 (閏年的定義為,四年一閏,逢百不閏,逢四百又閏。例如西元1004年為閏年,西元1100年不是閏年,西元1600年是閏年)
			
			\item 解題思維:
			\subitem 宣告年份 year,接著再按照閏年的規則判斷是否為閏年就好。
			
			\item 程式碼:
			\begin{cppcode}
				#include <iostream>
				using namespace std;
				
				int main()
				{
					int year;
					cin >> year;
					if (year%400==0) cout << "Yes";
					else if (year%100==0) cout << "No";
					else if (year%4==0) cout << "Yes";
					else cout << "No";
					return 0;
				}
			\end{cppcode}
		\end{enumerate}
		
	\item 講解︰F021:奇偶數%(第04關:分支)
		\begin{enumerate}
			\item 題目說明:
			\subitem 輸入一整數,輸出其奇偶性。
			
			\item 解題思維:
			\subitem 判斷整數n是否為奇數的方法,可求其除以2的餘數,若非0即為奇數。
			\begin{inside}
			if (n%2) { /* n 為奇數 */ }
			\end{inside}
			
			\item 程式碼:
			\begin{cppcode}
				#include <iostream>
				using namespace std;
				int main()
				{
					int n;
					cin >> n;
					if (n%2) cout << "odd";
					else cout << "even";
					return 0;
				}
			\end{cppcode}
		\end{enumerate}
	
	\item 練習︰A006:輸入之正負零%(第04關:分支)
		\begin{enumerate}
			\item 題目說明:
			\subitem 輸入一整數 N,如果 N 大於 0,則輸出 N>0,如果 N 等於 0,則輸出 N=0,如果 N 小於 0,則輸出 N<0。
			
			\item 解題思維:
			\subitem 先宣告整數 N,輸入值之後再判斷是大於零、等於零還是小於零,分別印出相對的敘述。
\begin{comment}			
			
			\item 程式碼:
			\begin{cppcode}
				#include <iostream>
				using namespace std;
				
				int main()
				{
					int n;
					cin >> n;
					if (n>0) cout << "N>0";
					else if (n==0) cout << "N=0";
					else cout << "N<0";
					return 0;
				}
			\end{cppcode}
\end{comment}			
		\end{enumerate}
	
\end{enumerate}
%\subsubsection{switch {\color{blue} (若有多餘時間)}}

%\section{8/15(二)下午:實驗室參訪}


