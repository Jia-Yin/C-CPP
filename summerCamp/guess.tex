\section{猜數字}

電腦先隨機產生一個1--100的整數,使用者試著猜出這個數。每次猜的時候,如果猜得比較小,電腦會回答more,如果猜得比較大,電腦會回覆less,一直到猜中為止。猜中的時候,電腦會回應Bingo! You did it!,然後問是否重新再玩一次。

\subsection{解題思惟}
\begin{enumerate}
	\item 如何產生隨機亂數呢?答:使用 rand() 函數,注意檔頭必須引入<stdlib.h>。rand() 函數會產生一個隨機的整數,其值從 0--RAND\_MAX。
	\item 如何產生1--100的隨機亂數?答:1+rand()\%100。
	\item 其餘的部份善用分支及迴圈指令即可完成。
	\item 問:為何每次玩的時候,都是一樣的數字?答:rand()取亂數的時候,會使用一個種子值來產生亂數,而這個種子值預設是固定的,所以每次產生的亂數都一樣。那怎麼改種子值呢?可以使用srand(種子值)來完成。
	\item 問:給定種子值之後,只要種子值不變,每次出來的亂數不也一樣嗎?答:可以使用時間函數來設定種子值,一般常用的方式為 srand(time(NULL)),其中time(NULL)函數會傳回一個從以前固定某時間到現在經歷的秒數,因為時間會一直改變,這樣就可以讓每次的種子值都不一樣了。使用time(NULL)這個函數,要另外引入time.h。
\end{enumerate}

\subsection{程式碼}
\begin{cppcode}
#include <iostream>
#include <cstdlib>
#include <ctime>

using namespace std;

int main()
{
    int number, guess;
    char yn;

    srand(time(NULL));
    do {
        number = 1 + rand()%100;
        cout << "I have a number from 1-100. Please guess!\n\n";

        do {
            cout << "Your guess: ";
            cin >> guess;
            if (guess < number) cout << "  more" << endl;
            if (guess > number) cout << "  less" << endl;
        } while (guess != number);

        cout << "\nBingo! You did it!\n\n";
        cout << "Do you want to play again? (y/n) : ";
        cin >> yn;
        cout << endl << endl;

    } while (yn=='y' || yn=='Y');

    return 0;
}
\end{cppcode}