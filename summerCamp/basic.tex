\section{基本語法}

	\subsection{輸入、輸出(cin, cout)}
	輸入輸出的詳細說明請見附錄二。
	\subsection{四則運算}
	
\begin{enumerate}
	\item 講解︰A001:Hello World
	\begin{enumerate}
	\item 題目說明:
	\subitem 請在命令視窗中印出 ``Hello World!"。
	
	\item 解題思維:
	\subitem 本題直接使用cout或printf函數印出想要顯示的文字即可。
	
	\item 程式碼:
	\begin{cppcode}
		#include <iostream>
		
		using namespace std;
		
		int main()
		{
			cout << "Hello World!";
			return 0;
		}
	\end{cppcode}
	\end{enumerate}
	
	\item 練習︰JT-01︰我可以把程式學好
		\begin{enumerate}
			\item 題目說明:
			\subitem 請印出字串 "Programming is easy!"
			
			\item 解題思維:
			\subitem 本題直接使用cout或printf函數印出想要顯示的文字即可。
\begin{comment}			
			\item 程式碼:
			\begin{cppcode}
				#include <iostream>
				
				using namespace std;
				
				int main()
				{
					cout << "Programming is easy!";
					return 0;
				}
				
			\end{cppcode}
\end{comment}
		\end{enumerate}
	\item 講解︰F001:兩數相加%(第01關:變數與計算)
		\begin{enumerate}
			\item 題目說明:
			\subitem 輸入兩整數,輸出兩數之和。
			
			\item 解題思維:
			%\subitem
			\begin{enumerate}
			\item 先宣告兩個變數。
			\begin{inside}
			int a, b; // 宣告變數
			\end{inside}
			\item 使用cin取得使用者輸入的兩個數字。
			\begin{inside}
			cin >> a >> b; // 取得輸入的值, 存入a和b
			\end{inside}
			\item 將剛剛取得的兩個數字相加,並用cout輸出。
			\begin{inside}
			cout << a+b;
			\end{inside}
			\end{enumerate} 
			
			\item 程式碼:
			\begin{cppcode}
				#include <iostream>
				
				using namespace std;
				
				int main()
				{
					int a, b;
					cin >> a >> b;
					cout << a+b;
					return 0;	
				}
			\end{cppcode}
		\end{enumerate}
		
	\item 練習︰F019:長方形面積%(第01關:變數與計算)
		\begin{enumerate}
			\item 題目說明:
			\subitem 輸入長和寬,輸出面積。
			
			\item 解題思維:
			\subitem 與上題解題思維大致相同,只是兩數相加變為兩數相乘。

\begin{comment}			
			\item 程式碼:
			\begin{cppcode}
				#include <iostream>
				
				using namespace std;
				
				int main()
				{
					int a, b;
					cin >> a >> b;
					cout << a*b;
					return 0;	
				}
			\end{cppcode}
\end{comment}
		\end{enumerate}
		
	\item 練習︰G001:長寬高算體積
		\begin{enumerate}
			\item 題目說明:
			\subitem 輸入長方體的長寬高,輸出其體積。
			
			\item 解題思維:
			\subitem 與上題解題思維大致相同,變數改成三個,輸出將三數相乘。
\begin{comment}			
			\item 程式碼:
			\begin{cppcode}
				#include <iostream>
				
				using namespace std;
				
				int main()
				{
					int a, b, c;
					cin >> a >> b >> c;
					cout << a*b*c;
					return 0;	
				}
			\end{cppcode}
\end{comment}
		\end{enumerate}
		
	\item 講解︰M90H011:整數商餘%(第01關:變數與計算){\color{blue} (cout and printf) }
		\begin{enumerate}
			\item 題目說明:
			\subitem 
			輸入兩整數m和n,輸出m除以n之商及餘數。
			\item 解題思維:
			\begin{enumerate}
			\item  先宣告兩個變數m和n,再使用scanf取得使用者輸入的兩個整數。
			\begin{inside}
			int m, n;
			scanf("%d%d", &m, &n); 
			\end{inside}
			\item  將剛剛取得的整數m除以整數n,並用printf輸出所除結果之商及餘數。
			\begin{inside}
			printf("\n%d / %d = %d", m, n, m / n);
			printf("\n%d mod %d = %d", m, n, m % n);
			\end{inside}
			\end{enumerate}
			
			\item 程式碼:
			\begin{cppcode}
				#include <stdio.h>
				
				int main()
				{
					int m, n;
					scanf("%d%d", &m, &n); 
					printf("\n%d / %d = %d", m, n, m / n);
					printf("\n%d mod %d = %d", m, n, m % n);
					return 0;
				}
			\end{cppcode}
		\end{enumerate}
	
	\item 練習︰使用printf完成下列兩題
	\begin{enumerate}
		\item F001︰兩數相加
		\item F019 ︰長方形面積
	\end{enumerate}
\end{enumerate}

