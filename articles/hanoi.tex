\documentclass[12pt,a4paper]{article}

\usepackage{jyw-program}

\begin{document}
\title{河內塔物件設計}
\author{Jia-Yin Wang}
\maketitle

\begin{abstract}
這個課程要討論如何設計一個可以使用的河內塔物件。我會提供一些思惟和實作的過程,但也會留一些作業讓大家討論和思考。本課程結束後,我希望各組可以把這個小題目完成。另外,我也會講解如何把程式碼變成程式庫,以及在Code::Blocks中使用程式庫的方法。
\end{abstract}
\section{設計物件操作界面}
\noindent 設計經驗分享:

\begin{enumerate}
	\item 在發展物件導向程式設計的時候,經常會先從一些小元件寫起,把一些小元件寫完之後,再使用它來設計出更大的元件。
	\item 當然一開始有個整體的構想是比較好的,但可能也沒有辦法想得很完整。之後在撰寫元件的時候,可能會有一些修正的過程,這是正常的。
	\item 在實作過程,慢慢會累積一些經驗,等到經驗豐富之後,碰到問題就可以比較快想出解決的辦法。
\end{enumerate}

\vspace{0.5cm}
\noindent 以河內塔為例,我目前所寫的物件操作介面如下:
\begin{cppcode}
class Hanoi {
public:
	Hanoi();
	
	void SetSticks(string left, string middle, string right);
	void SetDelay(int val) { m_Delay = val; }
	void ShowNumber(bool show) { m_sring = show; }
	void PutRings(int r, int n);
	void PutRings(string r, int n);
	void Move(int from, int to);
	void Move(string from, string to);
}
\end{cppcode}
這裡Hanoi是物件名稱,第一個Hanoi()函式是所謂的constructor,我是想用它來畫框和柱。接下來幾個方法構想如下:
\begin{enumerate}
	\item SetStick函式用來設定柱子的名稱,所以給三個字串。(Q1:可以使用字元嗎?)
	\item SetDelay函式用來設定畫格延遲時間,所以給一個整數。因為程式較短,所以直接把程式碼寫在後面。(延伸:查一下什麼叫做inline函式。)
	\item ShowNumber函式,這個函式在一開始設計的時候沒有想到,是在操作的時候,發現不清楚現在移動的是哪一個環,所以才把它加上去的。就是用數字標示每一個環,讓使用者易於了解現在是在操作哪一個環。當然,你也可以考慮不要在環上面放數字,而是把環的資訊顯示在螢幕某一個部份,這也是可以的。這個函式只設定旗標值而已。
	\item PutRings,這個函式是設定在哪一根柱上面放幾個環,所以有兩個參數。第一個指明哪一根柱,第二個說明放幾個環。一般河內塔問題會先說明共有幾個環,放在哪一根柱子上,就用這個函式來處理。最初設計這個函式的時候,只有數字的版本(0=左柱,1=中柱,2=右柱),後來想說柱子都有名稱,所以加上了用字串取代柱子編號的版本。
	\item Move函式,這個函式用來模擬移動的過程。最初覺得河內塔問題,如果可以用機械手臂和實體積木做為實作練習的平台,學習效果可能會更好,但實際上沒有辦法給大家一人一個機械手臂,所以用動畫來模擬。那這個函式就是模擬手臂移動環的過程。可能也是整個物件最難實作的部份。
\end{enumerate}
\vspace{1cm}

\section{實作過程}
第一個考量是使用圖形或文字介面?如果使用圖形介面的話,可能視覺效果會比較好,但是有兩個問題必須考量:1)圖形介面的程式設計,可能更為複雜,對同學來說,可能需要更多時間去學習和理解。2)設計出來的物件,希望幫助同學學習河內塔遞迴函式的設計,這其中可能會有逐步移動及除錯的需求,用圖形介面設計的話,可能更難和文字程式的除錯過程結合在一起。最後考量是使用文字介面撰寫。

第二個考量,文字介面設計的話,如何畫出簡單的圖形呢?答案是找網路資源。這種console類的程式庫網路上其實很多,我選擇的是rlutil,這個程式很小,簡單容易使用,而且C和\cc{}都可以使用,詳細資訊可以參考\url{https://github.com/tapio/rlutil}。(Q2:你可以找到其他類似功能的程式庫嗎?)

\subsection{constructor}
建構元不用給參數。基本上設定一些啟始參數,並且把框和柱畫出來。
\begin{cppcode}
const int OFFSET_X = 2;
const int OFFSET_Y = 1;
const int SEP_Y = 2;
const int WIDTH = 75;
const int HEIGHT = 22;
const int STICKWIDTH = 10;
	
Hanoi::Hanoi()
{
	m_stick[0] = "A";
	m_stick[1] = "B";
	m_stick[2] = "C";
	m_sring = false;
	m_Delay = 100;
	m_yBase = OFFSET_Y+HEIGHT-5;
	m_x[1] = OFFSET_X+WIDTH/2+1;
	m_x[0] = m_x[1]-2*STICKWIDTH-3;
	m_x[2] = m_x[1]+2*STICKWIDTH+3;
	DrawBound();
	DrawSticks();
	//ctor
}
\end{cppcode}
這裡 m\_yBase是柱子的y座標,m\_x陣列存的是柱子的x座標。畫框和畫柱分別由兩個函式來完成。一般而言,我們習慣把複雜的工作分成幾個簡單的部份,每個部份用一個函式來處理,這樣可以把問題簡化成幾個小題,個別加以解決。

這裡畫框和畫柱的兩個函式,為什麼沒有放在操作介面裡呢?基本上這應該是內部使用的函式,不需要讓使用者知道或使用,像這樣的函式,可以放在protected或private的區段中。protected區段,是繼承的物件可以看見並使用的; private區段,則只有這個物件本身內部可以看見並使用。在這裡我是把兩個函式放到private區段中。

\subsubsection{DrawBound}
怎麼在螢幕的某個地方畫出一個點呢?先使用rlutil的locate函式定位游標位置,接著設定背景顏色,然後畫一個空格就可以了。知道怎麼畫點,就可以畫線,可以畫框了。
\begin{cppcode}
void Hanoi::DrawBound()
{
	setBackgroundColor(WALLCOLOR);
	locate(OFFSET_X, OFFSET_Y);
	for (int i=0; i<WIDTH; i++) cout << " ";
	locate(OFFSET_X, OFFSET_Y+HEIGHT-1);
	for (int i=0; i<WIDTH; i++) cout << " ";
	
	for (int i=0; i<HEIGHT; i++) {
		locate(OFFSET_X, OFFSET_Y+i);
		cout << "  ";
		locate(OFFSET_X+WIDTH, OFFSET_Y+i);
		cout << "  ";
	}
}
\end{cppcode}
在上述程式中,第5列畫上邊界,第7列畫下邊界,第10和12列分別畫左邊界和右邊界。

\subsubsection{DrawStick}
怎麼畫柱呢?因為有三根柱,所以可以寫一個函式畫一根柱的,然後呼叫三次就可以了。每次給的是柱的水平位置(垂直位置為什麼不用給定呢?),以及柱子的標示(會印在柱子的下方)。這裡三個柱子的標示用字串陣列m\_stick來儲存。
\begin{cppcode}
void Hanoi::DrawSticks()
{
	DrawStick(m_x[0], m_stick[0]);
	DrawStick(m_x[1], m_stick[1]);
	DrawStick(m_x[2], m_stick[2]);
}

void Hanoi::DrawStick(int x0, string str)
{
	setBackgroundColor(STICKCOLOR);
	for (int i=-STICKWIDTH; i<=STICKWIDTH; i++) {
		locate(x0+i, m_yBase);
		cout << " ";
	}
	for (int i=1; i<=10; i++) {
		locate(x0, m_yBase-i);
		cout << " ";
	}
	setColor(FONTCOLOR);
	setBackgroundColor(0);
	locate(x0-str.length()/2, m_yBase + 2);
	cout << str;
}
\end{cppcode}	
第10到14列畫水平部份,第15到18列畫垂直部份,之後的程式碼是在柱子下方印出文字。

\subsection{SetStick}
這個部份相對簡單,只要設定柱子的標示,並且重新呼叫畫柱的函式就可以了。從這裡也可以看出,我們上面把畫框和畫柱的函式分開,是有好處的,我們只要針對要重新處理的部份進行呼叫即可。(Q3:可否把畫柱體及標示文字的部份再分成兩個函式?)
\begin{cppcode}
void Hanoi::SetSticks(string left, string middle, string right)
{
	m_stick[0] = left;
	m_stick[1] = middle;
	m_stick[2] = right;
	DrawSticks();
}
\end{cppcode}

\subsection{PutRings}
放環在柱子的函式,針對柱的標示,有數字和文字兩種版本,但實際上只要寫一套即可,另一套找出對應的數字或文字,加以呼叫即可。實作時,我是先寫數字的版本,寫完之後,才想到文字的版本。那文字的版本,就找出對應的數字,然後加以呼叫即可。
\begin{cppcode}
void Hanoi::PutRings(string r, int n)
{
	int s = -1;
	for (int i=0; i<3; i++) if (m_stick[i]==r) s = i;
	if (s<0) ShowError("No stick : " + r);
	PutRings(s, n);
}

void Hanoi::PutRings(int r, int n)
{
	if (n>7) {
		n = 7;
		ShowMessage("At most 7 rings, Set rings = 7");
	}
	for (int i=n; i>0; i--) {
		int y = m_yBase - m_stack[r].size() - 1;
		for (int j=-i; j<=i; j++) {
			resetColor();
			setColor(COLORS[i]);
			locate(m_x[r]+j, y);
			m_sring ? cout << i : cout << 'O';
		}
		m_stack[r].push(i);
	}
}
\end{cppcode}	
第11到14列,判斷環的個數,如果超過7個,就設定成7個,實際應用,設太多也沒有意義。
第15列,i用來標示第幾個環,所以是從n變到1。(Q4:為何從下往上畫,而不從上往下畫?)
第17到22列,畫第i個環。這邊要判斷一下m\_sring變數,看畫的時候要不要標號,如果要的話,要輸出數字而不是空格。

另外,ShowError和ShowMessage函式,是用來顯示一些警示或提醒的文字訊息,只要找一個空白處,將文字印出來即可。

\subsection{Move}
Move函式用來把環從一根柱子移動到另一根柱子,因為一定是移動最上面的環,所以只要提供出發點和目的地即可。因為柱子可以用數字或文字來標示,所以一樣有兩個版本,但只要針對其中一個撰寫,另一個變換符號後,直接呼叫寫好的函式即可。

每根柱子上面可能有若干個環,並且要支援移入移出的功能,這要用什麼來處理呢?基本上,使用陣列、向量vector、或者堆疊stack皆可,這裡我是使用堆疊。關於堆疊基本的使用說明,可以參考這一篇:
\url{http://blog.csdn.net/wallwind/article/details/6858634}。
\begin{cppcode}
	void Hanoi::Move(string from, string to)
	{
		int f=-1, t=-1;
		for (int i=0; i<3; i++) {
			if (m_stick[i]==from) f=i;
			if (m_stick[i]==to) t=i;
		}
		if (f<0) ShowError("No stick : "+from);
		if (t<0) ShowError("No stick : "+to);
		Move(f, t);
	}
	
	void Hanoi::Move(int from, int to)
	{
		string str = "Move from " + m_stick[from] + " to " + m_stick[to];
		ShowMessage(str);
		if (!m_stack[from].size()) ShowError("Move from empty stick!");
		else if (m_stack[to].size()>0 && 
			(m_stack[from].top() > m_stack[to].top()))
				ShowError("Move big on top of small!");
		int n = MoveToTop(from);
		MoveAtTop(from, to , n);
		MoveToStick(to, n);
		ShowMessage("");
		msleep(m_Delay);
	}
\end{cppcode}
第17列檢查出發點是否有柱子,如果沒有的話,要印出錯誤訊息。第18-19列檢查出發點最上面的環有沒有大於目的環最上面的環,有的話也要印出錯誤訊息。
第21列使用MoveToTop函式把環從出發柱移到上方,第22列使用MoveAtTop函式從出發柱上方移到目的柱上方,第23列使用MoveToStick函式從目的柱上方移到柱子裡。

\subsection{MoveToTop}
這個函式把環從出發柱移到上方的程式碼,基本上是內部使用的,所以一樣設定成private。基本上移動的方法,就是跑一個迴圈,每次在新的位置畫上柱環,並且把舊的柱環位置消去,然後停一個延遲時間,再繼續下一回合,連續下來,就成了移動的動作。這裡每次迴圈停留的延遲時間,可以用來控制移動的速度。另外要注意的是,因為消去柱環之後,可能中間的柱子部份會跑出來,所以可能要補畫柱子的部份。
\begin{cppcode}
int Hanoi::MoveToTop(int from)
{
	hidecursor();
	int yfrom = m_yBase - m_stack[from].size();
	int yto = m_yBase - 10 - SEP_Y;
	int n = m_stack[from].top();
	for (int y=yfrom; y>yto; y--) {
		resetColor(); setColor(COLORS[n]);
		locate(m_x[from]-n, y-1);
		for (int j=-n; j<=n; j++) m_sring ? cout << n : cout << 'O';
		resetColor();
		locate(m_x[from]-n, y);
		for (int j=-n; j<=n; j++) cout << ' ';
		if (y>OFFSET_Y+6) {
			locate(m_x[from], y);
			setBackgroundColor(STICKCOLOR);
			cout << ' ';
		}
		cout.flush();
		msleep(m_Delay);
	}
	m_stack[from].pop();
	return n;
}
\end{cppcode}
第4列計算要移動的柱子目前的y座標,第5列計算要移往的y座標,第7-21列是移動的程式,其中8-10列是畫新的環,11到13列是消去舊的環,第14列判斷是否在柱子的範圍內,如果是的話,使用第15-17列補畫柱子。第19列是強迫螢幕輸出目前的內容(緩衝資料不一定會立刻呈現出來,但要產生平滑的移動效果,所以強迫輸出)。第20列是產生延遲。第22列在柱環移出之後,將最上面的環pop出來,以更新柱子儲存的柱環資料。
	
Q5:請自行撰寫MoveAtTop以及MoveToStick兩個函式,並測試柱環的移動狀況。
\vspace{1cm}

\section{製作及使用程式庫}
程式庫大略分為兩種,一種是靜態的,一種是動態的。靜態的通常都是在產生執行檔前連結引入,會成為執行檔的部份,一般副檔名為.a;動態的則是在程式執行的時期,由程式尋找程式庫檔案引入並執行,通常在Windows中副檔名為.dll,在unix系統則為.so。使用Code::Blocks要產生

實際製作及使用方式待補。課堂講解。

\end{document}
