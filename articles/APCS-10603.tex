\documentclass[12pt,a4paper]{article}

\usepackage{jyw-program}

\begin{document}
\title{APCS實作題參考解答}
\author{Jia-Yin Wang}
\maketitle

\begin{abstract}
這份講義提供106年03月份的APC實作題的參考解答,主要是給同學做為學習參考之用。基本上這些題目對初學的同學來說,有一些可能相當困難,那同學可以根據自己的情況,看能學到哪裡就學到哪裡,不懂的話也沒有關係。只要慢慢學習下去,將來有一天也可以融會貫通。

另外這裡所提供的參考解答,主要是從初學者容易理解的方式來尋找可行的辦法。基本上解決一個問題,常常有很多種可能的方式,所以還是要鼓勵同學繼續學習更多程式的技巧和演算方法,以後可能就會找到其他更好的解法。

學習程式務求完全理解,最好還能實際上機測試,否則好像看過看懂,實際上在作答或應用的時候,還是沒有辦法寫得出來。希望同學在閱讀的同時,可以多加思考,並且實際上機測試,以求完全了解。
\end{abstract}
%\section{10503-1 成績指標}

\subsection{解題思惟}
\begin{enumerate}
	\item 這一題首先要輸入n和n筆成績,這部份使用陣列來處理就可以了。
	\item 第二個部份是要印出排序好的成績,這部份可以自己寫排序的函數,例如氣泡排序法。如果要考慮到速度的話,那氣泡排序法的複雜度是$O(n^2)$,在n很大的時候,可能會有超時的問題,這時必須改用快速排序法等$O(n\log(n))$的演算法,不過這部份對初學的同學來說比較困難,但我們可以直接使用\cc{}裡面STL函數。基本上先引入<algorithm>檔頭,假設陣列名稱為a,且要排序的個數為n,直接呼叫std::sort(a, a+n)就可以排序了。
	\item 另外還要找出不及格的最高分,和及格的最低分的問題。這部份有兩種作法,一種是從排序好的數列中尋找,另一種是從原始數列中尋找。基本上後者就是從一堆數列中找最大值(或最小值)的方法,這部份對一般同學來說比較熟悉,基本概念就是逐一比過,有更大(或更小)的就替換掉就可以了,不過在比之前要先確定這個數目是不及格(或及格)才進行比較。
\end{enumerate}

\subsection{程式碼}
\begin{cppcode}
#include <iostream>
#include <algorithm>

using namespace std;

int main()
{
	int n, score[200], t, tmax, tmin;
	
	while (cin >> n) {
		tmax = -999;
		tmin = 999;
		for (int i=0; i<n; i++) {
			cin >> score[i];
			if (score[i]<60 && score[i]>tmax) tmax = score[i];
			if (score[i]>59 && score[i]<tmin) tmin = score[i];
		}
		sort(score, score+n);
		for (int i=0; i<n; i++) {
			if (i) cout << " ";
			cout << score[i];
		}
		cout << endl;
		if (tmax==-999) cout << "best case\n";
		else cout << tmax << endl;
		if (tmin==999) cout << "worst case\n";
		else cout << tmin << endl;
	}
	return 0;
}
\end{cppcode}
%\section{10503-2 矩陣轉換}

\subsection{解題思惟}
\begin{enumerate}
	\item 這一題要處理二維矩陣轉換的問題,基本上要處理的有兩個轉換,一個是翻轉,另一個是旋轉。題目說A矩陣經由一些轉換得到B,那已知B和轉換的運算,要求出A。那我們可以從B矩陣經由相反的轉換過程把A找出來。也就是說,假如$T_i$代表第$i$個轉換,那我們知道$T_nT_{n-1}\cdots T_1(A)=B$,反過來說$A=T_1^{-1}T_2^{-1}\cdots T_n^{-1}(B)$。這邊$T_i$是給定的翻轉和旋轉兩種轉換之一,那翻轉的反轉換還是翻轉,旋轉的反轉換也是旋轉(不同方向),所以只要先把兩種轉換用函數實現出來,這一題就容易解答了。
	\item 要怎麼表達矩陣呢?基本上可以使用二維陣列,因為題目給定的R和C都不會超過10,所以宣告a[10][10]就足夠使用了。那我們宣告大一點,但實際上只使用R$\times$C或C$\times$R的大小就可以了。另外也可以使用一維陣列,自己處理二維元素和一維元素位置對應的問題,這樣也是可以的。如果讀者熟悉\cc{}裡面的向量(vector),也可以使用向量的向量來處理這個問題。
	\item 此處二維陣列連同其維度變數,都一併宣告為全域變數,然後撰寫翻轉和旋轉的函數來解題。翻轉的部份實作較容易,假設矩陣大小為$R\times C$,基本上是把矩陣的第$i$列和第$R-i$列對調。那旋轉的部份呢?稍微思考推導一下,可以知道逆時針轉90度,其實就等於轉置(transpose)加上翻轉。那轉置的話,是令$B_{ij}=A_{ji}$,這是很容易實作的。至於轉換的運算過程,可以用一個陣列存起來,最後倒過來讀取並施行於矩陣$B$,最後就可以得到矩陣$A$了。
	\item 原題目只有一筆測資,在高中生解題平台中的題目則改為多筆測資,讀到檔尾結束。如果只有一筆測資,處理上較為單純,先讀取$r, c, m$,接著讀取矩陣的$r\times c$個元素,以及$m$個運算元,再繼續進行解題。那如果是多筆測資的話,把讀取$r, c, m$的部份放到while的判斷式中,並把接下來的讀取和解題放在while迴圈中即可,如下所示。
	\begin{inside}
		while (cin >> r >> c >> m) {
			// 讀取矩陣及運算元並進行解題。
		}
	\end{inside}		
\end{enumerate}

\subsection{程式碼}
\begin{cppcode}
#include <iostream>

using namespace std;

int r, c, b[10][10], m, op[10000]; // 矩陣及運算元變數

void flip(); // 翻轉
void rotate(); // 旋轉
void transpose(); // 轉置

int main()
{
	while (cin >> r >> c >> m) { // 這部份假設多筆測資,
		for (int i=0; i<r; i++) { // 雙重迴圈讀取矩陣
			for (int j=0; j<c; j++) cin >> b[i][j];
		}
		for (int i=0; i<m; i++) cin >> op[i]; // 運算元
		for (int i=m-1; i>=0; i--) { // 運算次序倒過來
			if (op[i]==0) rotate(); // 逆時針旋轉90度
			if (op[i]==1) flip(); // 翻轉
		}
		cout << r << " " << c << endl; // 輸出維度
		for (int i=0; i<r; i++) { // 輸出矩陣
			for (int j=0; j<c; j++) {
				if (j) cout << " ";
				cout << b[i][j];
			}
			cout << endl;
		}
	}
	return 0;
}

void flip()
{
	for (int i=0; i<r/2; i++) { // 處理上半與下半的對調
		for (int j=0; j<c; j++) swap(b[i][j], b[r-1-i][j]);
	}
}

void transpose()
{
	int t = max(r, c); // 取r,c中較大者
	for (int i=0; i<t; i++) { // 取右上半部元素
		for (int j=i+1; j<t; j++) {
			swap(b[i][j], b[j][i]); // Bij 和 Bji 對調
		}
	}
	swap(r, c); // 維度對調
}

void rotate()
{
	transpose(); // 逆時旋轉90度等於轉置加上翻轉
	flip();
}	
\end{cppcode}	


\section{10603-1 秘密差}

\subsection{解題思惟}
\begin{enumerate}
	\item 這個題目首先必須處理數字的讀取,按照題目給的測試資料,位數可能高達1000,所以不能使用數字型態來讀取,而要使用字串來處理,否則位數多的時候不能通過,只能得基本分。\item 用字串讀取數字之後,奇數位和偶數位的數字要分別取正負號加總,最後再取絕對值就可以了。
\end{enumerate}

\subsection{程式碼}
以下程式碼用數字讀取輸入,但位數太大會有問題,只能得基本分。
\begin{cppcode}
#include <iostream>

using namespace std;

int main()
{
	int n, df=0, op=1; // op 用來做正負變換
	cin >> n;
	while (n) {
		df += op * (n%10); // n%10 ==> 取個位
		n /= 10; // 去掉個位
		op = -op; // 正負變換
	}
	if (df<0) df = -df; // 取絕對值
	cout << df << endl;
	return 0;
}
\end{cppcode}
以下程式碼用字串讀取輸入,位數再多都不怕。
\begin{cppcode}
#include <iostream>

using namespace std;

int main()
{
	int df=0, op=1;
	char num[1005];
	cin >> num;
	for (int i=0; num[i]; i++) {
		df += op * (num[i]-'0'); // 將文字型態的數字轉成純數字
		op = -op;
	}
	if (df<0) df = -df;
	cout << df << endl;
	return 0;
}
\end{cppcode}
\newpage
\section{10603-2 小群體}

\subsection{解題思惟}
\begin{enumerate}
	\item 先用陣列把N個人的好友編號存起來。
	\item 接下來從陣列最前面開始,依序找每個人的好友,找到好友編號之後,接下來再繼續找他的好友,這樣反覆讀取,都是同一個群組的,那找到什麼時候停止呢?
	\item 我們可以另外設一個陣列,來表示某個人是否曾經被找過。一開始把這個陣列全設為0,那如果某個人被找過了,就把他的位置改成1。這樣一來,上面的尋找過程,只要碰到被找過的人,就可以停止尋找了。另外在上面依序處理的過程中,被找過的人也不用再處理,可以直接跳過。
	\item 用另外一個變數計算總共有幾個群組就好了。
\end{enumerate}

\subsection{程式碼}
\begin{cppcode}
#include <iostream>

using namespace std;

int main()
{
	int gf[50005], n, mark[50005]={0}, group=0;
	cin >> n;
	for (int i=0; i<n; i++) cin >> gf[i];
	for (int i=0; i<n; i++) {
		if (mark[i]) continue; // 這個人被找過了,跳過
		group++; // 新增一個群組
		do {
			mark[i]=1; i=gf[i]; // 標記該人,並繼續找他的好友
		} while (mark[i]==0); // 如果好友還沒被找過,繼續處理
	}
	cout << group << endl;
	return 0;
}
\end{cppcode}
\newpage
\section{10603-3 數字龍捲風}

\subsection{解題思惟}
\begin{enumerate}
	\item 數字的讀取用二維陣列處理。
	\item 因為邊長n是奇數,所以中間的位置在n/2的地方。那輸出的順序要先找到規則,才能轉換成程式碼。基本上可以找到很多種不同的規則來處理,這邊提供兩種供參考。
	\item 第一種規則,從出發的方向開始,先走一格,接下來走的時候,總是先嘗試向右轉走,但如果右轉的數字已經讀取過,就改成直行。這樣就可以把所有數字讀完。另外因為要測試數字是否被讀取過,所以可另設一個同大小的二維陣列,把讀過的數字標記起來。至於什麼結束呢?可以設一個變數記錄已讀取的數目,如果已經到達$n\times n$就可以結束了。
	\item 第二種規則,從出發的方向開始,先走一格,接下來右轉走一格,接下來右轉走兩格,再右轉走兩格,接下來右轉走三格,再右轉走三格,依此類推,到最後右轉走n-1格兩次之後,最後再加上右轉走n-1格。
	\item 右轉和直行怎麼處理比較好呢?先談直行,基本上有上右下左四個方向,可以設定分別用0,1,2,3代表,那我們可以設一個水平位移陣列和垂直位移陣列來記錄四個方向的變化,例如往上移的話,水平位移是0,垂直位移是-1,其餘可類推。至於右轉的話,就是把目前代表的方向數字加1取4的餘數就可以了,例如目前如果往上,代表數字是0,加1之後為1,代表向右,餘可類推。
\end{enumerate}

\subsection{程式碼}
第一種規則:一直向右轉走,如果已經讀取過就改成直行。
\begin{cppcode}
#include <iostream>

using namespace std;

int main()
{
	int n, dir, cnt, dr[4]={0,-1,0,1}, dc[4]={-1,0,1,0};
	cin >> n >> dir; // 讀取 n 及方向
	int m[49][49], mark[49][49]={{0}}; // 數字陣列及記錄陣列
	for (int i=0; i<n; i++) {
		for (int j=0; j<n; j++) cin >> m[i][j]; // 讀取數字
	}
	int r=n/2, c=n/2; // 設定目前位置
	cout << m[r][c]; mark[r][c]=1; cnt=1; // 輸出原點、記錄、更新個數
	while (cnt<n*n) { // 尚未全部輸出時跑迴圈
		r+=dr[dir]; c+=dc[dir]; // 往目前方向走一格
		cout << m[r][c]; mark[r][c]=1; cnt++; // 輸出、記錄、更新個數
		// 嘗試新方向
		dir = (dir+1) % 4; // 右轉
		if (mark[r+dr[dir]][c+dc[dir]]) { // 如果右轉方向已標記過
			dir = (dir+3) % 4; // 左轉回原來方向
		}
	}
	cout << endl;
	return 0;
}
\end{cppcode}

第二種規則:從中心開始,每個方向跑兩次,並且跑的長度依次加1,最後一段要重覆再跑一次。
\begin{cppcode}
#include <iostream>

using namespace std;

int main()
{
	int a[49][49], n, dir, r, c;
	int dr[4]={0,-1,0,1}, dc[4]={-1,0,1,0};
	cin >> n >> dir;
	for (int r=0; r<n; r++) {
		for (int c=0; c<n; c++) cin >> a[r][c];
	}
	r=n/2; c=n/2; cout << a[r][c];
	for (int i=1; i<n; i++) { // 每段要走的長度i從1到n-1
		for (int j=0; j<2; j++) { // 走完一段再右轉一段共兩次
			for (int k=0; k<i; k++) { // 每段走的格數=i
				cr += dr[dir];  cc += dc[dir]; // 位移
				cout << a[cr][cc]; // 輸出
			}
			dir = (dir+1) % 4; // 右轉
		}
	}
	for (int k=0; k<n-1; k++) { // 最後一段要再走一遍
		cr += dr[dir];  cc += dc[dir];
		cout << a[cr][cc];
	}
	return 0;
}
\end{cppcode}

\newpage
\section{10603-4 基地台}

\subsection{解題思惟}
\begin{enumerate}
	\item 這題要求最小直徑d,因為是整數,那我們試著將d從1開始往上測試。
	\item 先把服務點排序,假設最左邊的點座標為x,要服務到這個點且基地台直徑為d的話,最好的情況是增設一個基地台在x+d/2,這樣的話,最右邊可服務到x+d。我們可以跳過所有可被服務到的點。
	\item 接下來如果有尚未被服務到的點,就找出下一個,假設位置為y,那我們可以像上一步一樣增設一個基地台在y+d/2,然後跳過所有可被服務的點(位置小於等於y+d)。
	\item 依此類推,一直到最後所有的點都可以被服務到,或基地台數量大於k。
	\item 檢查所有增設基地台的數量,如果小於等於k的話,那這個直徑就是我們的答案,所以用break跳出迴圈並印出結果。如果不是,把直徑增加一個最小的時間單位,然後重覆第2個步驟到上個步驟。
\end{enumerate}

\subsection{程式碼}
\begin{cppcode}
#include <iostream>
#include <algorithm>

using namespace std;

int main()
{
	int n, d, k, p[1000];
	cin >> n >> k;
	for (int i=0; i<n; i++) cin >> p[i]; // 輸入服務點
	sort(p, p+n); // 排序
	for (d=1; ; d++) { // 從1開始測試直徑
		int r=0, cnt=0, idx=0; // r=目前可服務到的範圍
		while (r<p[n-1]) { // 尚有未服務的點
			cnt++; // 新增一個基地台
			if (cnt>k) break;
			r=p[idx]+d; // 基地台放p[idx]+d/2,可服務到p[idx]+d
			while (idx<n-1 && p[idx+1]<=r) idx++; // 反覆移到下一個可服務的點
			idx++; // 再移到下一個(不能被服務到的)點
		}
		if (cnt<=k) break; // 如果總部不超過k即找到答案
	}
	cout << d << endl;
	return 0;
}
\end{cppcode}


\end{document}
