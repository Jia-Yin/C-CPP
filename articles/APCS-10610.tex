\documentclass[12pt,a4paper]{article}

\usepackage{jyw-program}

\begin{document}
\title{APCS實作題參考解答}
\author{Jia-Yin Wang}
\maketitle

\begin{abstract}
這份講義提供106年10月份的APC實作題的參考解答,主要是給同學做為學習參考之用。基本上這些題目對初學的同學來說,有一些可能相當困難,那同學可以根據自己的情況,看能學到哪裡就學到哪裡,不懂的話也沒有關係。只要慢慢學習下去,將來有一天也可以融會貫通。

另外這裡所提供的參考解答,主要是從初學者容易理解的方式來尋找可行的辦法。基本上解決一個問題,常常有很多種可能的方式,所以還是要鼓勵同學繼續學習更多程式的技巧和演算方法,以後可能就會找到其他更好的解法。

學習程式務求完全理解,最好還能實際上機測試,否則好像看過看懂,實際上在作答或應用的時候,還是沒有辦法寫得出來。希望同學在閱讀的同時,可以多加思考,並且實際上機測試,以求完全了解。
\end{abstract}
%\section{10503-1 成績指標}

\subsection{解題思惟}
\begin{enumerate}
	\item 這一題首先要輸入n和n筆成績,這部份使用陣列來處理就可以了。
	\item 第二個部份是要印出排序好的成績,這部份可以自己寫排序的函數,例如氣泡排序法。如果要考慮到速度的話,那氣泡排序法的複雜度是$O(n^2)$,在n很大的時候,可能會有超時的問題,這時必須改用快速排序法等$O(n\log(n))$的演算法,不過這部份對初學的同學來說比較困難,但我們可以直接使用\cc{}裡面STL函數。基本上先引入<algorithm>檔頭,假設陣列名稱為a,且要排序的個數為n,直接呼叫std::sort(a, a+n)就可以排序了。
	\item 另外還要找出不及格的最高分,和及格的最低分的問題。這部份有兩種作法,一種是從排序好的數列中尋找,另一種是從原始數列中尋找。基本上後者就是從一堆數列中找最大值(或最小值)的方法,這部份對一般同學來說比較熟悉,基本概念就是逐一比過,有更大(或更小)的就替換掉就可以了,不過在比之前要先確定這個數目是不及格(或及格)才進行比較。
\end{enumerate}

\subsection{程式碼}
\begin{cppcode}
#include <iostream>
#include <algorithm>

using namespace std;

int main()
{
	int n, score[200], t, tmax, tmin;
	
	while (cin >> n) {
		tmax = -999;
		tmin = 999;
		for (int i=0; i<n; i++) {
			cin >> score[i];
			if (score[i]<60 && score[i]>tmax) tmax = score[i];
			if (score[i]>59 && score[i]<tmin) tmin = score[i];
		}
		sort(score, score+n);
		for (int i=0; i<n; i++) {
			if (i) cout << " ";
			cout << score[i];
		}
		cout << endl;
		if (tmax==-999) cout << "best case\n";
		else cout << tmax << endl;
		if (tmin==999) cout << "worst case\n";
		else cout << tmin << endl;
	}
	return 0;
}
\end{cppcode}
%\section{10503-2 矩陣轉換}

\subsection{解題思惟}
\begin{enumerate}
	\item 這一題要處理二維矩陣轉換的問題,基本上要處理的有兩個轉換,一個是翻轉,另一個是旋轉。題目說A矩陣經由一些轉換得到B,那已知B和轉換的運算,要求出A。那我們可以從B矩陣經由相反的轉換過程把A找出來。也就是說,假如$T_i$代表第$i$個轉換,那我們知道$T_nT_{n-1}\cdots T_1(A)=B$,反過來說$A=T_1^{-1}T_2^{-1}\cdots T_n^{-1}(B)$。這邊$T_i$是給定的翻轉和旋轉兩種轉換之一,那翻轉的反轉換還是翻轉,旋轉的反轉換也是旋轉(不同方向),所以只要先把兩種轉換用函數實現出來,這一題就容易解答了。
	\item 要怎麼表達矩陣呢?基本上可以使用二維陣列,因為題目給定的R和C都不會超過10,所以宣告a[10][10]就足夠使用了。那我們宣告大一點,但實際上只使用R$\times$C或C$\times$R的大小就可以了。另外也可以使用一維陣列,自己處理二維元素和一維元素位置對應的問題,這樣也是可以的。如果讀者熟悉\cc{}裡面的向量(vector),也可以使用向量的向量來處理這個問題。
	\item 此處二維陣列連同其維度變數,都一併宣告為全域變數,然後撰寫翻轉和旋轉的函數來解題。翻轉的部份實作較容易,假設矩陣大小為$R\times C$,基本上是把矩陣的第$i$列和第$R-i$列對調。那旋轉的部份呢?稍微思考推導一下,可以知道逆時針轉90度,其實就等於轉置(transpose)加上翻轉。那轉置的話,是令$B_{ij}=A_{ji}$,這是很容易實作的。至於轉換的運算過程,可以用一個陣列存起來,最後倒過來讀取並施行於矩陣$B$,最後就可以得到矩陣$A$了。
	\item 原題目只有一筆測資,在高中生解題平台中的題目則改為多筆測資,讀到檔尾結束。如果只有一筆測資,處理上較為單純,先讀取$r, c, m$,接著讀取矩陣的$r\times c$個元素,以及$m$個運算元,再繼續進行解題。那如果是多筆測資的話,把讀取$r, c, m$的部份放到while的判斷式中,並把接下來的讀取和解題放在while迴圈中即可,如下所示。
	\begin{inside}
		while (cin >> r >> c >> m) {
			// 讀取矩陣及運算元並進行解題。
		}
	\end{inside}		
\end{enumerate}

\subsection{程式碼}
\begin{cppcode}
#include <iostream>

using namespace std;

int r, c, b[10][10], m, op[10000]; // 矩陣及運算元變數

void flip(); // 翻轉
void rotate(); // 旋轉
void transpose(); // 轉置

int main()
{
	while (cin >> r >> c >> m) { // 這部份假設多筆測資,
		for (int i=0; i<r; i++) { // 雙重迴圈讀取矩陣
			for (int j=0; j<c; j++) cin >> b[i][j];
		}
		for (int i=0; i<m; i++) cin >> op[i]; // 運算元
		for (int i=m-1; i>=0; i--) { // 運算次序倒過來
			if (op[i]==0) rotate(); // 逆時針旋轉90度
			if (op[i]==1) flip(); // 翻轉
		}
		cout << r << " " << c << endl; // 輸出維度
		for (int i=0; i<r; i++) { // 輸出矩陣
			for (int j=0; j<c; j++) {
				if (j) cout << " ";
				cout << b[i][j];
			}
			cout << endl;
		}
	}
	return 0;
}

void flip()
{
	for (int i=0; i<r/2; i++) { // 處理上半與下半的對調
		for (int j=0; j<c; j++) swap(b[i][j], b[r-1-i][j]);
	}
}

void transpose()
{
	int t = max(r, c); // 取r,c中較大者
	for (int i=0; i<t; i++) { // 取右上半部元素
		for (int j=i+1; j<t; j++) {
			swap(b[i][j], b[j][i]); // Bij 和 Bji 對調
		}
	}
	swap(r, c); // 維度對調
}

void rotate()
{
	transpose(); // 逆時旋轉90度等於轉置加上翻轉
	flip();
}	
\end{cppcode}	


\section{10610-1 邏輯運算子}

\subsection{解題思惟}
\begin{enumerate}
	\item 在C/\cc{}中,做邏輯判斷時,除了給判斷式,也可以給數值,給數值的時候,0會看成是false,而非0的數則看成true。
	\item 題目所給的AND運算,如果把得到的值c=1當做true,c=0當做false,那麼就可以把它看成是邏輯運算的AND(\&\&)。怎麼把c=1變成true,c=0變成false呢?只要使用c==1的判斷式就可以了。所以給定a, b, c之後,要看AND成立與否,可以用下列的式子來判斷
	\begin{inside}
		if ((a && b) == (c==1)) cout << "AND" << endl;
	\end{inside}
	\item 同樣的道理,題目所給的OR運算,把得到的值1當做true,0當做false,那麼就可以把它看成是邏輯運算的OR(||)。所以給定a, b, c之後,要看OR成立與否,可以用下列的式子來判斷
	\begin{inside}
	if ((a || b) == (c==1)) cout << "OR" << endl;
	\end{inside}
	\item 題目所給的XOR運算,把得到的值1當做true,0當做false,那麼可以把它看成是邏輯運算的(a\&\&!b) || (!a\&\&b)。所以給定a, b, c之後,要看OR成立與否,可以用下列的式子來判斷
	\begin{inside}
	if (((a && !b) || (!a && b)) == (c==1)) cout << "XOR" << endl;
	\end{inside}
	\item 以上三個式子可以針對給定的a, b, c三數,分別判斷AND, OR, 及XOR是否成立。但題目說,如果三個都不成立的話,要輸出IMPOSSIBLE,那這個部份,在程式中可以設定一個旗標來處理,如果上面三個式子,任一個被執行了,就設定旗標,如果最後發現旗標還沒有被設定,就輸出IMPOSSIBLE。
\end{enumerate}

\subsection{程式碼}
\begin{cppcode}
#include <iostream>

using namespace std;

int main()
{
	int a, b, c, found=0;
	cin >> a >> b >> c;
	if ((a && b) == (c==1)) { cout << "AND" << endl; found = 1; }
	if ((a || b) == (c==1)) { cout << "OR" << endl; found = 1; }
	if (((a && !b) || (!a && b)) == (c==1)) { cout << "XOR" << endl; found = 1; }
	if (!found) cout << "IMPOSSIBLE" << endl;
	return 0;
}

\end{cppcode}
\newpage
\section{10610-2 交錯字串}

\subsection{解題思惟}
\begin{enumerate}
	\item 這個題目針對輸入的字串,只要判斷大小寫就好了,至於字母是什麼其實並不重要。另外要處理的是大小寫交錯長度的問題,那如果我們把輸入轉換成交錯的連續大小寫長度陣列,應該會有幫助,例如字串``aafAXbbCDCCC",有3個連續小寫,2個連續大寫,2個連續小寫,5個連續大寫,也就是把它轉換成\{3,2,2,5\}四個數的陣列。這樣的話後對於判斷最長連續交錯長度會有很大幫助。
	\item 怎麼判斷大小寫呢?這個簡單,寫個函數來處理就好了。
	\begin{inside}
	int capital(char c)
	{
		if (c>='A' && c<='Z') return 1;
		return 0;
	}    
	\end{inside}
	\item 怎麼把給定的字串轉換成交錯的連續大小寫長度陣列呢?基本上就是看目前的字元是否和前一個字元的大小寫相同,如果相同,要持續累加個數,如果不同,就儲存目前的長度,然後記錄新的大小寫狀態和新的長度1。可以用以下的程式碼來實現:
	\begin{inside}
	int curcase, precase = capital(str[0]); // 記錄起始的大小寫狀態
	int cnt = 1; // 起始的連續長度為1
	for (int i=1; str[i]; i++) { // str[i]非0表示還有字元
		curcase = capital(str[i]); // 取得目前字元大小寫狀態
		if (curcase == precase) cnt++; // 如果和之前的大小寫相同則將長度加1
		else { // 不同的話將目前長度存起來,並重設連續長度
			h[idx++] = cnt;
			cnt = 1;
		}
		precase = curcase; // 儲存目前大小寫狀態
	}
	h[idx++] = cnt; // 儲存最後一組的長度
	\end{inside}
	\item 有了交錯的連續大小寫長度陣列h[]之後,怎麼計算最長的連續交錯長度呢?我們可以先寫一個函數,用來計算從h的某一個位置開始的最長交錯長度。程式碼如下:
	\begin{inside}
	int alen(int pos, int len) // pos是目前位置,len是陣列長度
	{
		if (h[pos] < k) return 0; // 找不到長度為k的連續大小寫
		int sum = k; // 找到了,長度至少為k
		pos++; // 看下一個數。下面的while:如果是k,就遞增位置,並持續加k
		while (pos<len && h[pos]==k) { sum += k; pos++; }
		if (pos<len && h[pos]>k) sum += k; // 最後如果大於k,還要加k
		return sum;
	}
	\end{inside}
	\item 有了alen函數之後,接下來就簡單了,使用alen函數,把位置從頭到尾算一次,看哪個地方開始的值最大就好了。
\end{enumerate}

\subsection{程式碼}
\begin{cppcode}
#include <iostream>

using namespace std;


int k, h[100005]={0}, idx=0;

int capital(char c);
int alen(int pos, int len);

int main()
{
	char str[100005];
	cin >> k >> str;
	int curcase, precase = capital(str[0]);
	int cnt = 1;
	for (int i=1; str[i]; i++) {
		curcase = capital(str[i]);
		if (curcase == precase) cnt++;
		else {
			h[idx++] = cnt;
			cnt = 1;
		}
		precase = curcase;
	}
	h[idx++] = cnt;
	
	int maxlen = 0;
	for (int i=0; i<idx; i++) {
		int len = alen(i, idx);
		if (len>maxlen) maxlen = len;
	}
	cout << maxlen << endl;
	return 0;
}

int capital(char c)
{
	if (c>='A' && c<='Z') return 1;
	return 0;
}    

int alen(int pos, int len)
{
	if (h[pos] < k) return 0;
	int sum = k;
	pos++;
	while (pos<len && h[pos]==k) { sum += k; pos++; }
	if (pos<len && h[pos]>k) sum += k;
	return sum;
}
\end{cppcode}

\subsection{另解}
這一題也可以把大寫字元改成1,小寫字元改成0,然後將輸入的01陣列的「部份和陣列」求出來($S_n=\sum_{i=1}^n$)。有了$S_n$之後,要計算位置$p$到位置$q$有幾個大寫字元,只要使用$S_q-S_{p-1}$就可以求得。那麼要檢查位置$p$開始是否有連續$k$個大寫,可以使用$S_{p+k-1}-S_{p-1}==k$來求得,要計算是否有連續$k$個小寫,則可以使用$S_{p+k-1}-S_{p-1}==0$來求得。使用這樣的方法,也可以很快速地求出解答。稍微複雜一點,有興趣的同學可以思考看看,程式碼如下:
\begin{cppcode}
#include <iostream>
#include <cstring>

using namespace std;

int slen, h[100005]={0};

int alen(int start, int k);

int main()
{
	char str[100005];
	int k;
	cin >> k >> str;
	slen = strlen(str);
	for (int i=0; i<slen; i++) {
		if (str[i]>='a' && str[i]<='z') h[i+1] = h[i];
		if (str[i]>='A' && str[i]<='Z') h[i+1] = h[i] + 1;
	}
	
	int maxlen = 0;
	for (int i=1; i<=slen; i++) {
		int len = alen(i, k);
		if (len > maxlen) maxlen = len;
	}
	cout << maxlen << endl;
	return 0;
}

int alen(int start, int k)
{
	int last = start+k-1;
	if (last>slen) return 0;
	int sum = h[last] - h[start-1];
	if ((sum != k) && (sum != 0)) return 0;
	sum = k - sum;
	while (last+k<=slen && h[last+k]-h[last]==sum) {
		last += k;
		sum = k - sum;
	}
	return last-start+1;
}
\end{cppcode}	
\newpage
\section{10610-3 樹狀圖分析}

\subsection{解題思惟}
\begin{enumerate}
	\item 一般儲存樹狀結構多使用指標,但我們也可以設法使用陣列來處理。基本上每個節點最多只有一個父節點,那這個題目共有n個節點,我們可以用一個陣列來表示,其中第i個元素($i\le 1\le n$)的值表示其父節點的位置;如果沒有父節點的話,就把它的值設為0。我們可以用下列程式來建構樹狀圖:
	\begin{inside}
		int n, k, node[100005]={0};
		cin >> n;
		for (int nd=1; nd<=n; nd++) {
			cin >> k;
			for (int i=0; i<k; i++) { // read child
				cin >> child;
				node[child] = nd; // connection
			}
		}
	\end{inside}
	\item 這個題目要計算每個節點的高度,這個部份怎麼處理呢?如果我們把葉節點存起來,它們的高度都是0,那麼從葉節點依次不斷尋找父節點,每次高度都增加1。應該注意的是,有些節點可能有超過一個以上的子節點,如果在往上尋找父節點的過程中,遇到一個父節點,其高度值已經存在,並且大於目前遞增的高度值,那這個遞增的高度值就沒有什麼用處了(因為高度會取最大的可能值)。假設要處理的葉節點為curnode,節點i的高度為h[i],以下程式碼可以處理往上找父節點遞增高度的問題:
	\begin{inside}
	int pnode = node[curnode];
	for (int lvl=1; pnode; lvl++) {
		if (h[pnode] >= lvl) break;
		h[pnode] = lvl;
		pnode = node[pnode];
	}
	\end{inside}
	\item 那要怎麼找到葉節點呢?基本上在讀入樹狀結構時,會讀取每個節點的子節點,如果其數目為0,就表示這個節點是葉節點,可以用以下程式碼處理:
	\begin{inside}
	int zeros[100000], zidx=0;
	for (int nd=1; nd<=n; nd++) {
		cin >> k;
		if (!k) zeros[zidx++] = nd; // k==0 ==> Add to zeros array
	}
	\end{inside}
	\item 這個題目還要找根節點,要怎麼找根節點呢?基本上當樹狀結構用陣列建構好之後,每個位置都會儲存它的父節點的位置,如果找到一個位置i,其儲存的值為0,就表示它沒有父節點,換句話說,i就是根節點,程式碼如下:
	\begin{inside}
	int root;
	for (int i=1; i<=n; i++) {
		if (!node[i]) root = i;
	}
	\end{inside}
	\item 有了以上各個部份的處理,最後只要把所有的節點的高度加起來就可以了。但要注意的是,這個題目給的節點數可能高達100000,如果整個串成一直線的話,它們的高度和會超過整數的最大值。所以求和的時候,要宣告成long long才不會發生溢位的問題。
\end{enumerate}

\subsection{程式碼}
\begin{cppcode}
#include <iostream>

using namespace std;

int n, node[100005]={0}, h[100005]={0}, zeros[100000], zidx=0;

int main()
{
	cin >> n;
	int k, child, level, pnode, curnode;
	for (int nd=1; nd<=n; nd++) {
		cin >> k;
		if (!k) zeros[zidx++] = nd;
		else for (int i=0; i<k; i++) { // read child
			cin >> child;
			node[child] = nd; // connection
		}
	}
	
	for (int i=0; i<zidx; i++) {
		curnode = zeros[i];
		pnode = node[curnode];
		for (int lvl=1; pnode; lvl++) {
			if (h[pnode] >= lvl) break;
			h[pnode] = lvl;
			pnode = node[pnode];
		}
	}
	
	int root;
	long long sum=0;
	for (int i=1; i<=n; i++) {
		if (!node[i]) root = i;
		sum += h[i];
	}
	cout << root << endl << sum << endl;
	return 0;
}
\end{cppcode}
\newpage
\section{10610-4 物品堆疊}

\subsection{解題思惟}
\begin{enumerate}
	\item 這個題目看起來似乎有點複雜,但仔細想一想,會發現一個很簡單的思路:我們先考慮一個特定的排列順序$T_1T_2\cdots T_N$ (從上到下),其中$T_1, T_2\cdots,T_N$為$1,2,\cdots N$的重排。現在只考慮$T_iT_{i+1}$的順序,如果我們把它們對調過來,會不會變得更好呢?
	\item 對調過來的話,$T_i$變到$T_{i+1}$的下面,這對於$T_{i+1}T_i$兩者上面的物品,取用所需的能量是不受影響的,對於$T_{i+1}T_i$兩者以下的物品,也不會有影響。但是對$T_i$來說,現在多了一個$T_{i+1}$在上面,所以要搬動的時候,要多搬動$W(T_{i+1})$的重量,另一方面,$T_{i+1}$現在跑到$T_i$上面了,所以搬動的時候,可以省下$W(T_i)$的重量,把兩者的搬動次數也考慮進去,對調後增加的能量為$W(T_{i+1})F(T_i)-W(T_i)F(T_{i+1})$。
	\item 如果上面增加的能量為負的,那表示我們把$T_iT_{i+1}$對調,會得到更省能量的方式。也就是說,對於一個最好的排列順序$T_1T_2\cdots T_N$來說,必須有$W(T_{i+1})F(T_i)-W(T_i)F(T_{i+1})\ge 0$,或者說,
	$$W(T_{i+1})/F(T_{i+1}) \ge W(T_i)/F(T_i)$$
	把$T_1\cdots T_N$全部考慮進去,會得到以下的不等式:
	$$W(T_1)/F(T_1)\le W(T_2)/F(T_2)\le\cdots\le W(T_N)/F(T_N)$$
	\item 所以這一題等於透過排序就可以把最好的順序找出來,要排的數目字是每個物品的重量除以其搬動次數,從小到大。
	\item 這一題要拿到全部的分數,有兩個地方必須注意:1)物品的個數可能高達100000,如果我們使用氣泡排序法等$O(n^2)$複雜度的排序法,會超時。這個部份可以直接呼叫std::sort函數,其複雜度為$O(n\log(n))$。2)計算總共需花費的能量時,因為每個物品重量可能高達1000,如果使用整數的話,也可能產生溢位的問題,所以用來計算總能量的變數必須宣告為long long才不會發生溢位的問題。
	\item 這邊在使用std::sort排序函數時,要排的數目是$1,2,\cdots N$,但是排序的標準是使用要排序的數字的函數($W_i/F_i$),所以使用了三個參數的版本,其中第三個參數是用來決定兩數大小的函數,基本上是回傳一個布林值,當順序正確時要回傳true。
\end{enumerate}

\subsection{程式碼}
\begin{cppcode}
#include <iostream>
#include <algorithm>

using namespace std;

int weight[100005], freq[100005], order[100005];

bool myfunction (int i, int j) { return weight[i]*freq[j] < weight[j]*freq[i]; }

int main()
{
	int n;
	cin >> n;
	for (int i=0; i<n; i++) cin >> weight[i];
	for (int i=0; i<n; i++) cin >> freq[i];
	for (int i=0; i<n; i++) order[i] = i;
	
	sort(order, order+n, myfunction);
	
	long long upweight=0, sum=0;
	for (int i=1; i<n; i++) {
		upweight += weight[order[i-1]];
		sum += upweight * freq[order[i]];
	}
	cout << sum << endl;
	return 0;
}
\end{cppcode}

如果學過結構的話,也可以把上面的程式碼修改如下:

\begin{cppcode}
#include <iostream>
#include <algorithm>

using namespace std;

struct item {
	long long weight;
	long long freq;
};

item items[100005];

bool myfunction (item i, item j) { return i.weight*j.freq < j.weight*i.freq; }

int main()
{
	int n;
	cin >> n;
	for (int i=0; i<n; i++) cin >> items[i].weight;
	for (int i=0; i<n; i++) cin >> items[i].freq;
	
	sort(items, items+n, myfunction);
	
	long long upweight=0, sum=0;
	for (int i=1; i<n; i++) {
		upweight += items[i-1].weight;
		sum += upweight * items[i].freq;
	}
	cout << sum << endl;
	return 0;
}
\end{cppcode}
	

\end{document}
