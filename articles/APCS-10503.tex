\documentclass[12pt,a4paper]{article}

\usepackage{jyw-program}

\begin{document}
\title{APCS實作題參考解答}
\author{Jia-Yin Wang}
\maketitle

\begin{abstract}
這份講義提供105年03月份的APC實作題的參考解答,主要是給同學做為學習參考之用。基本上這些題目對初學的同學來說,有一些可能相當困難,那同學可以根據自己的情況,看能學到哪裡就學到哪裡,不懂的話也沒有關係。只要慢慢學習下去,將來有一天也可以融會貫通。

另外這裡所提供的參考解答,主要是從初學者容易理解的方式來尋找可行的辦法。基本上解決一個問題,常常有很多種可能的方式,所以還是要鼓勵同學繼續學習更多程式的技巧和演算方法,以後可能就會找到其他更好的解法。

學習程式務求完全理解,最好還能實際上機測試,否則好像看過看懂,實際上在作答或應用的時候,還是沒有辦法寫得出來。希望同學在閱讀的同時,可以多加思考,並且實際上機測試,以求完全了解。
\end{abstract}
\section{10503-1 成績指標}

\subsection{解題思惟}
\begin{enumerate}
	\item 這一題首先要輸入n和n筆成績,這部份使用陣列來處理就可以了。
	\item 第二個部份是要印出排序好的成績,這部份可以自己寫排序的函數,例如氣泡排序法。如果要考慮到速度的話,那氣泡排序法的複雜度是$O(n^2)$,在n很大的時候,可能會有超時的問題,這時必須改用快速排序法等$O(n\log(n))$的演算法,不過這部份對初學的同學來說比較困難,但我們可以直接使用\cc{}裡面STL函數。基本上先引入<algorithm>檔頭,假設陣列名稱為a,且要排序的個數為n,直接呼叫std::sort(a, a+n)就可以排序了。
	\item 另外還要找出不及格的最高分,和及格的最低分的問題。這部份有兩種作法,一種是從排序好的數列中尋找,另一種是從原始數列中尋找。基本上後者就是從一堆數列中找最大值(或最小值)的方法,這部份對一般同學來說比較熟悉,基本概念就是逐一比過,有更大(或更小)的就替換掉就可以了,不過在比之前要先確定這個數目是不及格(或及格)才進行比較。
\end{enumerate}

\subsection{程式碼}
\begin{cppcode}
#include <iostream>
#include <algorithm>

using namespace std;

int main()
{
	int n, score[200], t, tmax, tmin;
	
	while (cin >> n) {
		tmax = -999;
		tmin = 999;
		for (int i=0; i<n; i++) {
			cin >> score[i];
			if (score[i]<60 && score[i]>tmax) tmax = score[i];
			if (score[i]>59 && score[i]<tmin) tmin = score[i];
		}
		sort(score, score+n);
		for (int i=0; i<n; i++) {
			if (i) cout << " ";
			cout << score[i];
		}
		cout << endl;
		if (tmax==-999) cout << "best case\n";
		else cout << tmax << endl;
		if (tmin==999) cout << "worst case\n";
		else cout << tmin << endl;
	}
	return 0;
}
\end{cppcode}
\newpage
\section{10503-2 矩陣轉換}

\subsection{解題思惟}
\begin{enumerate}
	\item 這一題要處理二維矩陣轉換的問題,基本上要處理的有兩個轉換,一個是翻轉,另一個是旋轉。題目說A矩陣經由一些轉換得到B,那已知B和轉換的運算,要求出A。那我們可以從B矩陣經由相反的轉換過程把A找出來。也就是說,假如$T_i$代表第$i$個轉換,那我們知道$T_nT_{n-1}\cdots T_1(A)=B$,反過來說$A=T_1^{-1}T_2^{-1}\cdots T_n^{-1}(B)$。這邊$T_i$是給定的翻轉和旋轉兩種轉換之一,那翻轉的反轉換還是翻轉,旋轉的反轉換也是旋轉(不同方向),所以只要先把兩種轉換用函數實現出來,這一題就容易解答了。
	\item 要怎麼表達矩陣呢?基本上可以使用二維陣列,因為題目給定的R和C都不會超過10,所以宣告a[10][10]就足夠使用了。那我們宣告大一點,但實際上只使用R$\times$C或C$\times$R的大小就可以了。另外也可以使用一維陣列,自己處理二維元素和一維元素位置對應的問題,這樣也是可以的。如果讀者熟悉\cc{}裡面的向量(vector),也可以使用向量的向量來處理這個問題。
	\item 此處二維陣列連同其維度變數,都一併宣告為全域變數,然後撰寫翻轉和旋轉的函數來解題。翻轉的部份實作較容易,假設矩陣大小為$R\times C$,基本上是把矩陣的第$i$列和第$R-i$列對調。那旋轉的部份呢?稍微思考推導一下,可以知道逆時針轉90度,其實就等於轉置(transpose)加上翻轉。那轉置的話,是令$B_{ij}=A_{ji}$,這是很容易實作的。至於轉換的運算過程,可以用一個陣列存起來,最後倒過來讀取並施行於矩陣$B$,最後就可以得到矩陣$A$了。
	\item 原題目只有一筆測資,在高中生解題平台中的題目則改為多筆測資,讀到檔尾結束。如果只有一筆測資,處理上較為單純,先讀取$r, c, m$,接著讀取矩陣的$r\times c$個元素,以及$m$個運算元,再繼續進行解題。那如果是多筆測資的話,把讀取$r, c, m$的部份放到while的判斷式中,並把接下來的讀取和解題放在while迴圈中即可,如下所示。
	\begin{inside}
		while (cin >> r >> c >> m) {
			// 讀取矩陣及運算元並進行解題。
		}
	\end{inside}		
\end{enumerate}

\subsection{程式碼}
\begin{cppcode}
#include <iostream>

using namespace std;

int r, c, b[10][10], m, op[10000]; // 矩陣及運算元變數

void flip(); // 翻轉
void rotate(); // 旋轉
void transpose(); // 轉置

int main()
{
	while (cin >> r >> c >> m) { // 這部份假設多筆測資,
		for (int i=0; i<r; i++) { // 雙重迴圈讀取矩陣
			for (int j=0; j<c; j++) cin >> b[i][j];
		}
		for (int i=0; i<m; i++) cin >> op[i]; // 運算元
		for (int i=m-1; i>=0; i--) { // 運算次序倒過來
			if (op[i]==0) rotate(); // 逆時針旋轉90度
			if (op[i]==1) flip(); // 翻轉
		}
		cout << r << " " << c << endl; // 輸出維度
		for (int i=0; i<r; i++) { // 輸出矩陣
			for (int j=0; j<c; j++) {
				if (j) cout << " ";
				cout << b[i][j];
			}
			cout << endl;
		}
	}
	return 0;
}

void flip()
{
	for (int i=0; i<r/2; i++) { // 處理上半與下半的對調
		for (int j=0; j<c; j++) swap(b[i][j], b[r-1-i][j]);
	}
}

void transpose()
{
	int t = max(r, c); // 取r,c中較大者
	for (int i=0; i<t; i++) { // 取右上半部元素
		for (int j=i+1; j<t; j++) {
			swap(b[i][j], b[j][i]); // Bij 和 Bji 對調
		}
	}
	swap(r, c); // 維度對調
}

void rotate()
{
	transpose(); // 逆時旋轉90度等於轉置加上翻轉
	flip();
}	
\end{cppcode}	

\newpage
\section{10503-3 線段覆蓋長度}

\subsection{解題思惟}
\begin{enumerate}
	\item 這一題等於是要計算所有線段聯集之後覆蓋的長度。
	\item 因為所有線段的端點都是整數,我們可以使用一個比較基本的辦法,就是用陣列來代表整個範圍中出現的單位線段,那每次讀取一個線段,例如讀到線段[a,b],我們就把a到b-1的位置全部標記起來。這樣一來,當所有線段都讀取完畢之後,我們只要計算總共被標記的位置有幾個就可以了。以下程式碼一即使用這個辦法。實際上測試時,會發現最後幾筆測資超時,表示計算複雜度還是太高。
	\item 這個題目最快的解法,稱為Klee's Algorithm,其演算法概念如下。首先將每個線段的端點值都存起來,而且要記錄是左端點還是右端點,這樣如果有n個線段的話,我們就會有2n個端點值。接著將這2n個值排序。排序好之後,從頭開始逐一巡訪每個端點值,兩個端點值之間都構成一個線段,如果這個線段在聯集中,就把它加總起來,如果不在聯集中,就跳過。那我們怎麼知道這個線段是不是在聯集中呢?我們從左往右逐一巡訪每個端點,如果這個端點是左端點,表示後面會有線段,如果這個端點是右端點,表示線段結束。那如果連續巡訪到兩個左端點,就表示目前有兩個線段還沒結束。所以我們可以這樣做,先宣告一個變數cnt=0,然後巡訪過程中,如果碰到左端點,就把cnt加1,表示目前多出現了一個線段,如果碰到右端點,就把cnt減1,表示減少了一個線段,這樣一來,只要目前的cnt為正,表示接下來的這個線段是在聯集中的,應該要加總;反之,如果cnt小於等於0,表示不在聯集中,不需加總,這樣問題就解決了。
	\item 在上面演算法中,排序的時候,是使用端點的值,但仍需保留該端點為左端點或右端點,一般來說要使用結構元素,其中包括數值和左右端點的旗標,然後針對結構元素排序。初學者如果對結構不熟,我們可以使用一個變通的辦法,我們把端點的數值乘以2,然後利用位元0來標記左端點或右端點,這樣的話對排序沒有影響,而且可以保留端點的類型。那排序好之後,把值除以2的整數商就是原先的端點值,把值除以2的餘數就是端點的類型,這樣就容易處理了。
\end{enumerate}

\subsection{程式碼}
程式碼一:使用陣列代表每個單位長度線段,並使用標記方式處理線段,無法通過所有測資。
\begin{cppcode}
#include <iostream>

using namespace std;

int main()
{
	int n, a, b, right=0;
	while (cin >> n) { // 假設有n筆測資
		int num[1000005]={0};
		for (int i=0; i<n; i++) {
			cin >> a >> b; // 讀取線段 
			for (int j=a; j<b; j++) num[j]=1; // 標記
			if (b>right) right=b; // 存取目前最右端的值
		}
		int sum=0;
		for (int i=0; i<right; i++) { // 檢視每個單位線段
			sum += num[i]; // 加總 (被標記的為1,未標記為0)
		}
		cout << sum << endl;
	}
	return 0;
}
\end{cppcode}

程式碼二:使用Klee's Algorithm。用來排序的端點值是原數值乘以2加上端點類型的值,此處設左端點為0,右端點為1。
\begin{cppcode}

#include <iostream>
#include <algorithm>

using namespace std;

int num[2*100005], n;

int main()
{
	int a, b;
	while (cin >> n) { // 讀取 n 值
		for (int i=0; i<n; i++) { // 讀取 n 個線段
			cin >> a >> b;
			num[2*i] = 2*a; // 左端點值,乘以2就好
			num[2*i+1] = 2*b+1; // 右端點值,乘以2加1
		}
		sort(num, num+2*n); // 端點值排序
		int sum=0, open=0; // open 用來計算目前的線段數
		for (int i=0; i<2*n; i++) {
			if (open>0) sum += num[i]/2-num[i-1]/2; // 在聯集中,應加總
			if (num[i]&1) open--; // 更新線段數,右減1
			else open++; // 更新線段數,左加1
		}
		cout << sum << endl;
	}
	return 0;
}
\end{cppcode}	

\newpage
\section{10503-4 血緣關係}

\subsection{解題思惟}
\begin{enumerate}
	\item 這一題比較複雜一點,提供兩種解答思路供作參考。
	\item 第一種思路,使用陣列處理。我們使用陣列parent,其第i個元素值表示第i個節點的父節點位置。另外我們還使用兩個陣列,一個陣列為leaf,用來儲存所有葉節點編號,一個陣列為isparent,記錄節點i是否為父節點。每次讀取一對父子節點時,就更新陣列parent及isparent的值。所有輸入讀取之後,利用isparent來找出所有葉節點並存入leaf陣列中。最後我們針對所有的葉節點進行巡訪,計算每個葉節點至最上層祖先的距離,以及任兩個葉節點之間的距離,找出其中的最大值即為答案。在計算兩葉節點之距離時,要先找出其共同祖先。所以我們撰寫了三個輔助函數,第一個函數計算從a節點上溯至b節點要幾步,如果到不了就回傳-1,這主要是用來輔助尋找共同祖先及計算距離使用的。第二個函數計算節點a至節點b的距離,其中會使用到第一個函數。最後一個函數用來計算某節點至最上面祖先節點的距離。詳細的程式碼及說明可參考程式碼一。
	\item 第二種思路是同時使用向量(vector)來儲存某節點的所有子節點,因為各節點的子節點數是不定的,使用向量會比較便利。接著找出所有節點的高度,其方法如下:針對每個葉節點,設其高度為0,並開始不斷上溯,每次上溯一步高度便加1,如果這時上溯的父節點已設有高度,便進行比較,看是否需更新,如不需更新,則此葉節點的更新任務完成,可繼續巡訪下一葉節點,如需更新,則更新之後,再繼續找更上面的父親節,並重覆同樣步驟直到更新任務完成,再繼續巡訪下一葉節點。當所有葉節點巡訪完成之後,即得到所有節點的高度。接下來要計算最遠兩節點可分兩種情況,第一種情況,某節點只有一子節點,那就計算它自己的高度(等於和最遠葉節點的距離),第二種情況,某節點有兩個以上子節點,那麼就找出所有子節點中高度最大的兩個,把兩個值加起來再加上2就等於兩個最深葉節點的距離了。所以兩種情況的最大值就是答案了。詳細的程式碼及說明可參考程式碼二。
	\item 以上兩個演算法中,第一個演算法較慢,當節點數較多的時候會超時,無法通過所有測資。第二個演算法較快,可以通過所有測資。
\end{enumerate}

\subsection{程式碼}
程式碼一:這個方法較慢,當節點數較多的時候會超時,無法通過所有測資。
\begin{cppcode}
#include <iostream>

using namespace std;

int n, parent[100005];

int fromAtoB(int a, int b); // 計算從a上溯到b要幾步,到不了就回傳-1
int disAtoB(int a, int b); // 計算a與b的距離
int disAtoRoot(int a); // 計算a到最上面的祖先之距離

int main()
{
	int n, a, b, leaf[100000], isparent[100000], lfcnt;
	while (cin >> n) {
		for (int i=0; i<n; i++) { // 啟始值設定
			parent[i] = -1;  // 預設其父節點為空
			isparent[i] = 0; // 預設都不是父節點
		}
		for (int i=0; i<n-1; i++) {
			cin >> a >> b;
			parent[b] = a; // 標記:a為b之父節點
			isparent[a] = 1; // a是父節點
		}
		// 找出所有葉節點
		lfcnt=0;
		for (int i=0; i<n; i++) {
			if (isparent[i]) continue;
			leaf[lfcnt++] = i;
		}
		int maxdis = 0, dis;
		for (int i=0; i<lfcnt; i++) { // 從每個葉節點尋找
			if (disAtoRoot(leaf[i]) > maxdis) { // 至祖先距離為最大值?
				maxdis = disAtoRoot(leaf[i]); // 更新目前最大距離
			}
			for (int j=i+1; j<lfcnt; j++) { // 計算任兩個葉節點之距離
				dis = disAtoB(leaf[i], leaf[j]);
				if (dis > maxdis) maxdis = dis; // 超過最大值的話要更新
			}
		}
		cout << maxdis << endl;
	}
	return 0;
}

int disAtoRoot(int a)
{
	int step = 0;
	while (parent[a]>=0) { step++; a=parent[a]; } // 上溯並更新步數
	return step;
}

int disAtoB(int a, int b)
{
	int common=parent[a], step=1; // common為共同祖先,從a父節點找起
	
	while (fromAtoB(b, common)<0) { // 如果b到不了common
		step++; // 步數更新
		common = parent[common]; // 繼續往上找共同祖先
	}
	return step + fromAtoB(b, common); // 祖先至兩邊的距離和
}

int fromAtoB(int a, int b) // a上溯到b的步數,到不了的話回傳-1
{
	int step, found=0; // 步數,是否到得
	for (step=1; parent[a]>=0; step++) { // a父節點存在?
		if (parent[a]==b) { found=1; break; } // 是否為b?
		a = parent[a]; // 不是的話,繼續往上
	}
	if (found) return step; // 找到回傳步數
	else return -1; // 找不到回傳-1
}
\end{cppcode}
程式碼二:同時使用向量來儲存某節點的所有子節點,以及用陣列儲存某節點的高度。此方法較快,可通過所有測資。
\begin{cppcode}
#include <iostream>
#include <vector>

using namespace std;

int main()
{
	int n, a, b;
	while (cin >> n) {
		int high[100005]={0}, parent[100005];
		vector<int> node[100005], leaves;
		for (int i=0; i<n; i++) parent[i] = -1; // 所有節點之父為空
		for (int i=0; i<n-1; i++) {
			cin >> a >> b;
			node[a].push_back(b); // a節點之子節點加入b
			parent[b] = a; // b之父為a
		}
		// Find leaves
		for (int i=0; i<n; i++) { // 子節點數為0者即為葉節點
			if (node[i].size()==0) leaves.push_back(i);
		}
		// Determine height for each node
		for (int i=0; i<leaves.size(); i++) { // 使用每一個葉節點
			int num = leaves[i]; // num為葉節點編號
			int pnum = parent[num]; // pnum為父節點編號
			while (pnum>=0) { // 父節點非空,高=max(子高度+1,已設高度)
				if (high[num]+1 <= high[pnum]) break; // 不需更新
				high[pnum] = high[num] + 1; // 更新父節點高度
				num = pnum; // 這兩列更新目前節點及父節點編號值
				pnum = parent[num];
			}
		}
		// 最大距離情況有二:1)某點至最深葉節點,2)某點的兩個最深葉節點
		int maxdis = 0;
		for (int i=0; i<n; i++) { // 巡訪每個節點
			if (node[i].size()==0) continue; // 無子節點,跳過
			if (node[i].size()==1) { // 只有一葉節點,檢查情況1
				if (high[i] > maxdis) maxdis = high[i];
				continue;
			}
			// 以下為情況2,需找出兩最深葉節點
			int maxd1=0, maxd2=0, d; 
			for (int j=0; j<node[i].size(); j++) { // 巡訪所有子節點
				d = high[node[i][j]]; // 計算其高度
				if (d>maxd1) { maxd2=maxd1; maxd1=d; } // 更新二最大值
				else if (d>maxd2) maxd2=d; // 更新次大值
			}
			// 以下為情況2之距離,比較看是否為最大值
			if (maxd1+maxd2+2>maxdis) maxdis = maxd1+maxd2+2;
		}
		cout << maxdis << endl;
	}
	return 0;
}


// 以下程式只得 40% ,在 N 可以到2000的時候,測資點 < 10M,超時

\end{cppcode}	

\end{document}
