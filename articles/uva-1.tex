\chapter{uva一顆星選題}

\section{uva 10162 Last Digit}
輸入一個整數n ($1\le n\le 2\times 10^{100}$),令
$$ S = 1^1 + 2^2 + 3^3 + \cdots + n^n, $$
請計算$S$的個位數。

\subsection{解題思惟}
本題看起來很難,但逐一計算會發現規律,有規律之後這一題就不難解了。
\begin{enumerate}
	\item $1^1$的個位數為1。$11^{11}$的個位數同$1^{11}$仍為1。依次類推,只要個位數是1的數,次方之後個位數都是1。
	\item $2^2$的個位數為4,繼續不斷乘以2,其個位數依次為4, 8, 6, 2, ...,可以看出4個就循環一次,所以$12^{12}$的個位數同$2^{12}$,其個位數為6,接著$22^{22}$個位數同$2^{22}$,其個位數為4。依次類推,可以發現$2^2$, $12^{12}$, $22^{22}$, $32^{32}$, ...,其個位數次依為4, 6, 4, 6, ...。基本上每20個循環一次。
	\item 同樣地,針對3, 4, ..., 9, 計算個位數的變化規則,列表如表\ref{dtable}。
	\begin{table}[h]
	\caption{個位數變化規則}
	\label{dtable}
	\centering
	\begin{tabular}{|c|c|c|c|c|c|c|c|c|c|c|}
		\hline 
		個位數 & 0 & 1 & 2 & 3 & 4 & 5 & 6 & 7 & 8 & 9 \\ 
		\hline 
		變化規律 & 0 & 1 & 2,4,8,6 & 3,9,7,1 & 4,6 & 5 & 6 & 7,9,3,1 & 8,4,2,6 & 9,1 \\ 
		\hline 
	\end{tabular} 
	\end{table}
	\item 從表中可以看出,基本上循環週期只有1, 2, 4三種。但基數的個位是每10個循環一次,所以$n^n$其個位數是每20個循環一次。(1, 2, 4, 10的最小公倍數),前20個的個位數列表如表\ref{stable}:
	\begin{table}[h]
	\caption{$n^n$個位數}
	\label{stable}
	\centering
	\begin{tabular}{|c|c|c|c|c|c|c|c|c|c|c|c|c|c|c|c|c|c|c|c|c|}
		\hline 
		$n$ & 0 & 1 & 2 & 3 & 4 & 5 & 6 & 7 & 8 & 9 & 10 & 11 & 12 & 13 & 14 & 15 & 16 & 17 & 18 & 19\\ 
		\hline 
		$n^n$個位 & 0 & 1 & 4 & 7 & 6 & 5 & 6 & 3 & 6 & 9 & 0 & 1 & 6 & 3 & 6 & 5 & 6 & 7 & 4 & 9\\ 
		\hline 
	\end{tabular} 
	\end{table}
	\item 已經知道$n^n$的個位數是每20個循環一次,那麼
	$S=1^1+2^2+\cdots+n^n$ 的個位數的規律為何?基本上每20個就會加總同樣的個位數,那麼先算前20個個位數的和,加總後其個位數為4,換句說,$S$的個位數,每20個會加一個4,所以20個一數,依次加4, 8, 2, 6, 0,所以5次後,個位數就不變了,所以$S$的個位數的週期為20*5=100。
	\item 了解上述的規律之後,只要依上述兩個表,先算出$S$的前100項的個位數,那麼只要看輸入數字的最後2位去查詢它的個位數即可。
	\item 本題另外要注意的點是輸入的數字極大,所以要使用字串來進行處理,不能直接讀取數字,會有溢位的問題。
\end{enumerate}

\subsection{程式碼}
\begin{cppcode}
#include <cstdio>

int ren[] = {0,1,4,7,6,5,6,3,6,9,0,1,6,3,6,5,6,7,4,9};

int main() {
	int sum = 0, ans[100];
	char num[105];
	for (int i=0; i<100; i++) { // 算S前100項的個位數
		sum += ren[i%20];
		ans[i] = sum%10;
	}
	while (scanf("%s", num)) {
		if (num[0]=='0' && num[1]==0) break;
		int len=0;
		while (num[len]) len++;
		if (len==1) sum = num[0]-'0'; // 只有1位
		else sum = (num[len-2]-'0')*10 + (num[len-1]-'0');
		printf("%d\n", ans[sum]);
	}
	return 0;
}
\end{cppcode}

\section{uva 10268 498'}
本題要計算多項式
$$a_0x^n + a_1x^{n-1} + \cdots + a_n$$
在$x=x_0$的微分值,也就是計算
$$a_0nx_0^{n-1}+a_1(n-1)x_0^{n-2}+\cdots+a_{n-1}$$

\subsection{輸入格式}
每筆資料佔兩列,第一列為$x_0$的值,第二列為$a_0, a_1, \cdots$。總共幾筆資料不定,一直處理到檔尾為止。
\subsection{輸出格式}
每筆資料輸出佔一列,直接印出在$x_0$的微分值。

\subsection{解題思惟}
1. 依正常方式,須要先了解次冪,然後逐項相加。
2. 依特殊迭代公式,不須要先了解次冪,每次讀取新值做一次迭代。
\subsection{程式碼}
使用\cc{}的字串資料流
\begin{cppcode}
#include <iostream>
#include <sstream>
#include <string>

using namespace std;

int main()
{
	int x, sum, c, deg;
	string s;
	while (cin >> x) {
		getline(cin, s);
		getline(cin, s);
		deg = 0;
		for (int i=0; i<s.length()-1; i++) if (s[i]==' ' && s[i+1]!=' ') deg++;
		istringstream sin(s);
		sum = 0;
		for (int i=0; i<deg; i++) {
			sin >> c;
			sum = sum*x + c*(deg-i);
		}
		cout << sum << endl;
	}
	return 0;
}
	
\end{cppcode}

使用C
\begin{cppcode}
#include <stdio.h>

#define MAXLINE 1000000

int main()
{
	char s[MAXLINE];
	int sum, offset, x, c, deg, i;
	while (scanf("%d", &x)!=EOF) {
	fgets(s, MAXLINE-1, stdin);
	fgets(s, MAXLINE-1, stdin);
	deg = 0;
	for (i=0; s[i+1]; i++) if (s[i]==' ' && s[i+1]!=' ') deg++; // find degree
	sum = offset = 0;
	for (i=0; i<deg; i++) {
		while (s[offset]==' ') offset++; // skip spaces
		sscanf(s+offset, "%d", &c); // read number
		while (s[offset]=='-' || (s[offset]>='0' && s[offset]<='9')) offset++; // skip number
		while (s[offset]==' ') offset++; // skip spaces
		sum = sum*x + c*(deg-i);
	}
	printf("%d\n", sum);
}
return 0;
}
\end{cppcode}
特殊處理方式
\begin{cppcode}
#include <cstdlib>
#include <cstring>
#include <cstdio>

using namespace std;

int main()
{
	int x, a, temp;
	
	while (scanf("%d",&x) != EOF) {
		getchar();
		temp = getchar();
		int sum = 0, ans = 0;
		while (temp != '\n' && temp != EOF) {
			if (temp == '-' || temp >= '0' && temp <= '9') {
				ungetc(temp, stdin);
				scanf("%d",&a);
				ans = ans * x + sum;
				sum = sum * x + a;
			}
			temp = getchar();
		}
		printf("%d\n",ans);
	}
	return 0;
}
\end{cppcode}

\section{uva 10468 To carry or not to carry}
本題其實就是要計算兩個數的XOR。

\subsection{程式碼}
\begin{cppcode}
	#include <iostream>
	
	using namespace std;
	
	int main()
	{
		int a, b;
		while (cin >> a >> b) {
			cout << (a^b) << endl;
		}
		return 0;
	}
\end{cppcode}

\section{uva 10922 2 the 9s}
要計算一個整數是否為9的倍數,只要把所有的位數和算出來,看是否為9的倍數即可,這個過程可以不斷重覆,直到最後的位數和為個位數即可,重覆的過程稱為該數的9的級數。本題要求針對輸入的數字,計算是否為9的倍數,以及其級數。
\subsection{輸入格式}
每列一個數,位數可能很長,最後一列以0結束。
\subsection{輸出格式}
如果輸入的數是9的倍數,例如999999999999999999999,則輸出如下格式:
\begin{inside}
999999999999999999999 is a multiple of 9 and has 9-degree 3.
\end{inside}
如果輸入的數不是9的倍數,例如9999999999999999999999999999998,則輸出如下格式:
\begin{inside}
9999999999999999999999999999998 is not a multiple of 9.
\end{inside}
\subsection{解題思惟}

\subsection{程式碼}
\begin{cppcode}
#include <iostream>
#include <string>

using namespace std;

int main()
{
	string s;
	while (cin >> s) {
		if (s[0]=='0') break;
		int d=1, sum=s[0]-'0';
		if (s.length()>1) {
			for (int i=1; i<s.length(); i++) sum+=s[i]-'0';
			while (sum > 9) {
				int ns=0;
				while (sum) { ns+=sum%10; sum/=10; }
				d++;
				sum = ns;
			}
		}
		if (sum==9) cout << s << " is a multiple of 9 and has 9-degree " << d << ".\n";
		else cout << s << " is not a multiple of 9.\n";    
	}
	return 0;
}
\end{cppcode}

\section{uva 10931 Parity}
將輸入的整數列印成二進位形式,並計算其1的個數。
\subsection{輸入格式}
每列一個數,最後一列以0結束。
\subsection{輸出格式}
假設輸入的數為21,其輸出形式為:
\begin{inside}
The parity of 10101 is 3 (mod 2).
\end{inside}

\subsection{程式碼}
\begin{cppcode}
#include <iostream>

using namespace std;

int main()
{
	int n, mask, first, parity;
	while (cin >> n) {
		if (!n) break;
		first = 1; parity = 0;
		for (mask=0x40000000; mask; mask>>=1) {
			if (mask & n) {
				if (first) { cout << "The parity of "; first = 0; }
				cout << 1;
				parity++;
			} else if (!first) cout << 0;
		}
		cout << " is " << parity << " (mod 2)." << endl;
	}
	return 0;
}
\end{cppcode}

\section{uva 11150 Cola}
假設有n瓶可樂,喝完之後,每3個空瓶可再換一瓶,如果空瓶不夠,也可以先跟別人借,但要能夠還才行。問最多總共可以喝到幾瓶。
\subsection{解題思惟}
每兩個空瓶,可以先借一個空瓶來,湊成3個空瓶,就可以換一瓶來喝,並將空瓶還給別人。所以兩個空瓶可以視做一瓶。
\subsection{程式碼}
\begin{cppcode}
#include <iostream>

using namespace std;

int main()
{
	int n;
	while (cin >> n) cout << n + n/2 << endl;
	return 0;
}	
\end{cppcode}

\section{uva 11342 Three-Square}
輸入整數K,找到最小順序的a, b, c,使得$a\le b\le c$且$K=a^2+b^2+c^2$。

\subsection{輪入格式}
第一列的數字表示要測試的案例數,之後每一列一個數為K值。

\subsection{輸出格式}
如果該數可以分成三個數的平方和,則輸出a, b, c,如果不行的話,則輸出-1。

\subsection{解題思惟}
跑兩個迴圈處理,適當設定上限以減少迴圈次數。

\subsection{程式碼}
\begin{cppcode}
#include <iostream>
#include <cmath>

using namespace std;

int main()
{
	int n, k, smin1, smin2, a, b, c, found;
	cin >> n;
	while (n--) {
		cin >> k;
		smin1 = (int)sqrt((k+1)/3.0);
		found = 0;
		for (a=0; a<=smin1 && !found; a++) {
			smin2 = (int)sqrt((k+1-a*a)/2.0);
			for (b=a; b<=smin2 && !found; b++) {
				c = (int)sqrt(k-a*a-b*b+0.5);
				if (a*a+b*b+c*c==k) { found=1; break; }
			}
			if (found) a--;
		}
		if (found) cout << a << " " << b << " " << c << endl;
		else cout << "-1\n";           
	}
	return 0;
}
\end{cppcode}

\section{uva 11461 Square Numbers}
輸入a, b兩個整數 ($0<a\le b\le 100000$),算出兩個數中間共有幾個平方數(包括a和b)。

\subsection{輸入格式}
每列兩個數,表示a和b,最後一列以0 0結束。

\subsection{輸出格式}
每列一個數,表示a和b中間所有的平方數個數。

\subsection{解題思惟}
計算不小於$\sqrt{a}$的數以及不大於$\sqrt{b}$的數,兩數相減加1即可得出答案。

\subsection{程式碼}
\begin{cppcode}
#include <iostream>
#include <cmath>

using namespace std;

int main()
{
	int a, b;
	while (cin >> a >> b) {
		if (!a && !b) break;
		int m1 = (int)sqrt(a+0.5);
		if (m1*m1 < a) m1++;
		int m2 = (int)sqrt(b+0.5);
		cout << m2-m1+1 << endl;
	}
	return 0;
}
\end{cppcode}