\documentclass[12pt,a4paper]{article}

\usepackage{jyw-program}

\begin{document}
\title{APCS實作題參考解答}
\author{Jia-Yin Wang}
\maketitle

\begin{abstract}
這份講義提供105年10月份的APC實作題的參考解答,主要是給同學做為學習參考之用。基本上這些題目對初學的同學來說,有一些可能相當困難,那同學可以根據自己的情況,看能學到哪裡就學到哪裡,不懂的話也沒有關係。只要慢慢學習下去,將來有一天也可以融會貫通。

另外這裡所提供的參考解答,主要是從初學者容易理解的方式來尋找可行的辦法。基本上解決一個問題,常常有很多種可能的方式,所以還是要鼓勵同學繼續學習更多程式的技巧和演算方法,以後可能就會找到其他更好的解法。

學習程式務求完全理解,最好還能實際上機測試,否則好像看過看懂,實際上在作答或應用的時候,還是沒有辦法寫得出來。希望同學在閱讀的同時,可以多加思考,並且實際上機測試,以求完全了解。
\end{abstract}
%\section{10503-1 成績指標}

\subsection{解題思惟}
\begin{enumerate}
	\item 這一題首先要輸入n和n筆成績,這部份使用陣列來處理就可以了。
	\item 第二個部份是要印出排序好的成績,這部份可以自己寫排序的函數,例如氣泡排序法。如果要考慮到速度的話,那氣泡排序法的複雜度是$O(n^2)$,在n很大的時候,可能會有超時的問題,這時必須改用快速排序法等$O(n\log(n))$的演算法,不過這部份對初學的同學來說比較困難,但我們可以直接使用\cc{}裡面STL函數。基本上先引入<algorithm>檔頭,假設陣列名稱為a,且要排序的個數為n,直接呼叫std::sort(a, a+n)就可以排序了。
	\item 另外還要找出不及格的最高分,和及格的最低分的問題。這部份有兩種作法,一種是從排序好的數列中尋找,另一種是從原始數列中尋找。基本上後者就是從一堆數列中找最大值(或最小值)的方法,這部份對一般同學來說比較熟悉,基本概念就是逐一比過,有更大(或更小)的就替換掉就可以了,不過在比之前要先確定這個數目是不及格(或及格)才進行比較。
\end{enumerate}

\subsection{程式碼}
\begin{cppcode}
#include <iostream>
#include <algorithm>

using namespace std;

int main()
{
	int n, score[200], t, tmax, tmin;
	
	while (cin >> n) {
		tmax = -999;
		tmin = 999;
		for (int i=0; i<n; i++) {
			cin >> score[i];
			if (score[i]<60 && score[i]>tmax) tmax = score[i];
			if (score[i]>59 && score[i]<tmin) tmin = score[i];
		}
		sort(score, score+n);
		for (int i=0; i<n; i++) {
			if (i) cout << " ";
			cout << score[i];
		}
		cout << endl;
		if (tmax==-999) cout << "best case\n";
		else cout << tmax << endl;
		if (tmin==999) cout << "worst case\n";
		else cout << tmin << endl;
	}
	return 0;
}
\end{cppcode}
%\section{10503-2 矩陣轉換}

\subsection{解題思惟}
\begin{enumerate}
	\item 這一題要處理二維矩陣轉換的問題,基本上要處理的有兩個轉換,一個是翻轉,另一個是旋轉。題目說A矩陣經由一些轉換得到B,那已知B和轉換的運算,要求出A。那我們可以從B矩陣經由相反的轉換過程把A找出來。也就是說,假如$T_i$代表第$i$個轉換,那我們知道$T_nT_{n-1}\cdots T_1(A)=B$,反過來說$A=T_1^{-1}T_2^{-1}\cdots T_n^{-1}(B)$。這邊$T_i$是給定的翻轉和旋轉兩種轉換之一,那翻轉的反轉換還是翻轉,旋轉的反轉換也是旋轉(不同方向),所以只要先把兩種轉換用函數實現出來,這一題就容易解答了。
	\item 要怎麼表達矩陣呢?基本上可以使用二維陣列,因為題目給定的R和C都不會超過10,所以宣告a[10][10]就足夠使用了。那我們宣告大一點,但實際上只使用R$\times$C或C$\times$R的大小就可以了。另外也可以使用一維陣列,自己處理二維元素和一維元素位置對應的問題,這樣也是可以的。如果讀者熟悉\cc{}裡面的向量(vector),也可以使用向量的向量來處理這個問題。
	\item 此處二維陣列連同其維度變數,都一併宣告為全域變數,然後撰寫翻轉和旋轉的函數來解題。翻轉的部份實作較容易,假設矩陣大小為$R\times C$,基本上是把矩陣的第$i$列和第$R-i$列對調。那旋轉的部份呢?稍微思考推導一下,可以知道逆時針轉90度,其實就等於轉置(transpose)加上翻轉。那轉置的話,是令$B_{ij}=A_{ji}$,這是很容易實作的。至於轉換的運算過程,可以用一個陣列存起來,最後倒過來讀取並施行於矩陣$B$,最後就可以得到矩陣$A$了。
	\item 原題目只有一筆測資,在高中生解題平台中的題目則改為多筆測資,讀到檔尾結束。如果只有一筆測資,處理上較為單純,先讀取$r, c, m$,接著讀取矩陣的$r\times c$個元素,以及$m$個運算元,再繼續進行解題。那如果是多筆測資的話,把讀取$r, c, m$的部份放到while的判斷式中,並把接下來的讀取和解題放在while迴圈中即可,如下所示。
	\begin{inside}
		while (cin >> r >> c >> m) {
			// 讀取矩陣及運算元並進行解題。
		}
	\end{inside}		
\end{enumerate}

\subsection{程式碼}
\begin{cppcode}
#include <iostream>

using namespace std;

int r, c, b[10][10], m, op[10000]; // 矩陣及運算元變數

void flip(); // 翻轉
void rotate(); // 旋轉
void transpose(); // 轉置

int main()
{
	while (cin >> r >> c >> m) { // 這部份假設多筆測資,
		for (int i=0; i<r; i++) { // 雙重迴圈讀取矩陣
			for (int j=0; j<c; j++) cin >> b[i][j];
		}
		for (int i=0; i<m; i++) cin >> op[i]; // 運算元
		for (int i=m-1; i>=0; i--) { // 運算次序倒過來
			if (op[i]==0) rotate(); // 逆時針旋轉90度
			if (op[i]==1) flip(); // 翻轉
		}
		cout << r << " " << c << endl; // 輸出維度
		for (int i=0; i<r; i++) { // 輸出矩陣
			for (int j=0; j<c; j++) {
				if (j) cout << " ";
				cout << b[i][j];
			}
			cout << endl;
		}
	}
	return 0;
}

void flip()
{
	for (int i=0; i<r/2; i++) { // 處理上半與下半的對調
		for (int j=0; j<c; j++) swap(b[i][j], b[r-1-i][j]);
	}
}

void transpose()
{
	int t = max(r, c); // 取r,c中較大者
	for (int i=0; i<t; i++) { // 取右上半部元素
		for (int j=i+1; j<t; j++) {
			swap(b[i][j], b[j][i]); // Bij 和 Bji 對調
		}
	}
	swap(r, c); // 維度對調
}

void rotate()
{
	transpose(); // 逆時旋轉90度等於轉置加上翻轉
	flip();
}	
\end{cppcode}	


\section{10510-1 三角形辨別}

\subsection{解題思惟}
\begin{enumerate}
	\item 輸入三個數之後先排序。
	\item 排序之後依題示進行判斷即可。
\end{enumerate}

\subsection{程式碼}
\begin{cppcode}
#include <iostream>

using namespace std;

int main()
{
	int a, b, c;
	cin >> a >> b >> c;
	if (a>b) swap(a,b); // 此處三列用來排序
	if (b>c) swap(b,c);
	if (a>b) swap(a,b);
	cout << a << " " << b << " " << c << endl;
	if (a+b<=c) cout << "No";
	else if (a*a+b*b<c*c) cout << "Obtuse";
	else if (a*a+b*b==c*c) cout << "Right";
	else cout << "Acute";
	return 0;
}
\end{cppcode}

\newpage
\section{10510-2 最大和}

\subsection{解題思惟}
\begin{enumerate}
	\item 本題雖然是矩陣形式,但實際上只是要求每列的最大值,最後加總求和之後,並判斷各列最大值是否為和的因數。因此只要用一維陣列儲存每列的最大值即可。
	\item 求最大值的方法,可先預設最大值為0 (因為輸入的數都大於0),爾後每次讀取時比較並更新最大值即可。
\end{enumerate}

\subsection{程式碼}
\begin{cppcode}
#include <iostream>

using namespace std;

int main()
{
	int n, m, a[100], sum=0, k=0;
	cin >> n >> m;
	for (int i=0; i<n; i++) {
		int rmax = 0, num; // 預設最大值為0,num係用做讀取的變數
		for (int j=0; j<m; j++) {
			cin >> num;
			if (num > rmax) rmax = num; // 更新最大值
		}
		a[i] = rmax; // 儲存最大值
		sum += rmax; // 求和
	}
	cout << sum << endl;
	for (int i=0; i<n; i++) {
		if (sum % a[i] == 0) {
			if (k==0) k=1; // 列印第一個因數前面不加空白
			else cout << " "; // 其餘因數前面加空白
			cout << a[i];
		}
	}
	if (k==0) cout << -1; // 都沒有因數的話列印 -1
	return 0;
}
\end{cppcode}

\newpage
\section{10510-3 定時K彈}

\subsection{解題思惟}
\begin{enumerate}
	\item 這個題目可以使用陣列處理,每個位置儲存下一個數,然後某個位置爆掉的時候,簡單做一個跳過的連接處理就可以了。假設位置a接到b,位置b接到c,現在如果b爆掉的話,就把a接到c就好了。使用這樣的方式基本上就可以解題了,但是實際上測試的時候,會發現後面幾組測資都會超時,因為複雜度為$O(nm)$,並且n最大可能達到200000,m最大可能達到1000000的緣故。以下第一個程式碼會用此方式解題。
	\item 在上述的解題過程中,前進m步的時候,一個一個前進太慢,如果我們改用STL的vector來儲存數字,那麼前進m步只要直接計算位置索引就好了,這樣複雜度幾乎掉到$O(n)$應該就不會超時了。至於某個數引爆時,只要使用vector的erase函數將該數刪除即可。使用這個方法,對於STL的vector要有一些基本認識才行。以下第二個程式碼會使用此方式解題,可以通過所有測資。
	\item 這題有另外一種演算解法,這種題目稱為約瑟夫斯問題(Josephus problem),有興趣的讀者可以參閱\href{https://zh.wikipedia.org/wiki/%E7%BA%A6%E7%91%9F%E5%A4%AB%E6%96%AF%E9%97%AE%E9%A2%98}{wiki}上的說明。那這種問題的解法,基本上可以推導遞迴的關係來解題。先把數目1..n改成0..n-1,並假設後者的情況下,輸入n,m,k參數的解答為f(n,m,k)。很顯然f(n,m,0)=0 (沒有引爆彈);又m為n的倍數時,等於將n-1爆掉,爆掉之後,變成n-1個數的情況,所以答案為f(n-1,m,k-1);那如果m不是n的倍數呢?假設m除以n餘1,那麼等於0會爆掉,爆掉之後,等於還是n-1個數的情況,但每個數都加上了1,所以結果會等於1+f(n-1, m, k-1);那如果m除以n餘2呢?照理應該會等於2+f(n-1,m,k-1),但是這個數有可能超過n-1,所以最後還要求除以n的餘數,這樣類推下來可以得到下列遞迴公式:
		\begin{eqnarray*}
			f(n, m, 0) & = & 0 \\
			f(n, m, k) & = & ((m\ \%\ n) + f(n-1, m, k-1))\ \%\ n \\
			Ans & = & 1 + f(n, m, k)
		\end{eqnarray*}
	那上面這個遞迴公式,可以使用遞迴,或者改用迭代方式求解。
\end{enumerate}

\subsection{程式碼}
程式碼一:使用基本陣列方式解題,n和m比較大的時候會超時 (TLE)。
\begin{cppcode}
#include <iostream>

using namespace std;

int main()
{
	int n, m, k, next[200005];
	cin >> n >> m >> k;
	for (int i=1; i<n; i++) next[i] = i+1; // 每個數接到下一個數
	next[n] = 1; // 最後一個數n接到1
	
	int cnt = n, num = 1, lastnum = n; // cnt 為一個循環的個數
	while (k--) { // 引爆k次
		int p = (m-1) % cnt; // 前進的時侯,每個循環都會回到原位
		while (p--) { lastnum=num; num=next[num]; } // 前進m-1步
		next[lastnum] = next[num]; // 處理跳過num的連接
		num = next[num]; // 前進一步
		cnt--; // 循環個數減1
	}
	cout << num << endl;
	return 0;
}
\end{cppcode}

程式碼二:使用vector處理,可以通過所有測資。
\begin{cppcode}
#include <iostream>
#include <vector>

using namespace std;

int main()
{
	int n, m, k;
	while (cin >> n >> m >> k) {
		vector<int> v(n); // 宣告長度為n的vector
		for (int i=0; i<n; i++) v[i] = i+1; // 儲存啟始數字
		int idx = 0;
		while (k--) { // 引爆K次
			idx = (idx + m - 1) % v.size(); // 計算引爆點
			v.erase(v.begin() + idx); // 刪除引爆位置
		}
		idx = idx % v.size(); // v個數已減1,要重新計算位置
		cout << v[idx] << endl; // 輸出
	}
	return 0;
}
\end{cppcode}	

程式碼三:推導遞迴關係並使用迭代方式解題,可以通過所有測資。
\begin{cppcode}
#include <iostream>

using namespace std;

int main()
{
	int n, m, k, f=0;
	cin >> n >> m >> k;
	
	n = n-k; // 從 n-k 個球開始往上推算
	for (int i=0; i<k; i++) {
		n++; // 多一個球的狀況
		f = ((m%n) + f) % n; // 從n-1推n
	}
	cout << 1+f << endl;
	return 0;
}
\end{cppcode}


\newpage
\section{10510-4 棒球遊戲}

\subsection{解題思惟}
\begin{enumerate}
	\item 這題看起來覺得很難,但實際上仔細分析,會發現其實並不難,只是比較繁雜一點。
	\item 輸入的訊息用二維陣列儲存,因為每個事件的第一個字母都不相同,讀取之後存第一個字母就可以了,判斷上也會比較方便。
	\item 儲存輸入事件之後,依次取出判斷並更新狀態。基本上要更新的狀態包括各壘的情況、本局出局人數、總共出局人數、以及得分等資訊。
	\item 各壘的情況可用一個變數,以各位元代表各壘狀況,位元0表示一壘,位元1表示二壘,位元2表示三壘,那麼如果一三壘有人,這個數就是0b101=5。這樣表示的好處在哪裡呢?如果接下來擊出二壘打,就把所有位元往左移二個位元(或乘以4),那移完之後變成0b10100=20,第三個位元以後都表示得分,所以得分為0b10=2,或者20/8=2。此外第2個位元為1,表示三壘有人。所以出現安打的時候,只要處理位移,之後要計分或更新壘上的狀況都會很容易。
\end{enumerate}

\subsection{程式碼}
\begin{cppcode}
#include <iostream>

using namespace std;

int main()
{
	char event[9][5], ss[3];
	int k;
	for (int r=0; r<9; r++) {
		cin >> k; // 每位打手事件次數
		for (int i=0; i<k; i++) {
			cin >> ss; // 讀取事件
			info[r][i] = ss[0]; // 存第一個字元
		}
	}
	cin >> k; // 總出局人數
	
	int status = 0; // 各壘狀態
	int outplayer = 0; // 各局出局人數 
	int totalout = 0; // 總出局人數
	int score = 0; // 得分
	
	for (int i=0; i<5; i++) { // 事件最多五回合
		for (int r=0; r<9; r++) { // 打者順序
			switch (info[r][i]) { // 取出事件
				case '1': status = (status<<1) + 1; break; // 一壘打
				case '2': status = (status<<2) + 2; break; // 二壘打
				case '3': status = (status<<3) + 4; break; // 三壘打
				case 'H': status = (status<<4) + 8; break; // 全壘打
				case 'S': // 出局,繼續往下
				case 'F': // 出局,繼續往下
				case 'G': totalout++; outplayer++; break; // 出局
			}
			int sf = status >> 3; // 得分狀態
			status &= 7; // 更新壘上狀態
			while (sf) { // 計算得分狀態有幾個1,表示得幾分
				score += sf & 1;
				sf >>= 1;
			}
			if (totalout==k) { // 檢查總出局人數是否到達
				cout << score << endl; // 輸出積分
				return 0;
			}
			if (outplayer==3) { // 檢查該局出局人數是否達到3
				status = 0; // 重設狀態
				outplayer = 0; // 重設出局人數
			}
		}
	}
	
	return 0;
}
\end{cppcode}


\end{document}
