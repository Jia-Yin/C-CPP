\chapter{第六關:函數}

%1
\section{A010 畢氏數}
輸入一個正整數n,輸出所有的 $(a, b, c)$ 滿足
\begin{enumerate}
	\item $a, b, c$為三個正整數
	\item $a<b<c<n$
	\item $a^{2}+b^{2}=c^{2}$
	\item $a, b, c$ 三數的最大公因數為 1。
\end{enumerate}
\subsection{解題思惟}
\begin{enumerate}
	\item 
	先宣告一整數 $n$,接著再輸入 $n$ 值。
	\begin{inside}
		int n;
		cin >> n;
	\end{inside}
	\item
	再來寫一個三層迴圈分別代表 $a, b, c$ 的值,注意此處 $a<b<c<n$。
	\begin{inside}
	int n;
	cin >> n;
	for (int a=1; a<n; a++) {
		for (int b=a+1; b<n; b++) {
			for (int c=b+1; c<n; c++) {
				// ...
			}
		}
	}
	\end{inside}
	\item
	接著判斷是不是符合 $a^{2}+b^{2}=c^{2}$,如果不符合,則迴圈繼續跑。
	\begin{inside}
	for (int a=1; a<n; a++) {
		for (int b=a+1; b<n; b++) {
			for (int c=b+1; c<n; c++) {
				if (a*a+b*b != c*c) continue;	
			}
		}
	}
	\end{inside}
	\item 因為要判斷三個數的最大公因數是否為1,所以寫一個計算最大公因數的函式。
	\begin{inside}
	int gcd(int x, int y) {
		while (x%=y) swap(x, y); // 如果 x%y>0,交換 x, y 值
		return y;				
	}
	\end{inside}
	上面函式可以用來計算兩個數的最大公因數。如果要計算三個數$a, b, c$的最大公因數是否為1,
	可以使用 gcd(gcd($a, b$), $c$)==1 來判斷。
	\item 求$a, b, c$三數的最大公因數,也可以使用搜尋的方式,從$a$往下一直到1,看什麼時候可以找到$a, b, c$的公因數,如果一直到1才找到,那三個數的最大公因數就是1。
	\begin{inside}
		for (k=a; k>0; k--) if (a%k==0 && b%k==0 && c%k==0) break;
		if (k==1) ... // a, b, c 的最大公因數為1
	\end{inside}
\end{enumerate} 

\subsection{程式碼}
\begin{cppcode}
	#include <iostream>
	using namespace std;
	
	int gcd(int x, int y);
	
	int main() 
	{
		int n;
		cin >> n;
		for (int a=1; a<n; a++) {
			for (int b=a+1; b<n; b++) {
				for (int c=b+1; c<n; c++) {
					if (a*a+b*b != c*c) continue;
					if (gcd(gcd(a, b), c) != 1) continue;
					cout << a << "," << b << "," << c << endl;	
				}
			}
		}
		return 0;
	}

	int gcd(int x, int y) {
		while(x%=y) swap(x, y); // 如果 x%y>0,交換 x, y 值
		return y;				
	}
\end{cppcode}

%2
\section{A011 列舉質因數}
輸入一正整數N,輸出所有N的質因數。

\subsection{解題思維}

\begin{enumerate}
	\item
	先宣告一整數 N,接接著跑一個從 2 到 N 的迴圈來找出 N 的因數。
	\begin{inside}
	int n;
	cin >> n;
	for (int i=2; i<=n; i++) {
		if (n%i) continue; // 如果n除以i餘數不為0,則i非因數,迴圈繼續
		... // i是n的因數
	}
	\end{inside}
	\item
	找到N的一個因數$i$,還要判斷它是否為質數,我們可以再跑一個迴圈,讓$j$從2開始往上遞增,看何時才能成為$i$的因數,如果在$i$之前就可以找到,表示它不是質數,如果要等到$j=i$時才是$i$的因數,則$i$為質數。
	\begin{inside}
	int n, j;
	cin >> n;
	for (int i=2; i<=n; i++) {
		if (n%i) continue; // 如果n除以i餘數不為0,則i非因數,迴圈繼續
		for (j=2; j<i; j++) if (i%j==0) break;
		if (j==i) cout << i << endl;
	}
	\end{inside}
	\item 本題也可以另外寫一個函式,用來判斷一個整數是否為質數,方法可以參照A009那題。
	
\end{enumerate} 

\subsection{程式碼}
\begin{cppcode}
	#include <iostream>
	using namespace std;
	
	int main()
	{
		int n, i, j;
		cin >> n;
		for (i=2; i<=n; i++) {
			if (n%i) continue;
			for (j=2; j<i; j++) if (i%j==0) break;
			if (j==i) cout << i << endl;
		}
		return 0;
	}
\end{cppcode}

%3
\section{A056 遞迴計算最大公因數-計算函式呼叫次數}
遞迴計算最大公因數-計算函式呼叫次數\\輸入兩個正整數xy,輸出依序為函式呼叫次數及最大公因數。\\其gcd函式依照以下準則撰寫: \\
1. 如果 x小於0 改計算 y 與 -x 的最大公因數 \\
2. 如果 x大於y 改計算 y 與 x 的最大公因數 \\
3. 如果 x等於0 回傳y \\
4. 回傳 y%x 與 x 的最大公因數 \\
在過程中,你需要紀錄函式呼叫的次數。

\subsection{解題思維}

\begin{enumerate}
	\item
	首先撰寫gcd遞迴函式:
	\begin{inside}
	int gcd(int x, int y) {
		if(x<0) return gcd(y, -x); 
		//如果 x 小於 0,藉由回傳 (y, -x) 來改計算 y 與 -x 的最大公因數
		if(x>y) return gcd(y, x);
		//如果 x 大於 y,藉由回傳 (y, x) 來改計算 y 與 x 的對大公因數 
		if(x==0) return y; //如果 x 等於 0 回傳 y
		return gcd(y%x, x); //其他情況則回傳 (y%x, x) 來求最大公因數
	}	
	\end{inside}
	\item 要計算gcd函式被呼叫的次數,常用的方式有二,第一個是使用全域變數,第二個是使用區域變數。
	\item 使用全域變數的話,在main函式的上面宣告變數並給啟始值0,之後每次呼叫gcd函式的時候,把該變數加1。
	\begin{inside}
	int cnt = 0;
	int main() {...}
	int gcd(int x, int y) {
		cnt++;
		...
	}
	\end{inside}
	\item 使用區域變數的話,在gcd函式裡面宣告靜態變數並給啟始值0,每次呼叫gcd函式時,把該數加1。靜態變數的啟始值只有第一次會被執行,並且離開函式的時候,它的內容不會消失,因此可以用來計算呼叫次數。要注意的是,該區域變數屬於gcd函式,因此只有在gcd函式中才可以使用此變數,離開該函式便無法引用。
	\begin{inside}
	int gcd(int x, int y) {
		static int cnt = 0;
		cnt++;
		...
	}
	\end{inside}
	\item 如果熟悉指標(pointer),或參照(reference),也可以把計算次數的變數設為函式的參數之一,並用在呼叫函式的時候將該變數加1。
	\begin{inside}
	int gcd(int x, int y, int &c) {
		c++;
		if(x<0) return gcd(y, -x, c);
		if(x>y) return gcd(y, x, c);
		if(x==0) return y;
		return gcd(y%x, x, c);
	}	
	\end{inside}
\end{enumerate} 

\subsection{程式碼}
\begin{cppcode}
	#include <iostream>
	using namespace std;
	
	int cnt = 0;
	int gcd(int x, int y);
	
	int main()
	{
		int x, y;
		cin >> x >> y;
		int ans=gcd(x, y);
		cout << cnt << endl;
		cout << ans;
		return 0;
	}

	int gcd(int x, int y) {
		cnt++;
		if(x<0) return gcd(y, -x);
		if(x>y) return gcd(y, x);
		if(x==0) return y;
		return gcd(y%x, x);
	}	
\end{cppcode}

%4
\section{A057 遞迴算費式數列-計算函式呼叫次數}
費氏數列 f(0)=0, f(1)=1, f(n)=f(n-1)+f(n-2);請依照此定義計算費氏數列,並計算函式呼叫次數。本題輸入為一個正整數 n ,輸出為函式呼叫次數及 f(n)。

\subsection{解題思維}

\begin{enumerate}
	\item
	首先先寫費氏數列的遞迴函式 f(n)。
	\begin{inside}
	int fn(int n) {
		if(n<2) return n; // n為0回傳0,n為1回傳1
		return fn(n-1)+fn(n-2); //其他情況則回傳 fn(n-1)+fn(n-2)
	}	
	\end{inside}
	\item
	計算函式呼叫次數的方式,如上一題A056的方式處理即可。
\end{enumerate} 

\subsection{程式碼}
\begin{cppcode}
	#include <iostream>
	using namespace std;

	int cnt = 0;	
	int fn(int n);

	int main()
	{
		int n;
		cin >> n;
		int ans = fn(n);
		cout << cnt << endl;
		cout << "f(n)=" << ans;
		return 0;
	}
	
	int fn(int n) {
		if(n==0) return 0;
		if(n==1) return 1;
		return fn(n-1)+fn(n-2);
	}
\end{cppcode}

%5
\section{A059 遞迴計算N階乘}
輸入一正整數N,輸出N!。
\subsection{解題思維}

\begin{enumerate}
	\item 本題要求使用遞迴函式,假設函式為fact(n),因為 n! = n * (n-1)!,所以 fact(n) = n * fact(n-1),另外必須要有一個終止條件。
	\item 如果n為正整數,可以設n等於1為終止條件,直接回傳1即可。如果輸入包括0,則應該設0為終止條件,回傳1 (0!=1)。	
	\item 本題雖然題目說是正整數,但實際上測資會出現0,故使用0為終止條件。
\end{enumerate} 

\subsection{程式碼}
\begin{cppcode}
	#include <iostream>

	using namespace std;

	int fact(int n);
	
	int main()
	{
		int n;
		cin >> n;
		cout << fact(n);
		return 0;
	}

	int fact(int n) {
		if (n==0) return 1;
		return n * fact(n-1);
	}
\end{cppcode}

%6
\section{F031 函式-印n次}
輸入一整數n,運用print(n)印出n次``i love C++"。
\subsection{解題思維}

\begin{enumerate}
	\item
	print 函式,有一個輸入 n,並且輸出 n 次 ``i love C++":
	\begin{inside}
	void print (int n) {
		while(n--) cout << "i lovd C++\n";	
	}
	\end{inside}
	\item
	完成 print 函式後再代入主程式使用就好。
\end{enumerate} 

\subsection{程式碼}
\begin{cppcode}
	#include <iostream>

	using namespace std;
	
	void print(int n);
	
	int main()
	{
		int n;
		cin>>n;
		print(n);
		return 0;
	}
	
	void print(int n) {
		while(n--) cout << "i love C++\n";
	}
\end{cppcode}

%7
\section{G003 最大公因數(三數)}
輸入三個整數,輸出其最大公因數。
\subsection{解題思維}

\begin{enumerate}
	\item
	要求三個數的最大公因數,我們可以先求其中兩個數的最大公因數,再跟第三數求最大公因數。至於求兩個數的最大公因數,有很多解法,前面亦有討論,此處以遞迴方式求解。
	\begin{inside}
	int gcd(int x, int y) {
		if(x<0) return gcd(y, -x);
		if(x==0) return y;
		return gcd(y%x, x);
	}
	\end{inside}
	
	\item
	完成 gcd 函式後再代入主程式使用就好。
\end{enumerate} 

\subsection{程式碼}
\begin{cppcode}
	#include <iostream>

	using namespace std;
	
	int gcd(int x,int y);
	
	int main()
	{
		int n, a, b, c;
		cin >> a >> b >> c;
		cout << gcd(gcd(a, b), c);
		
		return 0;
	}

	int gcd(int x,int y) {
		if(x<0) return gcd(y, -x);
		if(x==0) return y;
		return gcd(y%x, x);
	}
\end{cppcode}

%8
\section{M90H008 函式疊代f(f(f(a)))}
假設$f(x)=(x+3)/(x+2)$,輸入一整數 a,求f(f(f(a)))。
\subsection{解題思維}

\begin{enumerate}
	\item
	只要照著題目的條件去寫 fun 函式即可。
	\begin{inside}
	double fun(double a) {
		return (a+3)/(a+2);
	}
	\end{inside}
	\item
	完成 fun 函式後再代入主程式使用就好。
\end{enumerate} 

\subsection{程式碼}
\begin{cppcode}
	#include <iostream>

	using namespace std;
	
	double fun(double a);
	
	int main()
	{
		double a;
		cin >> a;
		cout << "f(f(f(" << a << ")))=" << fun(fun(fun(a)));
		
		return 0;
	}
	
	double fun(double a) {
		return (a+3)/(a+2);
	}
\end{cppcode}

%9
\section{M90H009 函式疊代f(f(f(a)))}
假設$f(x)=(x+3)*(x+1)/(x+2)$,輸入一整數a,求f(f(f(a)))。
\subsection{解題思維}

\begin{enumerate}
	\item
	只要照著題目的條件去寫 fun 函式即可,但應該注意的是,本題輸入為整數,但使用函式計算的結果需為實數,其次又要再次呼叫函式,故應該把輸入輸出都設為浮點數,考量比較好的精確度,應該宣告為double。
	\begin{inside}
	double f(double a) {
		return (a+3)*(a+1)/(a+2);
	}
	\end{inside}
	\item 在主程式中使用三次呼叫即可。
\end{enumerate} 

\subsection{程式碼}
\begin{cppcode}
	#include <iostream>

	using namespace std;
	
	double f(double a);
	
	int main()
	{
		int a;
		cin >> a;
		cout << "f(f(f(" << a << ")))=" << f(f(f(a)));
		
		return 0;
	}
	
	double f(double a) {
		return (a+3)*(a+1)/(a+2);
	}
\end{cppcode}

%10
\section{M90H016 有k個正因數之最小數為何}
輸入一整數k,至少有k個正因數之最小正整數。
\subsection{解題思維}

\begin{enumerate}
	\item
	我們先寫一個 factors 函式,來計算某數k有多少個因數。基本上就是從1到k跑一個迴圈,看看有幾個數是k的正因數。
	\begin{inside}
	int factors(int k) {
		int n=0;
		for (int i=1; i<=k; i++) {
			if (k%i==0) n++;
		}
		return n;
	}
	\end{inside}
	\item
	完成 factors 函式後,接著就可以在主程式跑一個迴圈,找看看哪一個數的因數個數大於等於 k。
\end{enumerate} 

\subsection{程式碼}
\begin{cppcode}
	#include <iostream>

	using namespace std;
	
	int factors(int k);
	
	int main()
	{
		int k;
		cin >> k;
		for (int i=1; ; i++) {
			if (factors(i) >= k) {
				cout << i;
				break;
			}
		}
		return 0;
	}
	
	int factors(int k) {
		int n=0;
		for (int i=1; i<=k; i++) {
			if (k%i==0) n++;
		}
		return n;
	}
\end{cppcode}

%11
\section{M90H026 平面幾何-三角形面積}
輸入三角形三個點(ax, ay), (bx, by), (cx, cy),輸出其面積。
\subsection{解題思維}
\begin{enumerate}
	\item 已知三角形三點座標,要求此三角形面積的方法,可用向量外積方式簡略推導,如果讀者不能明瞭,可以查看一般教科書或網路上查詢其他說明。
	\item 假設兩個邊向量分別為$(a,b),\ (c,d)$,可以將其看成兩個三維向量
	$(a,b,0)$及$(c,d,0)$,其外積為
	$$\left|\begin{array}{ccc}
	\vec{i} & \vec{j} & \vec{k} \\
	a & b & 0 \\
	c & d & 0 \\
	\end{array}\right| = (ad-bc)\ \vec{k}$$
	因為外積為兩個向量形成的平行四邊形面積,所以兩個邊所形成的三角形面積會變成$|ad-bc|/2$。
	\item 已知三角形三個點,可以求兩個邊向量,再用上面的公式求面積。假設三點座標為(ax, ay), (bx, by), (cx, cy),則兩個向量邊為(bx-ax, by-ay)以及(cx-ax, cy-ay),故面積為0.5*abs((bx-ax)*(cy-ay)-(cx-ax)*(by-ay)),撰寫函式如下:
	\begin{inside}
	double tri(int ax, int ay, int bx, int by, int cx, int cy) {
		return 0.5*abs((bx-ax)*(cy-ay)-(by-ay)*(cx-ax));
	}		
	\end{inside}
	\item
	完成後再代入主程式。
\end{enumerate} 

\subsection{程式碼}
\begin{cppcode}
	#include <iostream>

	using namespace std;
	
	double tri(int ax, int ay, int bx, int by, int cx, int cy);
	
	int main(){
		int ax, ay, bx, by, cx, cy;
		cin >> ax >> ay >> bx >> by >> cx >> cy;
		cout << tri(ax, ay, bx, by, cx, cy);
		return 0;
	}
	
	double tri(int ax, int ay, int bx, int by, int cx, int cy) {
		return 0.5*abs((bx-ax)*(cy-ay)-(by-ay)*(cx-ax));
	}		
\end{cppcode}

%12
\section{M90H027 平面幾何-三角形周長}
輸入三角形三個點(ax, ay), (bx, by), (cx, cy),輸出其周長。
\subsection{解題思維}

\begin{enumerate}
	\item
	我們先寫一個求兩點連線長度的函式 len,有四個輸入分別代表兩個的點 x, y 座標。
	\begin{inside}
	double len(int ax, int ay, int bx, int by) {
		return sqrt((bx-ax)*(bx-ax)+(by-ay)*(by-ay)); 
	}			
	\end{inside}
	\item
	完成後再代入主程式,再把三個距離長度加起來即可。
\end{enumerate} 

\subsection{程式碼}
\begin{cppcode}
	#include <iostream>
	#include <cmath>

	using namespace std;
	
	double len(int ax, int ay, int bx, int by);
	
	int main()
	{
		int ax, ay, bx, by, cx, cy;
		cin >> ax >> ay >> bx >> by >> cx >> cy;
		cout << len(ax, ay, bx, by) + len(ax, ay, cx, cy) + len(bx, by, cx, cy);
		return 0;
	}
	
	double len(int ax, int ay, int bx, int by) {
		return sqrt((bx-ax)*(bx-ax)+(by-ay)*(by-ay));
	}
\end{cppcode}

%13
\section{M90H035 平面幾何-點到線之距離}
輸入平面上的點$P=(u, v)$及直線$L: ax+by=c$共五個參數$u, v, a, b, c$,輸出點$P$到直線$L$的最小距離。
\subsection{解題思維}
\begin{enumerate}
	\item 
$ax+by=c$的法向量為$(a,b)$,假設從$(u,v)$加上$k$倍法向量$(a,b)$落在直線$L$上,則
$$ a(u+ka) + b(v+kb) = c $$
所以
$$ k = \frac{c-au-bv}{a^2+b^2} $$
故點$P$到直線$L$的距離為
$$ |k\sqrt{a^2+b^2}| = |c-au-bv| / \sqrt{a^2+b^2}$$
	\item
	依上面公式,可以寫出一點到直線距離的函式 len:
	\begin{inside}
	double len(int u, int v, int a, int b, int c) {
		return abs(a*u+b*v-c)/sqrt(a*a+b*b);
	}
	\end{inside}
\end{enumerate} 

\subsection{程式碼}
\begin{cppcode}
	#include <iostream>
	#include <cmath>

	using namespace std;
	
	double len(int u, int v, int a, int b, int c);
	
	int main()
	{
		int u, v, a, b, c;
		cin >> u >> v >> a >> b >> c;
		cout << len(u, v, a, b ,c);
		return 0;
	}
	
	double len(int u, int v, int a, int b, int c) {
		return abs(a*u+b*v-c)/sqrt(a*a+b*b);
	}
\end{cppcode}

%14
\section{M90H037 平面幾何-點到點之距離}
輸入兩個點,輸出這兩個點的距離。
\subsection{解題思維}

\begin{enumerate}
	\item 這一題比較簡單,不一定要使用函式來解。但這一節主要練習函式,所以我們先寫一個計算兩點連線距離的len函式,這個函式有四個整數輸入,分別代表兩個點的 x, y 座標。
	\begin{inside}
	double len(int ax, int ay, int bx, int by) {
		...
	}
	\end{inside}
	\item
	接著把公式寫出來。
	\begin{inside}
	#include <cmath>
		
	double len(int ax, int ay, int bx, int by) {
		return sqrt((bx-ax)*(bx-ax)+(by-ay)*(by-ay));
	}			
	\end{inside}
	\item
	完成後再代入主程式。
\end{enumerate} 

\subsection{程式碼}
\begin{cppcode}
	#include <iostream>
	#include <cmath>

	using namespace std;
	
	double len(int ax, int ay, int bx, int by);
	
	int main()
	{
		int ax, ay, bx, by;
		cin >> ax >> ay >> bx >> by;
		cout << len(ax, ay, bx, by);
		
		return 0;
	}
	
	double len(int ax, int ay, int bx, int by) {
		return sqrt((bx-ax)*(bx-ax)+(by-ay)*(by-ay));
	}
\end{cppcode}

%15
\section{M90H042 複數絕對值}
輸入一個複數,輸出其絕對值。
\subsection{解題思維}

複數$a+b i$的絕對值定義為$\sqrt{a^2+b^2}$ = sqrt$(a^2+b^2)$。本題觀察輸入都是整數,所以可以把輸入宣告為整數。因為使用到sqrt函式,必須引入<math.h>或<cmath>。

\subsection{程式碼}
\begin{cppcode}
	#include <iostream>
	#include <cmath>

	using namespace std;
	
	int main()
	{
		int a, b;
		cin >> a >> b;
		cout << sqrt(a*a+b*b);
		return 0;
	}
\end{cppcode}

%16
\section{M90H048 級數-k方和}
輸入整數k和n,輸出$1^{k}+2^{k}+3^{k}+\cdots+n^{k}$。
\subsection{解題思維}

\begin{enumerate}
	\item
	我們先寫求次方 pow 函式,有兩個輸入分別代表整數及次方數。
	\begin{inside}
	int pow(int k, int n) {
		int ans=1;
		for (int i=0; i<k; i++) ans *= n;
		return ans;
	}
	\end{inside}
	\item
	完成後代入主程式。先宣告整數 n 及次方數 k,接著跑一個迴圈來求題目所需的答案。
	\begin{inside}
		int main()
		{
			int k, n, sum=0;
			cin >> k >> n;
			for (int i=1; i<=n; i++) sum += pow(k, i);
							
			return 0;
		}			
	\end{inside}
\end{enumerate} 

\subsection{程式碼}
\begin{cppcode}
	#include <iostream>

	using namespace std;
	
	int pow(int k, int n);
	
	int main()
	{
		int k, n, sum=0;
		cin >> k >> n;
		for(int i=1; i<=n; i++) sum += pow(k, i);
		cout << sum;
		return 0;
	}
	
	int pow(int k, int n) {
		int ans=1;
		for(int i=1; i<=k; i++) ans *= n;
		return ans;
	}
\end{cppcode}

%17
\section{M90H058 數列 $a(n+1)=(n+1)/(n+2)*a(n),\ a(1)=1$}
考慮數列 $a(n+1)=(n+1)/(n+2)*a(n),\ a(1)=1$。\\輸入一整數n,輸出a(n)。
\subsection{解題思維}

\begin{enumerate}
	\item 本題可以使用迴圈或者遞迴處理,但題目限制不可出現for, while等關鍵字,所以必須使用遞迴。
	\item 遞迴函式直接根據定義撰寫即可。
	\begin{inside}
	double a(int n) {
		if(n==1) return 1.0;
		return a(n-1)*n/(n+1);
	}			
	\end{inside}
\end{enumerate} 

\subsection{程式碼}
\begin{cppcode}
	#include <iostream>

	using namespace std;
	
	double a(int n);
	
	int main()
	{
		double n;
		cin >> n;
		cout << "a(" << n << ")=" << a(n);
		return 0;
	}
	
	double a(int n) {
		if(n==1) return 1.0;
		return a(n-1)*n/(n+1);
	}
\end{cppcode}

%18
\section{M90H059 數列 $a(1)=3,\ 5a(n+1)=4a(n)+1$}
考慮數列: $a(1)=3,\ 5a(n+1)=4a(n)+1$。\\ 
輸入一整數n,輸出a(n)。
\subsection{解題思維}

\begin{enumerate}
	\item
	我們先寫 a 函式,使用遞迴方式,按題目所給的條件,終止條件是$a(1)=3$,其他則是$a(n)=(4a(n-1)+1)/5$。照著條件我們便可以完成函式
	\begin{inside}
		double a(int n)
		{
			if(n==1) return 3;
			return (4*a(n-1)+1)/5.0;
		}			
	\end{inside}
	\item 
	完成後代入主程式。
\end{enumerate} 

\subsection{程式碼}
\begin{cppcode}
	#include <iostream>

	using namespace std;
	
	double a(int n);
	
	int main()
	{
		double n;
		cin >> n;
		cout << endl << "a(" << n << ")=" << a(n);
		return 0;
	}
	
	double a(int n) {
		if(n==1) return 3;
		return (4*a(n-1)+1)/5.0;
	}
\end{cppcode}

