\chapter{第十關:結構}

%1
\section{結構宣告}
完成結構宣告,使程式正常執行。

\subsection{解題思維}

宣告一結構Student,其成員變數為m\_name和m\_score。

\subsection{程式碼}
\begin{cppcode}
	#include<iostream>
	
	using namespace std;
	struct Student {
		string m_name;
		int m_score;
	};
	
	int main() 
	{
		struct Student stu;
		cout<<endl<<"Name:";
		cin>>stu.m_name;
		cout<<endl<<"score:";
		cin>>stu.m_score;
		cout<<endl<<stu.m_name<<" gets score: "<<stu.m_score;
		return 1;
	}
\end{cppcode}

%2
\section{結構初始化}
完成結構初始化,使程式正常執行。

\subsection{解題思維}

在定義結構時,預先設好成員變數的初始值。

\subsection{程式碼}
\begin{cppcode}
	#include<iostream>
	
	using namespace std;
	struct Student {
		char m_name[128] = "Arping";
		int m_score = 99;
	};
	
	int main() 
	{
		struct Student stu;
		cout<<endl<<stu.m_name<<" gets score: "<<stu.m_score;
		return 1;
	}
\end{cppcode}

%3
\section{結構物件複製}
完成結構複製,使程式正常執行。

\subsection{解題思維}

將輸入的資料存到結構x的欄位變數,再將x的資料複製到y,並輸出y的資料。

\subsection{程式碼}
\begin{cppcode}
	#include<iostream>
	
	using namespace std;
	struct Student {
		char m_name[128];
		int m_score;
	};
	
	int main()
	{
		struct Student x, y;
		cin>>x.m_name;
		cin>>x.m_score;
		y = x;
		cout<<endl<<y.m_name<<" gets score: "<<y.m_score;
		return 1;
	}
\end{cppcode}

%4
\section{函數參數結構化(傳身)}
將函數參數用結構傳入,使程式正常執行。

\subsection{解題思維}

寫一個函數,以傳身的方式傳遞參數,完成成績加10的功能。 

\subsection{程式碼}
\begin{cppcode}
	#include<iostream>
	
	using namespace std;
	struct Student {
		char m_name[128];
		int m_score;
	};
	
	void Add10(Student & stu)
	{
		stu.m_score = stu.m_score + 10;
	}
	
	int main()
	{
		struct Student stu;
		cin>>stu.m_name;
		cin>>stu.m_score;
		Add10(stu);
		cout<<endl<<stu.m_name<<" gets score: "<<stu.m_score;
		return 1;
	}
\end{cppcode}

%5
\section{函數參數結構化(傳址)}
將函數參數用結構傳入,使程式正常執行。
\subsection{解題思維}

寫一個函數,以傳址的方式傳遞參數,完成成績加10的功能。

\subsection{程式碼}
\begin{cppcode}
	#include<iostream>
	
	using namespace std;
	struct Student {
		char m_name[128];
		int m_score;
	};
	
	void Add10(Student * a)
	{
		a->m_score += 10;
	}
	
	int main()
	{
		struct Student stu;
		cin>>stu.m_name;
		cin>>stu.m_score;
		Add10(&stu);
		cout<<endl<<stu.m_name<<" gets score: "<<stu.m_score;
		return 1;
	}
\end{cppcode}