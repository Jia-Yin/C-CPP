\chapter{第十一關:類別}

%1
\section{類別:建構子}
完成程式寫作

\subsection{解題思維}
\begin{enumerate}
	\item 建構子 \emph{(Constructor)} 是指和類別名稱相同的方法。當我們建立新物件時,程式會自動執行建構子,常被用來做物件參數初始化的動作。
	
	\item 物件的建構子有以下三個特點:
	\begin{itemize}
		\item 與類別名稱相同。
		\item 沒有回傳值。
		\item 支援方法過載。
	\end{itemize}
	有了概念之後我們就可以開始解題。
	
	\item 從題目預先給的程式碼來看,題目已經在Student類別中建立了兩個建構子,一個建構子傳數字元指標,另一個則多傳入了兩個 int 參數。
	\begin{inside}
	class Student{
		public:
		Student(char*){
			cout<<endl<<"Student(char*);building a Student";
		}
		Student(char*, int, int) {
			cout<<endl<<"Student(char*,int,int);building a Student";
		}
	};
	\end{inside}
	\item 接著我們看題目給的 main 程式碼。
	\begin{inside}
	int main(){
		int n;
		cin>>n;
		Student me("arping");
		Student you("Alice",99,77);
		Student* pstu=new Student[n];
		delete[] pstu;
		return 0;
	}		
	\end{inside}
	我們可以發現除了剛剛講的兩個建構子之外,main裡面還多呼叫了一個沒有任何參數的建構子,所以我們要把這個建構子加入Student類別中。
	\item 再來只要依照輸出要求來建立建構子即可。
	\begin{inside}
	class Student{
		public:
		Student(char*){
			cout<<endl<<"Student(char*);building a Student";
		}
		Student(char*, int, int) {
			cout<<endl<<"Student(char*,int,int);building a Student";
		}
		Student() {
			cout<<endl<<"Stuendt();building a Student";
		}
	};
	\end{inside}
\end{enumerate}


\subsection{程式碼}
\begin{cppcode}
	#include <iostream>
	using namespace std;
	
	class Student{
		public:
		Student(char*) {
			cout<<endl<<"Student(char*);building a Student";
		}
		Student(char*, int, int) {
			cout<<endl<<"Student(char*, int, int);building a Student";
		}
		Student() {
			cout<<endl<<"Student();building a Student";
		}
	};
	
	int main(){
		int n;
		cin>>n;
		Student me("arping");
		Student you("Alice",99,77);
		Student* pstu=new Student[n];
		delete[] pstu;
		return 0;
	}
	
\end{cppcode}


%2
\section{類別:靜態成員變數}
完成程式寫作

\subsection{解題思維}
	\begin{enumerate}
	\item 題目要求我們建立一個靜態成員變數Total,當我們建立新物件時來計算學生數量。
	\item 再寫程式碼之前,我們先來了解靜態成員:
	\begin{itemize}
		\item 使用static宣告的變數稱為靜態變數。
		\item 靜態成員屬於類別,不屬於個別的物件。
		\item 和全域變數一樣,靜態變數也是永久儲存的。
	\end{itemize}
	\item 進一步舉例,假設我們見一個Circle類別,而每個不同的Circle都有著相同的圓周率${\pi}$
	 ,也就是不同的物件會有相同的資料成員,這時我們就可以將${\pi}$設為靜態成員,使${\pi}$變成大家共享的資料成員。
	
	\item 接著我們開始建立靜態成員total,且在每次建立一個新的物件時total都會加一。 
	\begin{inside}
	class Student{
		public:
		static int total;
		Student(char*) {
			total++;
			cout<<endl<<"Student(char*);building a Student;Total:"<<total;
		}
		Student(char*, int, int) {
			total++;
			cout<<endl<<"Student(char*,int,int);building a Student;Total:"<<total;
		}
		Student() {
			total++;
			cout<<endl<<"Student();building a Student;Total:"<<total;
		}
	};
	int Student::total=0;
	\end{inside}
	\end{enumerate}
\subsection{程式碼}
\begin{cppcode}
	#include <iostream>
	using namespace std;
	
	class Student{
		public:
		static int total;
		Student(char*) {
			total++;
			cout<<endl<<"Student(char*);building a Student;Total:"<<total;
		}
		Student(char*, int, int) {
			total++;
			cout<<endl<<"Student(char*,int,int);building a Student;Total:"<<total;
		}
		Student() {
			total++;
			cout<<endl<<"Student();building a Student;Total:"<<total;
		}
	};
	int Student::total=0;
	
	int main(){
		int n;
		cin>>n;
		Student me("arping");
		Student you("Alice",99,77);
		Student* pstu=new Student[n];
		delete[] pstu;
		return 0;
	}
	
\end{cppcode}

%3
\section{類別:靜態成員函數}
完成程式寫作

\subsection{解題思維}
\begin{enumerate}
	\item 與靜態成員類似,我們也可以把函式設為靜態成員函式,而靜態成員函式通常是拿來作為工具函式,如我們在circle類別中加上一個計算圓面積的函式cirArea()。
	\item 題目要求我們建立size及count兩個靜態函式,回傳同樣的都是total的值,所以我們只要照著題意去寫程式即可。
	\begin{inside}
	class Student{
		public:
		static int total;
		Student(char*) {
			total++;
			cout<<endl<<"Student(char*);make a Student;";
		}
		Student(char*, int, int) {
			total++;
			cout<<endl<<"Student(char*,int,int);make a Student;";
		}
		Student() {
			total++;
			cout<<endl<<"Student();make a Student;";
		}
		static int size() {
			return total;
		}
		static int count() {
			return total;
		}
	};
	int Student::total=0;	
	\end{inside}
\end{enumerate} 

\subsection{程式碼}
\begin{cppcode}
	#include <iostream>
	using namespace std;
	
	class Student{
		public:
		static int total;
		Student(char*) {
			total++;cout<<endl<<"Student(char*);make a Student;";
		}
		Student(char*, int, int) {
			total++;cout<<endl<<"Student(char*,int,int);make a Student;";
		}
		Student() {
			total++;cout<<endl<<"Student();make a Student;";
		}
		static int size() {
			return total;
		}
		static int count() {
			return total;
		}
	};
	int Student::total=0;
	
	int main(){
		int n;
		cin>>n;
		Student me("arping");cout<<endl<<"make "<<Student::size()<<" student(s)";
		Student you("Alice",99,77);cout<<endl<<"make "<<you.count()<<" student(s)";
		Student* pstu=new Student[n];cout<<endl<<"make "<<Student::size()<<" student(s)";
		delete[] pstu;
		return 0;
	}
	
	
\end{cppcode}
