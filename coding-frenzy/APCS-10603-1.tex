\section{10603-1 秘密差}

\subsection{題目}
將一個十進位正整數的奇數位數的和稱為$A$,偶數位數的和稱為$B$,則$A$與$B$的絕
對差值$|A-B|$稱為這個正整數的秘密差。
例如:263541 的奇數位數的和 $A = 6+5+1 = 12$,偶數位數的和 $B = 2+3+4 = 9$,所以
263541 的秘密差是$|12-9|= 3$。
給定一個十進位正整數$X$,請找出$X$的秘密差。

\subsection{解題思惟}
\begin{enumerate}
	\item 將一個正整數$X$的每一個位數拆開來,可以從個位開始,將$X$除以10所得到的餘數,就是個位,然後把$X$除以10得到的商,就是$X$去掉個位之後形成的數。把新的數再重覆以上的步驟,就可以得到十位,依此類推。一般而言,可以使用以下程式碼的框架:
	\begin{minted}{c}
	while (X) { // 也可以寫 X>0,假設X=12345
		d = X % 10; // d 是取出的個位數,依次會得到5,4,3,2,1
		X /= 10; // 把 X 的個位去掉
		... // 處理取出的個位數
	}
	\end{minted}
	\item 本題必須把奇數位的位數和偶數位的位數分開。這可以設一個變數flag,每次處理一個位數的時候,就把flag的值切換一次,例如比較常用的方式,提把flag設為1,每次切換可以設flag=1-flag,這樣就會得到1,0,1,0,1的切換值。然後根據flag是1或0,可以決定是奇數位或偶數位的位數。
	\item 本題把奇數位和偶數位的位數和算出之後,再相減得到差值,並取其正數就是答案了。
\end{enumerate}

\subsection{程式碼}
\begin{cppcode}
#include <stdio.h>

int main()
{
	int x, odd=0, even=0, flag=1;
	scanf("%d", &x);
	while (x) {
		if (flag) odd += x%10;
		else even += x%10;
		x /= 10;
		flag = 1-flag;
	}
	x = odd-even;
	if (x<0) x = -x;
	printf("%d\n", x);
	return 0;
}
\end{cppcode}
