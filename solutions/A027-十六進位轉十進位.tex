\section{A027 - 十六進位轉十進位}
輸入5個十六進位字母 0...9A...F,輸出輸出該數的十進位。

\subsection{解題思惟}
\begin{enumerate}
	\item 第一次讀一個字元,算出其十進位的值,接著讀第二個字元,算出該兩個字元十進位的值,依此類推,最後可以得到答案。
	\item 單一個字元轉成十進位的話,如果是0-9,值也是0-9,如果是A-F,值是10-15。所以當字元為0-9時,將字元減去`0'字元就可以得到相對的數目,如果是文字的話,減掉`A'字元加上10,就可以得到相對的數目。文字的部份,要注意大小寫。
	\item 將前一次所算出的結果乘以16,加上新進來的個位數,就可以得到下一個更新的值。總共重覆5次,就可以得到答案。
\end{enumerate}

\subsection{程式碼}
\begin{cppcode}
	#include <iostream>
	using namespace std;
	int main()
	{
		int ans=0;
		char c;
		for(int i=0; i<5; i++) {
			cin >> c;
			ans *= 16;
			if (c>='0' && c<='9') ans += c-'0';
			else if (c>='A' && c<='F') ans += c-'A'+10;
			else if (c>='a' && c<='f') ans += c-'a'+10;
		}
		cout<<ans;
		return 0;
	}
\end{cppcode}
