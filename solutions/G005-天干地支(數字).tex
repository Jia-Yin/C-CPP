\section{天干地支(數字)}
輸入一個數,依其除以10的餘數輸出"甲乙丙丁戊己庚辛壬癸"中之一字。甲(0)乙(1)丙(2)丁(3)戊(4)己(5)庚(6)辛(7)壬(8)癸(9) 依其除以12的餘數輸出"子丑寅卯辰巳午未申酉戌亥"之一字。子(0)丑(1)寅(2)卯(3)辰(4)巳(5)午(6)未(7)申(8)酉(9)戌(10)亥(11)。

\subsection{解題思維}
\begin{enumerate}
\item 本題可以使用分支或陣列的方式來加以解答,使用分支比較直接,但程式碼又臭又長;使用陣列則比較精簡有力。以下分別就兩種方式加以說明。
\item 使用分支的方式,可以用巢狀if-else,或者用switch來處理。以天干為例,前者的方式大致如下:
\begin{inside}
int t = n%10;
if (t==0) cout << "甲";
else if (t==1) cout << "乙";
else if (t==2) cout << "丙";
...
\end{inside}
後者的方式如下:
\begin{inside}
switch (n%10) {
case 0: cout << "甲"; break;
case 1: cout << "乙"; break;
case 2: cout << "丙"; break;
...
}
\end{inside}
地支的處理方式雷同。
\item 使用陣列的方式,是把天干和地支存在陣列中,然後根據給定的數算出要輸出的字元的位置,再把該位置的字元印出來。一般而言,繁體中文的Windows使用的中文編碼為Big5,每一個中文字佔用兩個字元(bytes),所以根據得到的餘數,把它乘以2,就可以得到相對位置。以天干為例,處理方式如下:
\begin{inside}
char T[] = "甲乙丙丁戊己庚辛壬癸";
int t = 2*(n%10);
cout << T[t] << T[t+1];
\end{inside}
另外,我們也可以將天干地支儲存成字串陣列,也就是用二維陣列來處理;或者在\cc{}中也可以
利用string陣列或者向量來處理。以下提供C的字串陣列供作參考。
\begin{inside}
char T[][3] = {"甲", "乙", "丙", "丁", "戊", 
			"己", "庚", "辛", "壬", "癸"}; // 2個字元的字串須佔用3個位元組!
cout << T[n%10];
\end{inside}

\subsection{程式碼}
\begin{cppcode}
#include <iostream>

using namespace std;

int main()
{
	char T[] = "甲乙丙丁戊己庚辛壬癸";
	char D[] = "子丑寅卯辰巳午未申酉戌亥";
	int n, t, d;
	cout << "\n請輸入一個數字:\n";
	cin >> n;
	t=2*(n%10), d=2*(n%12); // 計算天干地支字元開始的位置
	cout << "其代表天干地支為:" << T[t] << T[t+1] << D[d] << D[d+1];
	return 0;
}
\end{cppcode}
