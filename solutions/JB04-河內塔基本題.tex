\section{河內塔基本題}
依河內塔規則,將環從A移到C,B為輔助。輸入環的個數n,輸出所有移動過程。

\subsection{解題思惟}
本題使用遞迴解題較為容易。假設 hanoi(int n, char from, char to, char buf) 代表可以將n個環,從from移到to,且暫存為buf的函式。當n為1的時候,直接移動即可,
若n>1,則可以先將n-1個環從from移到buf,然後移動第n個環,再將n-1個環從buf移到to,如下:
\begin{inside}
	hanoi(n-1, from, buf, to);
	cout << from << " => " << to << endl;
	hanoi(n-1, buf, to, from);
\end{inside}	
			
\subsection{程式碼}
\begin{cppcode}
#include <iostream>

using namespace std;

void hanoi(int n, char from, char to, char buf);

int main()
{
	int n;
	cin >> n;
	hanoi(n, 'A', 'C', 'B');
	return 0;
}

void hanoi(int n, char from, char to, char buf)
{
	if (n==1) {
		cout << from << " => " << to << endl;
	} else {
		hanoi(n-1, from, buf, to);
		cout << from << " => " << to << endl;
		hanoi(n-1, buf, to, from);
	}
}
\end{cppcode}
