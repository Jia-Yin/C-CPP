\section{M90H064 - 根極限 $x^3+2x^2-3=0$ 之根}
\label{M90H064}
使用double型別,計算 $x^3+2x^2-3=0$ 之根。

\subsection{解題思惟}

令函式$f(x)=x^3+2x^2-3$,使用while迴圈不斷迭代尋找x,當$|f(x)-0|$介於誤差範圍時,結束迴圈並輸出x。在此提供兩種找x的方法,分別是「二分堪根法」以及「牛頓法」,詳細內容可以參見一般教科書或到網路上尋找說明。

\begin{enumerate}
	\item 二分勘根法
	當$f(a)\times f(b)<0$成立的時候,表示此函式至少有一根介於a、b之間。使用二分法不斷縮小a、b之間的範圍,最後就能找到根。
	\item 牛頓法:
	設函式$f'(x)$ 是$f(x)$的微分,牛頓法使用迭代式 $x_{n+1} = x_n - f(x_n)/f'(x_n)$,用此方法不斷迭代新的x,最後就能找到根。	
\end{enumerate}
此處當$|f(x)|<10^{-7}$時,即輸出此近似根。
\subsection{程式碼}
二分堪根法:
\begin{cppcode}
	#include <iostream>
	
	using namespace std;
	
	double f(double x);
	
	int main()
	{
		double a=1, b=2, x;
		while (true) {
			// 二分堪根法
			x = (a+b)/2;
			if(f(x) > -1E-7 && f(x) < 1E-7) break;
			if(f(a)*f(x) <= 0) b = x;
			if(f(b)*f(x) <= 0) a = x;
		}
		cout << x;		
		return 0;
	}
	
	double f(double x)
	{
		return (x*x*x)+(2*x*x)-3;
	}	
\end{cppcode}

牛頓法:

\begin{cppcode}
	#include <iostream>
	
	using namespace std;
	
	double f(double x);
	double df(double x);
	
	int main()
	{
		//牛頓切根法
		double x=1;		
		x=1;
		while (true) {
			if(f(x) > -1E-7 && f(x) < 1E-7) break;
			x = x-(f(x)/df(x));
		}
		cout << x;	
		return 0;
	}
	
	double f(double x)
	{
		return (x*x*x)+(2*x*x)-3;
	}
	
	double df(double x)
	{
		return (3*x*x)+(4*x);
	}
	
\end{cppcode}
