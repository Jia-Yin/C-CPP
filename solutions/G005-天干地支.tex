\section{G005 - 天干地支(數字)}
輸入一個數,依其除以10的餘數輸出``甲乙丙丁戊己庚辛壬癸"中之一字,
甲(0)乙(1)丙(2) 丁(3)戊(4)己(5)庚(6)辛(7)壬(8)癸(9);
依其除以12的餘數輸出
``子丑寅卯辰巳午未申酉戌亥"之一字,
子(0)丑(1)寅(2)卯(3)辰(4)巳(5)午(6)未(7)申(8)酉(9)戌(10)亥(11)。 
\subsection{解題思維}
使用陣列存入天干地支,使用\%取餘數判斷輸入數字之排序。
\subsection{程式碼}
\begin{cppcode}
#include <iostream>

using namespace std;

int main()
{	char T[][3]={"甲","乙","丙","丁","戊","己","庚","辛","壬","癸"};
	char D[][3]={"子","丑","寅","卯","辰","巳","午","未","申","酉","戌","亥"};
	cout << endl <<"請輸入一個數字:";
	int n;
	cin >>n;
	int t=n%10,d=n%12;
	cout << endl << "其代表天乾地支為:";
	cout << T[t] << D[t];
	return 0;
}
\end{cppcode}

