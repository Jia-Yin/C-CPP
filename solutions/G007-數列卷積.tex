\section{G007 - 數列卷積}
已知 f(0)=1 且 
f(n)=f(0)*f(n-1)+f(1)*f(n-2)+f(2)*f(n-3)+f(3)*f(n-4)+...+f(n-1)*f(0)。 \\
f(0)=1 \\
f(1)=f(0)*f(0)=1 \\
f(2)=f(0)*f(1)+f(1)*f(0)=2 \\
f(3)=f(0)*f(2)+f(1)*f(1)+f(2)*f(0)=5 \\
f(4)=f(0)*f(3)+f(1)*f(2)+f(2)*f(1)+f(3)*f(0)=14 \\
f(5)=f(0)*f(4)+f(1)*f(3)+f(2)*f(2)+f(3)*f(1)+f(4)*f(0)=42 \\
輸入一正整數 n , 輸出 f(n) 。
\subsection{解題思惟}
\begin{enumerate}
	\item 本題可以使用迭代法或遞迴來解。
	\item 使用迭代法的話,我們用陣列來存函式f的值,假設所求不超過1000,可以宣告f[1000],並設初始值f[0]=1。之後依次用公式求解f[i]的值,i從1到n。
	\item 使用遞迴的話,要設終止條件並依公式求解f[n]。在求解過程中,很多函式值會被重複計算
	多次,如果希望增加程式效率,可以把計算過的f[n]存起來,需要用到的時候,先判斷是否算過,
	如果算過直接取用,否則再用公式計算。
\end{enumerate}

\subsection{程式碼}
迭代解法:
\begin{cppcode}
#include <cstdio>

int main()
{
	int n, f[1000]={1};
	scanf("%d", &n);
	for (int i=1; i<=n; i++) {
		for (int j=0; j<i; j++) f[i] += f[j]*f[i-1-j];
	}
	printf("f(%d)=%d", n, f[n]);
	return 0;
}
\end{cppcode}
遞迴解法:
\begin{cppcode}
#include <cstdio>

int f(int n);

int main()
{
	int n;
	scanf("%d", &n);
	printf("f(%d)=%d", n, f(n));
	return 0;
}

int f(int n)
{
	if (n==0) return 1;
	int sum=0;
	for (int i=0; i<n; i++) sum += f(i)*f(n-1-i);
	return sum;
}
\end{cppcode}
