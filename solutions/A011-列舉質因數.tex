\section{A011 列舉質因數}
輸入一正整數N,輸出所有N的質因數。

\subsection{解題思維}

\begin{enumerate}
	\item
	先宣告一整數 N,接接著跑一個從 2 到 N 的迴圈來找出 N 的因數。
	\begin{inside}
	int n;
	cin >> n;
	for (int i=2; i<=n; i++) {
		if (n%i) continue; // 如果n除以i餘數不為0,則i非因數,迴圈繼續
		... // i是n的因數
	}
	\end{inside}
	\item
	找到N的一個因數$i$,還要判斷它是否為質數,我們可以再跑一個迴圈,讓$j$從2開始往上遞增,看何時才能成為$i$的因數,如果在$i$之前就可以找到,表示它不是質數,如果要等到$j=i$時才是$i$的因數,則$i$為質數。
	\begin{inside}
	int n, j;
	cin >> n;
	for (int i=2; i<=n; i++) {
		if (n%i) continue; // 如果n除以i餘數不為0,則i非因數,迴圈繼續
		for (j=2; j<i; j++) if (i%j==0) break;
		if (j==i) cout << i << endl;
	}
	\end{inside}
	\item 本題也可以另外寫一個函式,用來判斷一個整數是否為質數,方法可以參照A009那題。
	
\end{enumerate} 

\subsection{程式碼}
\begin{cppcode}
	#include <iostream>
	using namespace std;
	
	int main()
	{
		int n, i, j;
		cin >> n;
		for (i=2; i<=n; i++) {
			if (n%i) continue;
			for (j=2; j<i; j++) if (i%j==0) break;
			if (j==i) cout << i << endl;
		}
		return 0;
	}
\end{cppcode}
