\section{F006 - 字元輸入及顯示}
輸入一個字元,輸出此字元及其 ASCII 碼,接著再輸出 ASCII 表上的下一個字元及其 ASCII 碼。

\subsection{解題思維}
\begin{enumerate}
\item 字元在電腦中一般是以一個八位元的整數來儲存,符號與其對應的數字一般使用 American Standard Code for Information Interchange (ASCII) 編碼。在 C/\cc{}中,可以將此編碼視為一個八位元的整數,也可以視為一個字元。使用 printf 函式,可以透過指定格式字 \%d 與 \%c 在兩者中切換輸出。
\item 要輸入一個字元,我們要先宣告一個型態是 char 的變數,接著輸入字元。
	\begin{inside}
		char s;
		scanf("%c", &s);
	\end{inside}
\item 使用 printf 函式及指定格式字 \%d 與 \%c,可以將其輸出為整數或字元型態。
	\begin{inside}
		printf("%d %c", s, s);
	\end{inside}
\end{enumerate}
\newpage
\subsection{程式碼}
\begin{cppcode}
	#include <cstdio>
	
	int main()
	{
		char s;
		scanf("%c", &s);
		printf("%c:%d\n", s, s);
		printf("%c:%d", s+1, s+1);
		return 0;
	}
\end{cppcode}
註:本題在瘋狂程設測試時,不能通過,原因是因為測試字元前面會多一個空白,這是測資引發的問題。
修正方式是先讀一次字元,也就是把程式中 scanf 那一列重複兩次即可。
