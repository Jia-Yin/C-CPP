\section{M90H035 平面幾何-點到線之距離}
輸入平面上的點$P=(u, v)$及直線$L: ax+by=c$共五個參數$u, v, a, b, c$,輸出點$P$到直線$L$的最小距離。
\subsection{解題思維}
\begin{enumerate}
	\item 
$ax+by=c$的法向量為$(a,b)$,假設從$(u,v)$加上$k$倍法向量$(a,b)$落在直線$L$上,則
$$ a(u+ka) + b(v+kb) = c $$
所以
$$ k = \frac{c-au-bv}{a^2+b^2} $$
故點$P$到直線$L$的距離為
$$ |k\sqrt{a^2+b^2}| = |c-au-bv| / \sqrt{a^2+b^2}$$
	\item
	依上面公式,可以寫出一點到直線距離的函式 len:
	\begin{inside}
	double len(int u, int v, int a, int b, int c) {
		return abs(a*u+b*v-c)/sqrt(a*a+b*b);
	}
	\end{inside}
\end{enumerate} 

\subsection{程式碼}
\begin{cppcode}
	#include <iostream>
	#include <cmath>

	using namespace std;
	
	double len(int u, int v, int a, int b, int c);
	
	int main()
	{
		int u, v, a, b, c;
		cin >> u >> v >> a >> b >> c;
		cout << len(u, v, a, b ,c);
		return 0;
	}
	
	double len(int u, int v, int a, int b, int c) {
		return abs(a*u+b*v-c)/sqrt(a*a+b*b);
	}
\end{cppcode}
