\section{A014 - 次方求餘}
輸入三個正整數 n, p, d,輸出 n 的 p 次方除以 d 的餘數。
\subsection{解題思維}
\begin{enumerate}
\item 要求 n 的 p 次方,我們可以先設一個整數變數 m = 1,接著跑一個重覆 p 次的迴圈,每次將 m 乘以 n 來求出 n 的p次方。
	\begin{inside}
	int m = 1;
	for (int i=0; i<p; i++) m *= n;
	\end{inside}
\item 求得 n 的 p 次方後,接著再取餘數。
\item 以上的想法雖然正確,但實際依此方法解答的時候,卻會發生錯誤。原因是因為整數溢位的問題,基本上,n 的 p 次方可能會是一個很大的整數,所以必須考慮溢位的狀況。在數學上,求兩個數 a 和 b 的乘積除以 d 的餘數,可以先求 a 除以 d 的餘數,以及 b 除以 d 的餘數,將兩個餘數相乘,再除以 d 求餘數,這樣求出的答案是相同的。使用這樣的方法,可以避免掉溢位的問題。所以本題取餘數要跟求次方同時進行,才不會發生錯誤!
	\begin{inside}
	for (int i=0; i<p; i++) { 
		m *= n;
		m %= d;
	}
	\end{inside}	
\end{enumerate}

\subsection{程式碼}
\begin{cppcode}
	#include <iostream>
	
	using namespace std;
	
	int main()
	{
		int n, p, d;
		int m = 1;
		cin >> n >> p >> d;
		for (int i=0; i<p; i++) {
			m *= n;
			m %= d;
		}
		cout << m;
		return 0;
	}
\end{cppcode}

\subsection{延伸思考}
在求$n^p$的時候,我們也可以把它分成兩半,也就是兩個$n^{p/2}$相乘。如果$p$是奇數的話,那我們可以先求$n^{p-1}$次方,然後再將結果乘以$n$。使用這樣的概念,當我們求$n^p$除以$d$的餘數的時候,實際上只要先計算$n^{p/2}$除以$d$的餘數就可以了。這樣的話,當$p$是一個很巨大的數目的時候,計算量可以大大降低。其次,我們在計算$n^{p/2}$的時候,又可以先計算$n^{p/4}$,所以每次可以把次冪數折一半,計算出來之後,再把它自乘一次。不斷使用這種二分法,可以得到非常高效率的演算法,當然這樣的程式寫法會變得稍微複雜一點。那這裡先提供這樣的思惟角度和程式碼供讀者參考,不能了解也沒有關係,等以後學到遞迴的時候,讀者可以回過頭來看這個題目,就會壑然開朗了。
\begin{cppcode}
	#include <iostream>
	
	using namespace std;
	
	int npd(int n, int p, int d); // 計算n的p次方除以d的餘數
	
	int main()
	{
		int n, p, d;
		cin >> n >> p >> d;
		cout << npd(n, p, d);
		return 0;
	}
	
	int npd(int n, int p, int d)
	{
		if (p==1) return n % d; // 遞迴終止條件
		int k = npd(n, p/2, d);
		if (p%2) return (n*k*k) % d; // p is odd
		else return (k*k) % d; // p is even
	}
\end{cppcode}	

