\section{G006西元問天干地支}
已知西元1024年為甲子年,輸入一個西元年份,輸出其天干地支的年別。

\subsection{解題思維}
\begin{enumerate}
\item 本題可使用兩個一維陣列來分別儲存天干和地支,須注意在中文Windows中,每個中文字佔用兩個字元(bytes)。
\item 因為西元1024年為甲子年,甲子的位置都在陣列的第0個字元,所以輸入的年份可以先減去1024,得到相對於西元1024年的相對值。注意當輸入年份小於1024的時候,相對值會是負的。由於天干地支的共同最小週期是60,我們可以先把相對值加上60*20=1200,天干地支不會改變。
\item 接下來把相對的值除以10和12,分別求其餘數,即代表天干和地支相對的字元位置,但須注意,因為中文字佔用兩位字元大小,所以換算成陣列位置的時候,要先乘以2,並且輸出的時候,一次要輸出兩個字元。
\end{enumerate}
\subsection{程式碼}
\begin{cppcode}
#include <iostream>

using namespace std;

int main()
{
	char T[] = "甲乙丙丁戊己庚辛壬癸";
	char D[] = "子丑寅卯辰巳午未申酉戌亥";
	cout << endl << "請輸入西元年:";
	int n, m;
	cin >> n;
	m = n-1024+1200;
	int t=2*(m%10), d=2*(m%12); // 計算天干地支字元開始的位置
	cout << endl << "西元" << n << "年為 ";
	cout << T[t] << T[t+1] << D[d] << D[d+1] << " 年";
	return 0;
}
\end{cppcode}

\noindent 本題也可以將天干地支儲存成字串陣列,也就是二維陣列來處理;另外在\cc{}中也可以
存成string陣列或向量來處理。以下提供C的字串陣列供作參考。
\begin{cppcode}
	#include <iostream>
	
	using namespace std;
	
	int main()
	{
		char T[][3] = {"甲", "乙", "丙", "丁", "戊", 
			"己", "庚", "辛", "壬", "癸"}; // 2個字元的字串須佔用3個位元組!
		char D[][3] = {"子", "丑", "寅", "卯", "辰", "巳",
			"午", "未", "申", "酉", "戌", "亥"};
		cout << endl << "請輸入西元年:";
		int n, m;
		cin >> n;
		m = n-1024+1200;
		int t=m%10, d=m%12; // 計算天干地支字串開始的位置
		cout << endl << "西元" << n << "年為 ";
		cout << T[t] << D[d] << " 年";
		return 0;
	}
\end{cppcode}
