\section{A036 百數最小值所在地}
輸入100個正整數,將這些數最小值的一個所在地印出。
\subsection{解題思維}
\begin{enumerate}
	\item 本題雖然輸入100個正整數,但只是要找到一個最小值的位置,可以使用陣列也可以不使用陣列解答。依次說明。
	\item 使用陣列的話,可以先假設最小值的位置在0的位置,然後依次從位置1到99,找出陣列元素來和最小位置的值相比,如果更小的話,就把最小值位置更新。
	\begin{inside}
		int idx=0;
		for (int i=0; i<100; i++) scanf("%d", &a[i]);
		for (int i=1;  i<100; i++) {
			if (a[i]<a[idx]) idx=i;
		}
	\end{inside}
	\item 不使用陣列的話,因為輸入的次序依次為第0...99的位置的數字,所以位置可以直接使用迴圈索引,只要把最小值存起來就可以了,不用存其他值。
	\begin{inside}
		int idx, tmin, d;
		for (int i=0; i<100; i++) {
			scanf("%d", &d);
			if (i==0 || d<tmin) { idx=i; tmin=d; }
		}
	\end{inside}
\end{enumerate}

\subsection{程式碼}
\begin{cppcode}
	#include <stdio.h>
	
	int main()
	{
		int idx, tmin, d;
		
		for (int i=0; i<100; i++) {
			scanf("%d", &d);
			if (i==0 || d<tmin) { idx=i; tmin=d; }
		}
		
		printf("%d", idx);
		
		return 0;
	}
\end{cppcode}
