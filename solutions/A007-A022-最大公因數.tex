\section{A007\&A022 - 最大公因數}
輸入兩個正整數x, y,輸出x, y的最大公因數。

\subsection{解題思惟}
求最大公因數有很多方法,這裡先講解一個簡單的暴力法,主要的概念是從最大的可能值往下找,找到的第一個共同的因數就是最大公因數。
\begin{enumerate}
\item 假設x<y,如果不是的話,將兩個數交換。最大公因數不會大於x。
\item 跑一迴圈變數i從x到1,第一個發現是x和y的公因數的i就是最大公因數。
\item 其實第一點可以省略,只是當x很大而y很小的時候,會多跑很多次迴圈而已。
\end{enumerate}

\subsection{程式碼}
\begin{cppcode}
#include <iostream>

using namespace std;

int main()
{
	int x, y, i;
	cin >> x >> y;
	for (i=x; i>=1; i--) {
		if (x%i==0 && y%i==0) break;
	}
	cout << i;
	return 0;
}
\end{cppcode}

\subsection{解題思惟}
接著要講述的方法是輾轉相除法。
\begin{enumerate}
\item 如果x除以y整除的話,那麼y就是最大公因數。
\item 如果不能整除的話,可以先算出x除以y的餘數,這個數肯定比y還小,接著把y變成新的x,把剛才的餘數變成新的y,再回到上一個步驟。
\end{enumerate}
\subsection{程式碼}
\begin{cppcode}
#include <iostream>

using namespace std;

int main()
{
	int x, y, t;
	cin >> x >> y;
	while (x%y) { // x % y != 0
		x = x%y;
		t=x; x=y; y=t; // swap(x, y);
	}
	cout << y;
	return 0;
}
\end{cppcode}
註:本題在瘋狂程設測試時,沒有辦法通過。仔細檢查,發現輸入的數有的是負數,不像題目所說的都是正整數,因此會產生錯誤。修正的辦法,是在輸入x, y整數之後,加上負數的判斷並改成正數,也就是下面兩行:
\begin{inside}
if (x<0) x = -x;
if (y<0) y = -y;
\end{inside}
這樣的話,就沒有問題了。
