\section{M90H052 - 極限 $(2n+3)/(3n-1)$}
\label{M90H052}
請使用double型別,計算n趨近無限大時,$(2n+3)/(3n-1)$的逼近值。

\subsection{解題思惟}
當n趨近無限大時,$(2n+3)/(3n-1)$會慢慢逼近收斂值,因為浮點數有一定的精確度限制,
所以當n超過一個很大的數字以後,代入計算的值就不會動了,
我們可以利用這個特點結束求解的迴圈。程式流程如下:
\begin{enumerate}
	\item 宣告兩個double型別的變數ans及next,分別表示$n$及下一個$n$計算的結果。
	\item 使用while迴圈重複比較ans與next是否相等,若不相等則加大n並計算新的ans。
	當ans與temp相等時,結束while迴圈,並輸出答案ans。
	把n加大的時候,不一定每次加1,這裡我們每次加1000以加快收斂速度。
\end{enumerate}


\subsection{程式碼}
\begin{cppcode}
#include <iostream>

int main()
{
	double n=1.0, ans=0.0, next;
	
	next = (2*n+3)/(3*n-1);
	while (ans != next) {
		ans = next;
		n += 1000;
		next = (2*n+3)/(3*n-1);
	}
	std::cout << ans;
	return 0;
}
\end{cppcode}
