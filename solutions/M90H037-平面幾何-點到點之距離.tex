\section{M90H037 平面幾何-點到點之距離}
輸入兩個點,輸出這兩個點的距離。
\subsection{解題思維}

\begin{enumerate}
	\item 這一題比較簡單,不一定要使用函式來解。但這一節主要練習函式,所以我們先寫一個計算兩點連線距離的len函式,這個函式有四個整數輸入,分別代表兩個點的 x, y 座標。
	\begin{inside}
	double len(int ax, int ay, int bx, int by) {
		...
	}
	\end{inside}
	\item
	接著把公式寫出來。
	\begin{inside}
	#include <cmath>
		
	double len(int ax, int ay, int bx, int by) {
		return sqrt((bx-ax)*(bx-ax)+(by-ay)*(by-ay));
	}			
	\end{inside}
	\item
	完成後再代入主程式。
\end{enumerate} 

\subsection{程式碼}
\begin{cppcode}
	#include <iostream>
	#include <cmath>

	using namespace std;
	
	double len(int ax, int ay, int bx, int by);
	
	int main()
	{
		int ax, ay, bx, by;
		cin >> ax >> ay >> bx >> by;
		cout << len(ax, ay, bx, by);
		
		return 0;
	}
	
	double len(int ax, int ay, int bx, int by) {
		return sqrt((bx-ax)*(bx-ax)+(by-ay)*(by-ay));
	}
\end{cppcode}
