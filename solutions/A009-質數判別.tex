\section{A009 - 質數判別}
輸入一個正整數,如果是質數,則輸出 Yes,如果不是,則輸出 No。

\subsection{解題思惟}
\begin{enumerate}
\item 質數的特色是除了1之外,另外一個唯一的因數就是n本身。所以從1檢查到n,應該有2個因數。以下的程式應該就可以判斷了。
	\begin{inside}
	int cnt=0;
	for (i=1; i<=n; i++) if (n%i==0) cnt++; // 檢查到因數,把個數加1。
	if (cnt==2) cout<<"Yes"; else cout<<"No";
	\end{inside}
\item 也可以只檢查2, …, n-1是否為n的因數,如果其中有一個因數,那n就不是質數了,否則n就是質數。在這種情況下,1和2應該要另外處理,1不是質數,但2是質數。
\item 可以設定一個旗標變數,如果沒有因數存在,把值設為1 (表示n為質數),當有任一個因數存在時,把它設為0 (表示n非質數)。預設值可以為1,檢查到因數時設為0。像以下這樣:
	\begin{inside}
	int flag=1;
	for (i=2; i<n; i++) if (n%i==0) { flag=0; break; }
	\end{inside}
\item 綜合2-3兩點,也可以用以下方式解此問題。
	\begin{inside}
	int flag=1; // 預設flag為1
	if (n==1) flag=0; // 1不是質數
	else if (n>3) { // 2和3是質數,flag不用變動,處理n>3情況即可
		for (i=2; i<n; i++) if (n%i==0) { flag=0; break; }
	}
	\end{inside}
\item  進階思考:實際上檢查$n$的因數只要檢查到$\sqrt{n}$即可,因為如果有個因數$x$超過$\sqrt{n}$,那麼$n/x$就小於$\sqrt{n}$,會先被檢查到。另外$i\le\sqrt{n}$可以改成$i*i\le n$。
\item 偶數不是質數,也可以先過濾掉,奇數的部份,只要檢查有沒有奇數的因數就可以了。
\end{enumerate}

\subsection{程式碼}
\begin{cppcode}
	#include <iostream>
	
	using namespace std;
	
	int main()
	{
		int i, n, flag=1;
		cin >> n;
		if (n==1) flag=0; // 1 非質數
		else if (n>3) { // 2和3是質數,flag不用變動,處理n>3情況即可
			if (n%2==0) flag=0; // 偶數非質數
			else for (i=3; i*i<=n; i+=2) { // 只檢查奇數i,範圍i*i<=n
				if (n%i==0) { flag=0; break; } // 發現因數,設定旗標跳出
			}
		}
		if (flag) cout << "Yes";
		else cout << "No";
		return 0;
	}
\end{cppcode}
