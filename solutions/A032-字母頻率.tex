\section{A032 字母頻率}
輸入英文段落,求印各字母頻率,未出現者略過不印。

\subsection{解題思維}

\begin{enumerate}
	\item
	因為在C語言中,每個字元都對應到一個數字(請參照ASCII Code表),其數字範圍不超過256,因此我們可以使用大小為256的陣列來儲存每個字的出現頻率,以該字元的ASCII Code作為陣列索引值,而對應的值表示該字元的出現頻率。(ex:A的ASCII碼為65,所以f[65]就是A出現的頻率)
	\item 每次讀入一個字元,即使用該字元的值做為陣列索引,將該索引值位置的值加1。
	\item 印出時以字母當索引值,檢查對應的值是否大於0,如果大於0,則將其對應值印出來。
\end{enumerate} 

\subsection{程式碼}
\begin{cppcode}
	#include <stdio.h>
	
	int main()
	{
		int f[256] = { 0 }; // 設定所有對應值為0
		char ch;
		
		while (scanf("%c", &ch) == 1) { // 讀到檔尾
			f[ch]++;
		}
	
		for (int i='A'; i<='Z'; i++) { // A..Z的頻率
			if (f[i]) printf("%c:%d\n", i, f[i]);
		}
	
		for (int i='a'; i<='z'; i++) { // a..z的頻率
			if(f[i]) printf("%c:%d\n", i, f[i]);
		}
	
		return 0;
	}
	
\end{cppcode}
