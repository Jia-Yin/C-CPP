\section{A039 列印前N個質數}
輸入一個正數N(N<1024),列印前N個質數。
\subsection{解題思維}

\begin{enumerate}
	\item 這個題目要列印的質數個數不超過1024個,所以數字不會太大。處理的方式可以不使用陣列,也可以使用陣列。
	\item 如果不使用陣列的話,只要從2開始一直往上數,然後將是質數的列出來,並計算個數就可以了,主要是要判斷一個數是否為質數,這可以寫一個函式來處理,前面已經出現過類似的題目。
	\begin{inside}
		int cnt=0,;
		for (int i=2; cnt<n; i++) {
			if (isprime(i)) {
				printf("%d ", i);
				cnt++;
			}
		}
	\end{inside}
	\item 那如果使用陣列的話,我們可以從小到大,把目前為止已知的質數存起來,那麼接下來的數,如果不能被這些數整除的話,也就是質數了,這個方法會比較快。基本上除了2以外,只要判斷奇數就好了。
	\begin{inside}
		int cnt=1, primes[1024]={2}; // primes[0]=2, 其餘為0
		for (int i=3; cnt<n; i+=2) { // 只看奇數就好
			int isprime=1; // 假設為質數
			for (int k=0; k<cnt; k++) {
				if (i%primes[k]==0) { // 找到i的質因數
					isprime=0; // 找得到因數,所以不是質數
					break;
				}
				if (primes[k]*primes[k]>i) break; // 超過平方根就不用再找了
			}
			if (isprime) { // i為質數
				printf(" %d", i);
				primes[cnt++] = i;
			}
		}
	\end{inside}
\end{enumerate} 

\subsection{程式碼}
\begin{cppcode}
	#include <stdio.h>
	
	int main()
	{
		int n, cnt=1, primes[1024]={2}; // primes[0]=2, 其餘為0
		scanf("%d", &n);
		printf(" 2");
		for (int i=3; cnt<n; i+=2) { // 只看奇數就好
			int isprime=1; // 假設為質數
			for (int k=0; k<cnt; k++) {
				if (i%primes[k]==0) { // 找到i的質因數
					isprime=0; // 找得到因數,所以不是質數
					break;
				}
				if (primes[k]*primes[k]>i) break; // 超過平方根就不用再找了
			}
			if (isprime) { // i為質數
				printf(" %d", i);
				primes[cnt++] = i;
			}
		}
		
		return 0;
	}
	
\end{cppcode}
