\section{A045 - 大樂透包牌}
大樂透從1到49號中選6個號碼,開獎時共開出6個號碼及1個特別號,6個號碼全中者得頭獎。
阿平每期計算明牌,挑出8個號碼,想要將所有由這8個號碼所組的6個號碼全部簽。
請你設計程式供阿平輸入此8個號碼,然後印出所有的簽牌可能。

\subsection{解題思惟}

\begin{enumerate}
\item 將八個數讀到陣列裡。
	\begin{inside}
	int n[8];
	for(int i=0; i<8; i++) {
		scanf("%d", &n[i]);
	}
	\end{inside}
\item 每次剔除兩個數,輸出其他六個數。假設a和b是要剔除的數的索引值,所以只有當輸出索引值i不等於a且不等於b時,才會輸出。
	\begin{inside}
	for(a=0; a<8; a++) {
		for(b=a+1; b<8; b++) {
			for(i=0; i<8; i++) {
				if(i!=a && i!=b) printf("-%d", select[i]);
			}
		printf("\n");
		}
	}
	\end{inside}
\item 本題的輸出數字必須從小排到大,因此輸入數字之後必須先進行排序,排序的方法很多,
此處使用氣泡排序法,程式碼如下:
\begin{inside}
	for (int i=1; i<8; i++) for (int j=0; j<8-i; j++) {
		if (n[j] > n[j+1]) { t=n[j]; n[j]=n[j+1]; n[j+1]=t; }
	}
\end{inside}
如果使用\cc{},也可以引入<algorithm>,並使用std::sort(n, n+8)來進行排序。
\end{enumerate}

\subsection{程式碼}
\begin{cppcode}
#include <cstdio>

int main()
{
	int n[8], t;
	for (int i=0; i<8; i++) scanf("%d", &n[i]);
	for (int i=1; i<8; i++) for (int j=0; j<8-i; j++) {
		if (n[j] > n[j+1]) { t=n[j]; n[j]=n[j+1]; n[j+1]=t; }
	}
	for (int a=0; a<8; a++) {
		for(int b=a+1; b<8; b++) {
			for(int i=0; i<8; i++) {
				if(i!=a && i!=b) printf("-%d", n[i]);
			}
			printf("\n");
		}
	}
	return 0;
}
\end{cppcode}
