\section{A056 遞迴計算最大公因數-計算函式呼叫次數}
遞迴計算最大公因數-計算函式呼叫次數\\輸入兩個正整數xy,輸出依序為函式呼叫次數及最大公因數。\\其gcd函式依照以下準則撰寫: \\
1. 如果 x小於0 改計算 y 與 -x 的最大公因數 \\
2. 如果 x大於y 改計算 y 與 x 的最大公因數 \\
3. 如果 x等於0 回傳y \\
4. 回傳 y%x 與 x 的最大公因數 \\
在過程中,你需要紀錄函式呼叫的次數。

\subsection{解題思維}

\begin{enumerate}
	\item
	首先撰寫gcd遞迴函式:
	\begin{inside}
	int gcd(int x, int y) {
		if(x<0) return gcd(y, -x); 
		//如果 x 小於 0,藉由回傳 (y, -x) 來改計算 y 與 -x 的最大公因數
		if(x>y) return gcd(y, x);
		//如果 x 大於 y,藉由回傳 (y, x) 來改計算 y 與 x 的對大公因數 
		if(x==0) return y; //如果 x 等於 0 回傳 y
		return gcd(y%x, x); //其他情況則回傳 (y%x, x) 來求最大公因數
	}	
	\end{inside}
	\item 要計算gcd函式被呼叫的次數,常用的方式有二,第一個是使用全域變數,第二個是使用區域變數。
	\item 使用全域變數的話,在main函式的上面宣告變數並給啟始值0,之後每次呼叫gcd函式的時候,把該變數加1。
	\begin{inside}
	int cnt = 0;
	int main() {...}
	int gcd(int x, int y) {
		cnt++;
		...
	}
	\end{inside}
	\item 使用區域變數的話,在gcd函式裡面宣告靜態變數並給啟始值0,每次呼叫gcd函式時,把該數加1。靜態變數的啟始值只有第一次會被執行,並且離開函式的時候,它的內容不會消失,因此可以用來計算呼叫次數。要注意的是,該區域變數屬於gcd函式,因此只有在gcd函式中才可以使用此變數,離開該函式便無法引用。
	\begin{inside}
	int gcd(int x, int y) {
		static int cnt = 0;
		cnt++;
		...
	}
	\end{inside}
	\item 如果熟悉指標(pointer),或參照(reference),也可以把計算次數的變數設為函式的參數之一,並用在呼叫函式的時候將該變數加1。
	\begin{inside}
	int gcd(int x, int y, int &c) {
		c++;
		if(x<0) return gcd(y, -x, c);
		if(x>y) return gcd(y, x, c);
		if(x==0) return y;
		return gcd(y%x, x, c);
	}	
	\end{inside}
\end{enumerate} 

\subsection{程式碼}
\begin{cppcode}
	#include <iostream>
	using namespace std;
	
	int cnt = 0;
	int gcd(int x, int y);
	
	int main()
	{
		int x, y;
		cin >> x >> y;
		int ans=gcd(x, y);
		cout << cnt << endl;
		cout << ans;
		return 0;
	}

	int gcd(int x, int y) {
		cnt++;
		if(x<0) return gcd(y, -x);
		if(x>y) return gcd(y, x);
		if(x==0) return y;
		return gcd(y%x, x);
	}	
\end{cppcode}
