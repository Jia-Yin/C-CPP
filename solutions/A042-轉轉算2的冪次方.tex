\section{A042 - 轉轉算2的冪次方}
輸入一個整數n ($1\le n<31$),輸出$2^n$。

\subsection{解題思惟}
當n介於0到31之間時,$2^n$不會溢位,以下幾個思惟角度,都可以用來計算$2^n$。
\begin{enumerate}
\item 可以設ans=1,然後跑n次迴圈,每次將ans乘以2,最後ans會等於$2^n$。	
\begin{inside}
	int ans=1;
	for (int i=0; i<n; i++) ans *= 2;
\end{inside}
\item 也可以把乘以2的運算,改成整數向左移一位的運算,也就是把 ans*=2 改成 ans<{}<=1。
\item 使用移位的想法,$2^n$其實也就是1乘以2乘n次,也等於左移n個位元,所以答案其實是 1<{}<n。
\end{enumerate}

\subsection{程式碼}
這裡使用第一個思惟角度來解答。
\begin{cppcode}
#include <iostream>
using namespace std;

int main() 
{
	int n, ans=1;
	cin >> n;
	for (int i=0; i<n; i++) ans *= 2;
	cout << ans;
	return 0;
}
\end{cppcode}
