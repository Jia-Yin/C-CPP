\section{A043 - 十六進位檢視器}
十六進位檢視器:輸入一段連文字,印出該連文字的十六進位資料。

\subsection{解題思惟}
\begin{enumerate}
\item 因為要輸入一串字元,可以使用字元陣列,假設輸入不含空白,長度不超過1023,可用以下宣告及程式碼:
\begin{inside}
	char doc[1024];
	cin >> doc;
\end{inside}
\item 要針對每一個字元做處理,可以跑一個for迴圈,每次取一個字元,當字元的值不等於0時 (0為字串結束字符),進行該字元的處理。
\begin{inside}
	for (int i=0; doc[i]; i++) { // 第i個字元 doc[i] 不為0時須處理
\end{inside}
\item 每一個字元要轉成16進位符號,因為字元有8個位元,16進位符號有4個位元,所以每個字元會有兩個16進位符號。
\item 左邊的符號是前4個位元,假設字元為ch,可以將ch右移4個位元,這樣就只留下前面的4個位元了。
\begin{inside}
	ch_high = ch>>4; // 去掉後4個位元,只保留前4個位元
\end{inside}
\item 右邊的符號是後4個位元,可以將ch和0xF做位元AND運算得到。實際上0xF也就是二進位中最後4個位元皆為1的數,而任一個位元和位元1做位元AND運算的時候,等於保留該位元,和位元0做位元AND運算的時候,等於去掉該位元。所以和0xF做運算,等於保留最後4個位元。
\begin{inside}
	ch_low = ch & 0xF;
\end{inside}
\item 取得4個位元後,其值為$0\cdots 15$,對應到$0,1,\cdots, 9$, A, B, C, D, E, F,可以使用陣列先儲存16個字符,取索引值印出即可。或者也可以將數值k和10比較,大於等於10的話,將數值k改成字元 `A'+k-10,小於10的話,將數值k改成字元 `0'+k,這樣也可以。以下會使用陣列的方式處理,比較簡明易懂。
\begin{inside}
	char face[16] = {'0','1','2','3','4','5','6','7','8',
					 '9','A','B','C','D','E','F'};
	cout << face[ch_high] << face[ch_low];
\end{inside}
在上式中,因為字元陣列也可以看成是字串,所以還可以簡化如下:
\begin{inside}
	char face[] = "0123456789ABCDEF";
	cout << face[ch_high] << face[ch_low];
\end{inside}
\item 本題也可以使用C語言的printf函式,取格式指定字 \%02X 印出該字元即可。其中X表示大寫16進位形式,02表示佔兩位,不足的話前面補0。
\end{enumerate}

\subsection{程式碼}
使用陣列將數字改成16進位字元:
\begin{cppcode}
#include <iostream>

using namespace std;

int main() {   
	char face[] = "0123456789ABCDEF";
	char doc[1024];
	cin >> doc;
	for (int i=0; doc[i]; i++) {
		int high = doc[i] >> 4;
		int low = doc[i] & 0xF;
		cout << face[high] << face[low] << " ";
	}
	return 0;
}
\end{cppcode}
使用printf函式的版本
\begin{cppcode}
	#include <cstdio>
	
	int main() {   
		char doc[1024];
		scanf("%s", doc);
		for (int i=0; doc[i]; i++) {
			printf("%02X ", doc[i]);
		}
		return 0;
	}
\end{cppcode}
