\section{M90H040 - 複數除法}
輸入兩個複數X和Y,輸出X除以Y的值(四捨五入至小數點後3位)。

\subsection{解題思維}
複數的除法為:$$\frac{a+bi}{c+di}=
\frac{(a+bi)*(c-di)}{(c+di)*(c-di)}=
\frac{(ac+bd)+(-ab+bc)i}{c^2+d^2}=
\frac{ac+bd}{c^2+d^2}+\frac{-ad+bc}{c^2+d^2}i$$
將輸入的數字套入公式即可得到答案,若運算式太複雜可以使用變數暫存。

\subsection{程式碼}
\begin{cppcode}
#include <cstdio>

int main()
{
	float a, b, c, d;
	scanf("%f%f%f%f", &a, &b, &c, &d);
	float denom = c * c + d * d; // 分母
	float re_part = (a * c + b * d) / denom;
	float im_part = (-a * d + b * c) / denom;
	printf("%.3f", re_part);
	if(im_part) printf("%+.3fi", im_part);
	return 0;
}
\end{cppcode}
註:本題在瘋狂程設測試時,沒有辦法通過。仔細檢查,發現實際上輸出並未遵循題目所述,取四捨五入到小數點後三位,而是直接以cout將浮點數輸出。\cc{}預設的浮點數輸出,大約等同於C的\%g指示詞,因此若要通過瘋狂程設的檢查,可將上述程式碼中的\%.3f改成\%g,或者使用cout輸出即可。本題所給的參考答案係以題目所述為主。