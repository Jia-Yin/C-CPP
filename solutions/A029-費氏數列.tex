\section{A029 - 費式數列}
費氏數列定義如下 $f(0)=0, f(1)=1, f(n)=f(n-1)+f(n-2)$;請從螢幕輸入一個正整數n,輸出$f(n)$。

\subsection{解題思惟}

\begin{enumerate}
\item 本題可以使用遞迴或非遞迴方式求解。先討論遞迴寫法。
\item 遞迴函式f(n),如果n<2,回傳n即可,n>=2時,回傳f(n-1)+f(n-2)即可。
\item 如果不使用遞迴的話,可以設fn, fnm1, fnm2三個變數,分別表示f(n), f(n-1)及f(n-2),先設好f(n-1)及f(n-2)的初始值,然後每次計算fn=fnm1+fnm2,算完後令 fnm2=fnm1, fnm1=fn,不斷重覆即可,重覆次數為n-1次。
\end{enumerate}

\subsection{程式碼}
遞迴版本
\begin{cppcode}
#include <cstdio>

int f(int n);

int main()
{
	int n;
	scanf("%d", &n);
	printf("%d", f(n));
	return 0;
}

int f(int n)
{
	if (n<2) return n;
	return f(n-1)+f(n-2);
}
\end{cppcode}
非遞迴版本
\begin{cppcode}
#include <cstdio>

int main()
{
	int n, fn, fnm1=1, fnm2=0;
	scanf("%d", &n);
	if (n<2) fn=n;
	else for (int i=2; i<=n; i++) {
		fn = fnm1 + fnm2;
		fnm2 = fnm1;
		fnm1 = fn;
	}
	printf("%d", fn);
	return 0;
}
\end{cppcode}
