\section{A010 畢氏數}
輸入一個正整數n,輸出所有的 $(a, b, c)$ 滿足
\begin{enumerate}
	\item $a, b, c$為三個正整數
	\item $a<b<c<n$
	\item $a^{2}+b^{2}=c^{2}$
	\item $a, b, c$ 三數的最大公因數為 1。
\end{enumerate}
\subsection{解題思惟}
\begin{enumerate}
	\item 
	先宣告一整數 $n$,接著再輸入 $n$ 值。
	\begin{inside}
		int n;
		cin >> n;
	\end{inside}
	\item
	再來寫一個三層迴圈分別代表 $a, b, c$ 的值,注意此處 $a<b<c<n$。
	\begin{inside}
	int n;
	cin >> n;
	for (int a=1; a<n; a++) {
		for (int b=a+1; b<n; b++) {
			for (int c=b+1; c<n; c++) {
				// ...
			}
		}
	}
	\end{inside}
	\item
	接著判斷是不是符合 $a^{2}+b^{2}=c^{2}$,如果不符合,則迴圈繼續跑。
	\begin{inside}
	for (int a=1; a<n; a++) {
		for (int b=a+1; b<n; b++) {
			for (int c=b+1; c<n; c++) {
				if (a*a+b*b != c*c) continue;	
			}
		}
	}
	\end{inside}
	\item 因為要判斷三個數的最大公因數是否為1,所以寫一個計算最大公因數的函式。
	\begin{inside}
	int gcd(int x, int y) {
		while (x%=y) swap(x, y); // 如果 x%y>0,交換 x, y 值
		return y;				
	}
	\end{inside}
	上面函式可以用來計算兩個數的最大公因數。如果要計算三個數$a, b, c$的最大公因數是否為1,
	可以使用 gcd(gcd($a, b$), $c$)==1 來判斷。
	\item 求$a, b, c$三數的最大公因數,也可以使用搜尋的方式,從$a$往下一直到1,看什麼時候可以找到$a, b, c$的公因數,如果一直到1才找到,那三個數的最大公因數就是1。
	\begin{inside}
		for (k=a; k>0; k--) if (a%k==0 && b%k==0 && c%k==0) break;
		if (k==1) ... // a, b, c 的最大公因數為1
	\end{inside}
\end{enumerate} 

\subsection{程式碼}
\begin{cppcode}
	#include <iostream>
	using namespace std;
	
	int gcd(int x, int y);
	
	int main() 
	{
		int n;
		cin >> n;
		for (int a=1; a<n; a++) {
			for (int b=a+1; b<n; b++) {
				for (int c=b+1; c<n; c++) {
					if (a*a+b*b != c*c) continue;
					if (gcd(gcd(a, b), c) != 1) continue;
					cout << a << "," << b << "," << c << endl;	
				}
			}
		}
		return 0;
	}

	int gcd(int x, int y) {
		while(x%=y) swap(x, y); // 如果 x%y>0,交換 x, y 值
		return y;				
	}
\end{cppcode}
