\section{A028 - 十進位轉七進位}
輸入一個正整數N,輸出該數的七進位表達式。

\subsection{解題思惟}
\begin{enumerate}
	\item 十進位整數可以直接讀取。轉成七進位時,一般作法是把該數除以7,得到的餘數為最末位,商的部份繼續除以7,得到的餘數為次末位,依此類推,至不能再除為止。
	\item 因為得到的位數是從個位開始到最高位,但輸出是要倒過來,所以要先把得到的值存起來,最後倒過來輸出。存起來的方式可以使用陣列。
	\item 本題也可以從最高位算起,但是要先計算最高位的權重(即7的最高次冪)。計算的方式,是從權重1開始,每次把權重乘以7,一直到大於該數大小之前,此即為最高位。
	\item 設最高位權重為d,則該數n除以d即得到最高位數,其次將d除以7,得到次高位權重,將n除以該權重,然後再求7的餘數得次高位數,依此類推。使用本方法計算的話,可以逐位輸出,不需要先儲存每一位的數字。
\end{enumerate}

\subsection{程式碼}
一般解法
\begin{cppcode}
#include <stdio.h>

int main()
{
	int n, a[20], idx=0;
	scanf("%d", &n);
	while (n) { a[idx++]=n%7; n/=7; }
	for (n=idx-1; n>=0; n--) printf("%d", a[n]);
	return 0;
}
\end{cppcode}
	
由高位起首解法
\begin{cppcode}
#include <stdio.h>

int main()
{
	int n, d=1;
	scanf("%d", &n);
	while (d<=n) d*=7;
	for (d/=7; d; d/=7) printf("%d", n/d%7);
	return 0;
}    
\end{cppcode}
