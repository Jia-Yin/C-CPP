\section{A005 - 一元二次方根}
從命令視窗輸入一元二次方程式 $ax^{2}+bx+c=0$ 的三個係數$a, b, c$,若該一元二次方程式有解,請印出它的兩個解,若無解則印出 ``No"。請使用 double 做為係數及解的型態。另外兩個解的公式為 $\displaystyle\frac{-b\pm\sqrt{d}}{2a}$,其中 $d=b^2-4ac$。輸出請四捨五入到小數點後三位。

\subsection{解題思維}
\begin{enumerate}
	\item 因為求解會用到開根號函式sqrt,所以我們也要引入math函式庫。
	\item 輸出要四捨五入到小數點三位,這部份使用printf函式會比較簡單一點,所以本題優先考慮使用C的輸入輸出函式。
	\item 使用scanf輸入double的時候,格式指定字要使用\%lf。\%f是給float使用的。
	\item 一般來說,printf輸出浮點數的時候,都會先轉成double型態,格式指定字是使用\%f。在C99之後的標準,為了和scanf一致,也可以使用\%lf,\%f和\%lf的意義對printf來說是相同的。要四捨五入並輸出至小數點後三位,格式指定字用\%.3f。
	\item 最後根據判斷式 $d=b^2-4ac$ 是否小於0決定程式分支即可。
\end{enumerate}

\subsection{程式碼}
\begin{cppcode}
	#include <cstdio>
	#include <cmath>
	
	int main()
	{
		double a, b, c, d;
		scanf("%lf%lf%lf", &a, &b, &c);
		d = b*b - 4*a*c;
		if (d>=0) {
			printf("%.3f\n%.3f", (-b+sqrt(d))/(2*a),
					(-b-sqrt(d))/(2*a));
		} else printf("No");
		return 0;
	}
\end{cppcode}
註:本題在瘋狂程設測試時,沒有辦法通過。仔細檢查,發現實際上輸出並未遵循題目所述,取四捨五入到小數點後三位,而是直接以cout將浮點數輸出。\cc{}預設的浮點數輸出,大約等同於C的\%g指示詞,因此若要通過瘋狂程設的檢查,可將上述程式碼中的\%.3f改成\%g,或者使用cout輸出即可。本題所給的參考答案係以題目所述為主。
