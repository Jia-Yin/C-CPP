\section{F033 - 天干地支(列舉)}
一個正整數對應之天干地支,依其除以10的餘數輸出天干"甲乙丙丁戊己庚辛壬癸"中之一字。甲(0)乙(1)丙(2)丁(3)戊(4)己(5)庚(6)辛(7)壬(8)癸(9) 依其除以12的餘數輸出地支"子丑寅卯辰巳午未申酉戌亥"之一字。子(0)丑(1)寅(2)卯(3)辰(4)巳(5)午(6)未(7)申(8)酉(9)戌(10)亥(11) 例如: 0>甲子 1>乙丑 2>丙寅 .... 現在請你列出六十個天干地支變化。
\subsection{解題思惟}
本題與前一關的天干地支(數字)那題類似,不同的是那一題只針對一個數字輸出答案,
本題是要把所有六十個變化印出來。把輸入的部份改成for迴圈即可。
\subsection{程式碼}
\begin{cppcode}
#include <iostream>

using namespace std;

int main()
{
	char T[] = "甲乙丙丁戊己庚辛壬癸";
	char D[] = "子丑寅卯辰巳午未申酉戌亥";
	int t, d;
	for (int n=0; n<60; n++) {
		t=2*(n%10), d=2*(n%12); // 計算天干地支字元開始的位置
		if (n%10==0) cout << endl;
		cout << " " << T[t] << T[t+1] << D[d] << D[d+1];
	}
	return 0;
}
\end{cppcode}
