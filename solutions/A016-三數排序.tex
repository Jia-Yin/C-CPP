\section{A016 - 三數排序}
輸入三個正整數 a、b、c,將 a、b、c 從小排到大並輸出。
\subsection{解題思維}
\begin{enumerate}
\item a,b,c三數排序時,可以先比a和b,如果a>b則交換兩個數,使a<b,之後再比b和c,順序不對就交換,使b<c,此時c為最大值。最後再比較和調整一次a和b即可。
	\begin{inside}
	if (a>b) swap(a, b);
	if (b>c) swap(b, c);
	if (a>b) swap(a, b);
	\end{inside}
\item 交換兩數x和y,在\cc{}中可以直接使用swap函式,如果是在C裡面,則常用的方法是宣告另一個暫存變數t,然後使用以下敘述:
	\begin{inside}
		t=x; x=y; y=t;
	\end{inside}
\item 其實排序有很多種方法,以上的排序法稱為氣泡排序法,它的基本原理是兩兩相比,順序不對就互相交換,這樣依序比過一輪,最大的就跑到最右邊了。之後再比第二輪,可以把次大的送到右邊第二位,依此類推。還不明白的讀者可自行上網查詢,或查看一般程式書籍,以進一步了解其原理和作法。
\end{enumerate}

\subsection{程式碼}
\begin{cppcode}
	#include <iostream>

	using namespace std;
	
	int main()
	{
		int a, b, c;
		cin >> a >> b >> c;
		if (a>b) swap(a, b);
		if (b>c) swap(b, c);
		if (a>b) swap(a, b);
		cout << a << " " << b << " " << c;
		return 0;
	}
\end{cppcode}

