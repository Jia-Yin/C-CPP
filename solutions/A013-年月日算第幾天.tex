\section{A013 - 年月日算第幾天}
輸入西元年月日,算該日是該年的第幾天。

\subsection{解題思惟}
\begin{enumerate}
	\item 將每月的天數存在陣列中。陣列可宣告13個,第0個元素不用,第1到12個元素分別表示1月到12月的天數,這樣使用起來會比較便利。
	\item 先計算該年為平年或閏年。基本上每4年閏一次,但每100年少一次,每400年又要加回一次。可用以下方式檢查:
	\begin{inside}
	if (y%4==0) leap = true;
	if (y%100==0) leap = false;
	if (y%400==0) leap = true;
	\end{inside}
	或者也可以使用以下方式:
	\begin{inside}
	leap = false;
	if (y%400==0) leap = true;
	else if (y%100!=0 && y%4==0) leap = true;
	\end{inside}
	\item 2月的天數先以28天計算,如果是閏年的話,把它加1。
	\item 假設輸入的月和日分別為m和d,將1至m-1月的天數加起來,再加上d就是答案了。
	\item 本題如果不用陣列的話,也可以使用switch來解題,第m月的天數可用以下程式碼決定:
	\begin{inside}
	switch (m) {
	case 1:case 3:case 5:case 7:case 8:case 10:case 12: days=31; break;
	case 4:case 6:case 9:case 11: days=30; break;
	case 2: if(leap) days=29; else days=28;
	}
	\end{inside}
\end{enumerate}

\subsection{程式碼}
\begin{cppcode}
#include <iostream>

using namespace std;

int main()
{
	int month[13]={0,31,28,31,30,31,30,31,31,30,31,30,31};
	int yy, mm, dd, i, day=0;
	cin >> yy >> mm >> dd;
	
	if (yy%400==0) month[2]++;
	else if (yy%100!=0 && yy%4==0) month[2]++;
	for (i=1; i<mm; i++) day += month[i];
	day += dd;
	cout << day;
	return 0;
}
\end{cppcode}
