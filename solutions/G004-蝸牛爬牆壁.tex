\section{G004 - 蝸牛爬牆壁}
資訊學院牆壁高 a 公尺,蝸牛從 b 公尺高的地方往上爬,白天可以往上爬 c 公尺,但是晚上睡著了會下降 d 公尺,請問蝸牛幾天後可以爬上屋頂?
\subsection{解題思惟}
\begin{enumerate}
	\item 這一題可以有兩類解法,第一種依題意設立變數a, b, c, d與天數,然後使用迴圈和題目規則去計算所需的天數;第二種則是使用代數的推導去計算所需的天數。
	\item 第一種的作法比較直接,假設有a,b,c,d, 及day=0,可跑以下迴圈計算所需的天數:
	\begin{inside}
		while (b<a) { // 還沒爬到
			day++; b+=c; // 多一天,並且可以爬到高度 b=b+c
			if (b>=a) break; // 已經爬到了,跳出迴圈
			b -= d; // 還沒到的話,晚上會掉下 d 公尺
		}
	\end{inside}
	\item 第二種解法,假設所需天數為x,則x-1天白天可以到達的高度為 b+(c-d)(x-2)+c,這個值應該小於a;而第x天白天可以到達的高度為 b+(c-d)(x-1)+c,這個高度應該大於等於a,故可以得到以下的不等式:
	$$ b+(c-d)(x-2)+c < a \le b+(c-d)(x-1)+c$$
	稍作整理可以得到以下的不等式:
	$$ \frac{a-b-c}{c-d}+1 \le x < \frac{a-b-c}{c-d}+2$$
	換句話說,滿足$x\ge \frac{a-b-c}{c-d}+1$的最小整數$x$就是答案。如果m和n都是整數,要找整數$x\ge\frac{m}{n}$,因為在C/\cc{}語言中,整數除以整數只取整數,所以稍加思考,可以得到最小整數$x=\frac{m+n-1}{n}$(思考一下為什麼?),所以這一題的答案為$x=\frac{a-b-d-1}{c-d}+1$。讀者可以實際驗證看看是否正確。
\end{enumerate}
\subsection{程式碼}
\begin{cppcode}
#include <stdio.h>
int main()
{
	int a, b, c, d, day=0;
	scanf("%d%d%d%d", &a, &b, &c, &d);
	while (b<a) {
		day++; b+=c;
		if (b>=a) break;
		b-=d;
	}
	printf("%d", day);
	return 0;
}
\end{cppcode}
