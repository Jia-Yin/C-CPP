\section{F002 - 溢位數 int}
求最大整數及其加1之後的值。

\subsection{解題思維}
\begin{enumerate}
	\item 一般而言,int型態使用32個位元,其最大值是$2^{31}-1$ = 0X7FFFFFFF,如果再往上加的話會溢位,變成 $-2^{31}$ = 0X80000000,也是整數型態的最小值。不過int型態使用32個位元並不是絕對的,有一些系統的編譯器並不是這樣。那比較一般的情況,我們可以使用while迴圈將正整數1一直往上加,直到變成負數前的那個整數就是最大的整數。
	\begin{inside}
		int n = 1;
		while (n>0 && n+1>0) n++;
	\end{inside}
	\item 或者,我們也可以讓while迴圈將正整數1一直往上加,直到變成負數後跳出來,然後再減掉1也可以。
	\begin{inside}
		int n = 1;
		while (n>0) n++;
		n--;
	\end{inside}
	使用以上兩個方法解題的時候,應注意可能會有超時的問題。
	\item 另外,我們也可以把正整數1每次乘以2,這也等於將其二進位值往左移一位。一直乘以2或者移位,直到變成負數為止,那就表示溢位了,然後再將其減掉1也可以。
	\begin{inside}
		int n = 1;
		while (n>0) n <<= 1; // 或 n *= 2;
		n--;
	\end{inside}
\end{enumerate}

\subsection{程式碼}
以上各種方法,都可以用來求解本題的答案,但要注意可能有超時的問題。下面的程式碼是以最後的方式來撰寫的。
\begin{cppcode}
#include <iostream>

using namespace std;

int main()
{
	int n = 1;
	while (n>0) n <<= 1; // 或 n *= 2;
	n--;
	cout << n << endl << n+1;
	return 0;
}
\end{cppcode}
