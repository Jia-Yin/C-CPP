\section{M90H021 - 最大公因數(3數)}
輸入3整數,輸出其最大公因數。
\subsection{解題思惟}
\begin{enumerate}
	\item 求解最大公因數有多種方法,在前面的題目中(A007, A022)針對兩個數的最大公因數已曾加以探討。至於求解三個數的最大公因數,可以在求得兩數的
	最大公因數之後,再用此數和第三個數求一次最大公因數即可。
	\item 或者最簡單直覺的辦法是跑一個for迴圈,從任一個數的絕對值開始往下跑到1,第一個能夠
	整除三個數的就是最大公因數。
\end{enumerate}
\subsection{程式碼}
\begin{cppcode}
#include <iostream>

using namespace std;

int main()
{
	int a, b, c, n;
	cin >> a >> b >> c;
	if (a<0) a = -a;
	for (n=a; n>0; n--) {
		if (a%n==0 && b%n==0 && c%n==0) break;
	}
	cout << n;
	return 0;
}
\end{cppcode}
