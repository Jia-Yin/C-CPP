\section{A033 巴斯卡三角形}
一個正數N(N<20),列出前N列巴斯卡三角形。

\subsection{解題思維}

\begin{enumerate}
	\item 這一題希望印出的巴斯卡三角形形式如下:
	\begin{inside}
	1
	1 1
	1 2 1
	1 3 3 1
	...
	\end{inside}
	\item 可以使用一維陣列或二維陣列來處理這個問題,也可以使用遞迴函式來處理,依次說明如下。
	\item 使用一維陣列的話,有幾個觀察點:1)第i列有i個元素;2)第i列最前和最後都是1,其它第k個元素值為上一列第k-1和第k個元素值的和。假設我們知道第i列的所有元素,可以據此來計算第i+1列的所有元素。
	\item 如果我們把第i列最左邊的1前面再補一個0,這樣的話第i列的索引位置依次為0,1,...i,那我們可以用以下的方式來計算第i+1列的元素:把i+1的位置設為1(最右邊為1),接著把i的位置的值加上i-1的位置的值,接著把i-1的位置的值加上i-2位置的值,依次進行一直到位置1為止。程式如下:
	\begin{inside}
		// 從第i列計算第i+1列的值
		bsk[i+1] = 1;
		for (int k=i; k>0; k--) bsk[k] += bsk[k-1];
	\end{inside}
	\item 使用二維陣列來解答的話,基本概念與上面類似,不同的是第i列和第i+1列存在不同的陣列索引位置中,其關係為bsk[i][j]=bsk[i-1][j-1]+bsk[i-1][j]。適當設定啟始值 (左右邊為1,超出範圍為0),用迴圈即可處理。程式碼如下:
	\begin{inside}
		// 從第i列計算第i+1列的值
		bsk[i+1][0] = bsk[i+1][i] = 1; // 頭尾為1
		for (int j=1; j<i; j++) {
			bsk[i+1][j] = bsk[i][j] + bsk[i][j-1]; // 中間值為上面和上左的和
		}
	\end{inside}
	\item 使用遞迴處理的話,假設第i列第j個位置的值為bsk(i,j),則bsk(i,j)函式可以如下定義:
	\begin{inside}
		int bsk(int i, int j)
		{
			if (j==0 || j==i) return 1; // 終止條件一:左右邊界
			if (j>i) return 0; // 終止條件二:超出邊界
			return bsk(i-1, j) + bsk(i-1, j-1);
		}
	\end{inside}
	\item 以下分別提供三種參考解答如下。
\end{enumerate} 

\subsection{程式碼}
使用一維陣列解答:
\begin{cppcode}
#include <stdio.h>

int main()
{
	int n, bsk[20]={0};
	scanf("%d", &n);
	for (int i=0; i<=n; i++) { // 每一列跑一次迴圈
		if (i==0) bsk[1] = 1; // 第一列直接設定第1個數為1
		else { // 從第i列計算第i+1列的值
			bsk[i+1] = 1;
			for (int k=i; k>0; k--) bsk[k] += bsk[k-1];
		}
		for (int k=1; k<=i+1; k++) printf(" %d", bsk[k]);
		printf("\n");
	}
	
	return 0;
}
\end{cppcode}
使用二維陣列解答:
\begin{cppcode}
	#include <stdio.h>
	
	int main()
	{
		int n, bsk[20][20]={{0}};
		scanf("%d", &n);
		
		for (int i=0; i<=n; i++) bsk[i][0] = bsk[i][i] = 1;
		
		for (int i=2; i<=n; i++) { // 前兩列不用再處理
			for (int j=1; j<i; j++) {
				bsk[i][j] = bsk[i-1][j-1] + bsk[i-1][j];
			}
		}
		
		for (int i=0; i<=n; i++) {
			for (int j=0; j<=i; j++) {
				printf(" %d", bsk[i][j]);
			}
			printf("\n");
		}
		
		return 0;
	}
\end{cppcode}
使用遞迴處理:
\begin{cppcode}
#include <stdio.h>

int bsk(int i, int j);

int main()
{
	int n;
	scanf("%d", &n);
	
	for (int i=0; i<=n; i++) {
		for (int j=0; j<=i; j++) printf(" %d", bsk(i,j));
		printf("\n");
	}
	return 0;
}

int bsk(int i, int j)
{
	if (j==0 || j==i) return 1; // 終止條件一:左右邊界
	if (j>i) return 0; // 終止條件二:超出邊界
	return bsk(i-1, j) + bsk(i-1, j-1);
}
\end{cppcode}