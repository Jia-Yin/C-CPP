\section{M90H016 有k個正因數之最小數為何}
輸入一整數k,至少有k個正因數之最小正整數。
\subsection{解題思維}

\begin{enumerate}
	\item
	我們先寫一個 factors 函式,來計算某數k有多少個因數。基本上就是從1到k跑一個迴圈,看看有幾個數是k的正因數。
	\begin{inside}
	int factors(int k) {
		int n=0;
		for (int i=1; i<=k; i++) {
			if (k%i==0) n++;
		}
		return n;
	}
	\end{inside}
	\item
	完成 factors 函式後,接著就可以在主程式跑一個迴圈,找看看哪一個數的因數個數大於等於 k。
\end{enumerate} 

\subsection{程式碼}
\begin{cppcode}
	#include <iostream>

	using namespace std;
	
	int factors(int k);
	
	int main()
	{
		int k;
		cin >> k;
		for (int i=1; ; i++) {
			if (factors(i) >= k) {
				cout << i;
				break;
			}
		}
		return 0;
	}
	
	int factors(int k) {
		int n=0;
		for (int i=1; i<=k; i++) {
			if (k%i==0) n++;
		}
		return n;
	}
\end{cppcode}
