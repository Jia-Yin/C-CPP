\section{M90H007 - 考試調分}
在一次程設小考中,同學成績表現不好,老師決定採線性調整分數,將最低分調成60分,將最高分調成100分,所有同學調整後的分數採四捨五入進整數。輸入60個同學的成績,輸出同學調整後的成績。
\subsection{解題思惟}
先找出最高以及最低的成績,最高分會變成100,最低分變成60,因為是線性變換,可以算出新成績與舊成績的轉換公式如下:
$$
\mbox{新成績} = 60 + \mbox{(舊成績$-$最低分)}*{\frac{100-60}{\mbox{最高分$-$最低分}}}
$$
另外本題最後輸出成績會使用到四捨五入,使用printf函式會比較方便,故採用C的輸入輸出函式。
\begin{enumerate}
\item 先取得每位同學的成績,因為有60位同學,所以必須使用陣列。
\begin{inside}
	int score[60]; 
	for (int i=0; i<60; i++) scanf("%d", &score[i]);
\end{inside}
\item 找出最高分與最低分,因最高分不低於0分,可設smax=0,之後每次讀入一個分數的時候,檢查是否高於smax,如果高於smax,則將smax替換為輸入值。又最低分不高於100分,可設smin=100,之後每次讀入一個分數的時候,檢查是否低於smin,如果低於smin,則將smin替換為輸入值。結合第1點,改寫如下:
	\begin{inside}
	int score[60], smin = 100, smax = 0;
	for (int i=0; i<60; i++) {
		scanf("%d", &score[i]);
		if (score[i] > smax) smax = score[i];
		if (score[i] < smin) smin = score[i];
	}
	\end{inside}
\item 取得最高分及最低分之後,先計算線性變換的比值。須注意比值為浮點數,故計算比值公式的時候,算式中不可全為整數,否則會被當作整數運算,而捨棄小數位。基本上在上述公式中,將40改成40.0即可。
	\begin{inside}
	float ratio = 40.0 / (smax - smin); // 40.0 為浮點數,不可寫成 40
	\end{inside}
\item 接著依線性變換公式計算每位同學的新成績,並將其印出來,印出值需依四捨五入方式取至整數位,可使用printf函式,搭配 \%.0f 的指示詞,表示輸出浮點數時小數點取0位,且須要四捨五入。
	\begin{inside}
	for (int i=0; i<60; i++) {
		float newScore = 60 + (score[i]-smin) * ratio; // 計算新成績
		printf("%.0f\n", newScore); // 四捨五入,小數點取0位
	}
	\end{inside}
\end{enumerate}

\subsection{程式碼}
\begin{cppcode}
#include <cstdio>

int main()
{
	int score[60], smin = 100, smax = 0;

	for (int i=0; i<60; i++) {
		scanf("%d", &score[i]);
		if (score[i] > smax) smax = score[i];
		if (score[i] < smin) smin = score[i];
	}
	
	float ratio = 40.0 / (smax - smin); // 比值
	
	for (int i=0; i<60; i++) {
		float newScore = 60 + (score[i]-smin) * ratio; // 計算新成績
		printf("%.0f\n", newScore); // 四捨五入
	}
	return 0;
}
\end{cppcode}
