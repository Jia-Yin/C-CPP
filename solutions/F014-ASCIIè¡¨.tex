\section{F014 - ASCII表}
輸出 32 到 127 號 ASCII 字元,每16個字換一行。
\subsection{解題思惟}
\begin{enumerate}
\item 輸出字元,在C中使用printf及\%c指示詞。
如果是在\cc{}中,針對字元c,使用cout<<c,
如果是整數i,則先轉成字元再輸出:cout<<(char)i。
\item 本題在輸出時,每16個字換一行,並且觀察輸出,最後一排輸出完畢不用換行。
假設輸出的ASCII碼為i,當i為16的倍數時,印出字元前要先換行,另外第一次印出(i=32)不用換行。
可用以下程式碼:
\begin{inside}
if (i!=32 && i%16==0) cout << endl;
\end{inside}
\end{enumerate}

\subsection{程式碼}
\begin{cppcode}
#include <iostream>

using namespace std;

int main()
{
	for(int i=32; i<128; i++)
	{
		if (i!=32 && i%16==0) cout << endl;
		cout << char(i);
	}
}
\end{cppcode}
