\section{M90H010 - $2^{n}$ 個位數疊加}
輸入一整數n,計算 $2^{1}+2^{2}+2^{3}+...+2^{n}$ 之個位數。


\subsection{解題思維}
\begin{enumerate}
\item 計算 $2^k$ 的方法,可以直接使用 $1<<k$,這個表達式的意義,是將1向左移k個位元,由於在二進位中,左移一個位元,等於是乘以2,所以左移k個位元,等於計算$2^k$。
\item 但這裡應該要注意,當k比較大的時候,會產生溢位的問題,所以不能單純使用這個方法來計算。由於最後要計算的是除以10的餘數,所以計算$2^k$比較保險的方法,是比照次方求餘那題的方法,計算k次迴圈,每次乘以2,並且只保留除以10的餘數,這樣就可以避免掉溢位的問題。
	\begin{inside}
		s = 1;
		for (int i=0; i<k; i++) {
			s *= 2;
			s %= 10;
		}
	\end{inside}
\item 如果仔細觀察,可以發現上面的程式碼,當每k變成k+1的時候,其實前面k次的計算是多餘的,也就是說在算k+1的時候,從第k次的結果再乘一次2就可以了。這樣的話,整個計算的程式碼可以寫成下面的方式:
	\begin{inside}
		sum = 0;
		s = 1;
		for (int k=1; k<=n; k++) {
			s *= 2;
			s %= 10;
			sum += s;
		}
		cout << sum % 10;
	\end{inside}
\item 另外我們也可以用等比級數求和的方式,算出上式的答案,其實就是 $2^{n+1}-2$,那麼我們只要算出$2^{n+1}$的個位減去2就可以了。
	\begin{inside}
		s = 1;
		for (int k=1; k<=n+1; k++) {
			s *= 2;
			s %= 10;
		}
		cout << (s-2)%10;
	\end{inside}
\end{enumerate}

\subsection{程式碼}
\begin{cppcode}
	#include <iostream>

	using namespace std;
	
	int main()
	{
		int n, s=1, sum=0;
		cin >> n;
		for (int k=1; k<=n; k++) {
			s *= 2;
			s %= 10;
			sum += s;
		}
		cout << sum % 10;
		return 0;
	}
\end{cppcode}

\subsection{延伸思考}
聰明的讀者應該可以發現,$2^k$除以10的餘數,也就是個位數,其實是有週期性的,當k從1開始往上遞增的時候,其個位數依次為2, 4, 8, 6, 2, 4, 8, 6,...,很容易可以發現其週期為4,所以也可以直接計算k除以4的餘數,再找出對應的個位數字即可。另外$2^{1}+2^{2}+2^{3}+...+2^{n}$是一個等比級數的和,我們也可以很容易計算出其值為$2^{n+1}-2$,使用求和公式搭配上面的週期性質,可以看出來,當n從1開始往上遞增的時候,答案其實也是週期性的,每4次一個循環,並且前4個數分別為$2^2-2=2$, $2^3-2=6$, $2^4-2=14$, 以及$2^5-2=30$的個位數,也就是2, 6, 4, 0,根據這樣的規則,本題只要求出n除以4的餘數,就可以馬上知道答案了。
\begin{cppcode}
	#include <iostream>
	
	using namespace std;
	
	int main()
	{
		int n, ans[]={0, 2, 6, 4};
		cin >> n;
		cout << ans[n%4];
		return 0;
	}
\end{cppcode}
