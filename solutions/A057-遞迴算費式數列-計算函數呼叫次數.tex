\section{A057 遞迴算費式數列-計算函式呼叫次數}
費氏數列 f(0)=0, f(1)=1, f(n)=f(n-1)+f(n-2);請依照此定義計算費氏數列,並計算函式呼叫次數。本題輸入為一個正整數 n ,輸出為函式呼叫次數及 f(n)。

\subsection{解題思維}

\begin{enumerate}
	\item
	首先先寫費氏數列的遞迴函式 f(n)。
	\begin{inside}
	int fn(int n) {
		if(n<2) return n; // n為0回傳0,n為1回傳1
		return fn(n-1)+fn(n-2); //其他情況則回傳 fn(n-1)+fn(n-2)
	}	
	\end{inside}
	\item
	計算函式呼叫次數的方式,如上一題A056的方式處理即可。
\end{enumerate} 

\subsection{程式碼}
\begin{cppcode}
	#include <iostream>
	using namespace std;

	int cnt = 0;	
	int fn(int n);

	int main()
	{
		int n;
		cin >> n;
		int ans = fn(n);
		cout << cnt << endl;
		cout << "f(n)=" << ans;
		return 0;
	}
	
	int fn(int n) {
		if(n==0) return 0;
		if(n==1) return 1;
		return fn(n-1)+fn(n-2);
	}
\end{cppcode}
