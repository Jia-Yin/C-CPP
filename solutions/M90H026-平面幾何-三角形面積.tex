\section{M90H026 平面幾何-三角形面積}
輸入三角形三個點(ax, ay), (bx, by), (cx, cy),輸出其面積。
\subsection{解題思維}
\begin{enumerate}
	\item 已知三角形三點座標,要求此三角形面積的方法,可用向量外積方式簡略推導,如果讀者不能明瞭,可以查看一般教科書或網路上查詢其他說明。
	\item 假設兩個邊向量分別為$(a,b),\ (c,d)$,可以將其看成兩個三維向量
	$(a,b,0)$及$(c,d,0)$,其外積為
	$$\left|\begin{array}{ccc}
	\vec{i} & \vec{j} & \vec{k} \\
	a & b & 0 \\
	c & d & 0 \\
	\end{array}\right| = (ad-bc)\ \vec{k}$$
	因為外積為兩個向量形成的平行四邊形面積,所以兩個邊所形成的三角形面積會變成$|ad-bc|/2$。
	\item 已知三角形三個點,可以求兩個邊向量,再用上面的公式求面積。假設三點座標為(ax, ay), (bx, by), (cx, cy),則兩個向量邊為(bx-ax, by-ay)以及(cx-ax, cy-ay),故面積為0.5*abs((bx-ax)*(cy-ay)-(cx-ax)*(by-ay)),撰寫函式如下:
	\begin{inside}
	double tri(int ax, int ay, int bx, int by, int cx, int cy) {
		return 0.5*abs((bx-ax)*(cy-ay)-(by-ay)*(cx-ax));
	}		
	\end{inside}
	\item
	完成後再代入主程式。
\end{enumerate} 

\subsection{程式碼}
\begin{cppcode}
	#include <iostream>

	using namespace std;
	
	double tri(int ax, int ay, int bx, int by, int cx, int cy);
	
	int main(){
		int ax, ay, bx, by, cx, cy;
		cin >> ax >> ay >> bx >> by >> cx >> cy;
		cout << tri(ax, ay, bx, by, cx, cy);
		return 0;
	}
	
	double tri(int ax, int ay, int bx, int by, int cx, int cy) {
		return 0.5*abs((bx-ax)*(cy-ay)-(by-ay)*(cx-ax));
	}		
\end{cppcode}
