\section{A025 - 判斷閏年}
輸入西元年,如果該年是閏年,則輸出Yes,若該年不是閏年,則輸出No。 (閏年的定義為,四年一閏,逢百不閏,逢四百又閏。例如西元1004年為閏年,西元1100年不是閏年,西元1600年是閏年)

\subsection{解題思維}
這一題可以使用巢狀的if-else依規則次序進行判斷,或者也可以使用複合的邏輯判斷式進行。方法有很多種,以下舉出三個方法供作參考。
\subsection{程式碼}
從400的倍數,100的倍數,4的倍數依次判斷
\begin{cppcode}
	#include <iostream>
	
	using namespace std;
	
	int main()
	{
		int year;
		cin >> year;
		if (year%400==0) cout << "Yes"; // 400的倍數
		else if (year%100==0) cout << "No"; // 否,但是100的倍數
		else if (year%4==0) cout << "Yes"; // 否,但是4的倍數
		else cout << "No";
		return 0;
	}
\end{cppcode}

\noindent 
從4的倍數,100的倍數,400的倍數依次判斷,這邊是判斷非其倍數為主。要判斷一數n不是k的倍數,可以寫 if (n\%k != 0),但是運算式不為0本來就代表true,所以也可以簡寫成 if (n\%k)。
\begin{cppcode}
	#include <iostream>
	
	using namespace std;
	
	int main()
	{
		int year;
		cin >> year;
		if (year%4) cout << "No"; // 不是4的倍數
		else if (year%100) cout << "Yes"; // 是,但不是100的倍數
		else if (year%400) cout << "No"; // 是,但不是400的倍數
		else cout << "Yes"; // 最後表示是400的倍數
		return 0;
	}
\end{cppcode}

\noindent 使用複合的邏輯判斷式
\begin{cppcode}
	#include <iostream>
	
	using namespace std;
	
	int main()
	{
		int year;
		cin >> year;
		// 直接判斷是否為400的倍數,或者是4的倍數但不是100的倍數
		if (year%400==0 || (year%4==0 && year%100)) cout << "Yes";
		else cout << "No";
		return 0;
	}
\end{cppcode}
