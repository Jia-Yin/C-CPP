\section{A017 - 三角形種類}
輸入為三角形三邊長(正整數) a, b, c,輸出三角形類型(英文):isosceles triangle (等腰三角形), not a triangle (不成三角形), regular triangle (正三角形), rectangular triangle (直角三角形), obtuse triangle (鈍角三角形), acute triangle (銳角三角形), isosceles right triangle (等腰直角三角形)。

\subsection{解題思維}
\begin{enumerate}
\item 先將三角形的三個邊a, b, c進行排序,這樣後面要判斷是哪一種三角形會比較方便。
\item 接著使用巢狀的if-else進行分支判斷,判斷的順序,可以有很多方式。基本原則是找出一個條件和其他所有的條件都相反的,把它當作if的條件式,其餘的通通歸到else那一邊,然後在else的區塊裡,再繼續使用if-else細分出其他的各種種類。
\end{enumerate}

\subsection{程式碼}
\begin{cppcode}
	#include <iostream>
	
	using namespace std;
	
	int main()
	{
		int a, b, c;
		cin >> a >> b >> c;
		if(a>b) swap(a, b);
		if(b>c) swap(b, c);
		if(a>b) swap(a, b);
		
		// 三角形種類
		if(a+b<=c) { // 兩小邊和小於等於大邊 => 不成三角形
			cout << "not a triangle";
		} else if(a==c) { // 最大和最小相等,表示三邊都相等 => 正三角形
			cout << "regular triangle"; 
		} else if((a==b) && (a*a+b*b==c*c)) { // 小邊相等,平方和等於大邊平方
			cout << "isosceles righttriangle"; // => 等腰直角三角形
		} else if(a==b || b==c) { // 任兩邊相等 => 等腰三角形
			cout << "isosceles triangle";
		} else if(a*a+b*b==c*c) { // 兩小邊平方和等於大邊平方
			cout << "rectangular triangle"; // => 直角三角形
		} else if(a*a+b*b<c*c) { // 兩小邊平方和小於大邊平方
			cout << "obtuse triangle"; // => 鈍角三角形
		} else cout << "acute triangle"; // => 最後剩下銳角三角形
		return 0;
	}
\end{cppcode}
